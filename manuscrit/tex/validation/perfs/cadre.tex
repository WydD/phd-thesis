\section{Cadre d'expérimentation}\label{sec:valid:perfs:cadre}
Dans cette partie, nous présentons notre cadre expérimental. L'ensemble de la distribution Astronef-Asteroid est sous forme de \textit{bundles Java-OSGi}. Ainsi, ces prototypes peuvent être déployés sur toute plateforme possédant la technologie \textit{Java}. De plus, l'environnement \textit{OSGi} nous permet d'exploiter un environnement modulaire basé sur les architectures à service. La plateforme \textit{OSGi} doit embarquer le \textit{bundle} \textit{iPojo} (\url{http://felix.apache.org/site/apache-felix-ipojo.html}) pour pouvoir utiliser l'architecture à composants orientés service. Nous utilisons dans nos expériences la plateforme \textit{OSGi} \textit{Apache Felix} en version 3.0.2.

La distribution Astronef est fournie en 3 \textit{bundles} obligatoires à déployer pour pouvoir utiliser l'ensemble des fonctionnalités présentés dans cette thèse (api core parser). Ces \textit{bundles} embarquent aussi le moteur \textit{Prova} (\url{http://prova.ws}, version 3.1.9 minimale) capable d'exécuter l'ensemble des règles présentés. Les extensions à Astronef sont aussi sous forme de \textit{bundles} dont les classes dépendent de l'\textit{api}. C'est le cas du \textit{bundle} \textit{préférences} qui ajout les opérateurs \textbf{Best} et \textbf{KBest}. Astronef est disponible sous licence Apache 2.0 à l'adresse \url{http://astral.googlecode.com}.

Asteroid est une extension d'Astronef distribuée en un seul \textit{bundle}. Il embarque le SGBD \textit{H2} (\url{http://h2database.com}). Ce SGBD est entièrement en \textit{Java} ce qui permet une cohérence des technologies. Le binaire ou le code source d'Asteroid n'est actuellement pas disponible au public.

L'ordinateur utilisé pour les expérimentations utilise un processeur \textit{Intel Xeon}, quadri-cœur de fréquences 2.8Ghz. Il possède 6Go de mémoire vive et un disque dur d'une vitesse de 7200RPM. Le système d'exploitation utilisé est Linux Ubuntu 11.04. La plupart des expérimentations se sont faites dans un environnement clôt en isolant les processus sur trois cœurs dédiés afin d'éviter les interférences. Enfin, les expérimentations ont été faites dans des conditions les plus stables possibles, après que le \textit{JIT} soit passé, après initialisation des caches internes et avec un \textit{garbage collector} le plus stable possible.
