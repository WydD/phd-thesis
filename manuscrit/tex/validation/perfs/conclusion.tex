\section{Conclusion}\label{sec:valid:perfs:conclusion}
La performance est un élément clé pour la gestion des flux de données. En effet, le fait de gérer des données temps réel indique qu'il faut pouvoir les traiter le plus rapidement possible pour ne pas risquer des blocages. Nous avons vu dans ce chapitre que notre approche nous permet de générer des plans de requêtes efficaces autant pour la gestion de flux de données que pour la jointure avec un SGBD.

Nous avons tout de même perçu la limite de notre approche. En effet, nous avons opté pour un système de règle. Cela nous permet d'avoir des résultats rapides et une intégration très efficace. Toutefois, pour la spécification de l'algorithme \textit{Pane} (macrobloc fenêtre-agrégat) par exemple, nous avons dû prévoir le cas où une projection se serait glissée entre la fenêtre et l'agrégat à cause de l'optimisation logique. Ce cas-là n'est pas grave, car la sémantique reste identique avec ou sans. Mais dans le cas d'une sélection, ou éventuellement un autre opérateur, nous ne pouvons peut-être plus former notre macrobloc, nous privant d'un gain certain de performance.

Nous avons présenté nos contributions et validé leurs résultats. Nous avons ainsi un système d'observation efficace capable d'interroger tout type de données. Le chapitre suivant présente une extension de ces travaux pour personnaliser les résultats en fonction de l'utilisateur.
