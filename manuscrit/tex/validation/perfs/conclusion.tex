\section{Conclusion}\label{sec:valid:perfs:conclusion}
La performance est un élément clé pour la gestion des flux de données. En effet, le fait de gérer des données temps réel indique qu'il faut pouvoir les traiter le plus rapidement possible pour ne pas risquer des blocages. Nous avons vu dans ce chapitre que notre approche nous permet de générer des plans de requêtes efficaces autant pour : la gestion de flux de données, la jointure avec un SGBD, ou encore avec un nouvel opérateur (\textbf{Best/KBest}).

Nous avons tout de même perçu la limite de notre approche. En effet, nous avons opté pour un système de règle. Cela nous permet d'avoir des résultats rapides et une intégration très efficace. Toutefois, pour la spécification de l'algorithme \textit{Pane} (macro-bloc fenêtre-agrégat) par exemple, nous avons du prévoir le cas où une projection se serait glissée entre la fenêtre et l'agrégat (ce qui arrive en pratique). Ce cas là n'est pas grave car la sémantique reste identique avec ou sans. Mais dans le cas d'une sélection, ou autre opérateur, nous ne pouvons plus former notre macro-bloc, nous privant d'un gain certain de performance.