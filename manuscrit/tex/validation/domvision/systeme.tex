\section{Adaptation au système observé}\label{sec:valid:domvision:systeme}
Dans cette section, nous analysons le système au sens où nous l'avons décrit jusqu'ici dans Astronef et Asteroid. Nous devons modéliser le système pour en extraire son schéma qui permettra de représenter le catalogue dans la base d'Asteroid. Puis nous nous intéressons aux données temps-réel. En enfin, nous présentons les flux à notre disposition dans ce réseau.

\subsection{Le schéma du système}
Nous présentons ici la vision que nous avons du système que nous observons. Il est important de voir que cette vision ne doit pas être lié aux sources de données disponibles. Elle a pour but de refléter l'interprétation de ou des experts souhaitant observer le système. Nous définissons ici 3 concepts principaux : les équipements, les applications et les interfaces réseaux.

\subsubsection{Équipements}
La notion d'équipement est une vision orienté utilisateur. En effet, un \textit{équipement} est un chassis physique. Les équipements virtuels ne sont ainsi pas représentés. Ce choix permet de représenter à l'utilisateur les appareils qu'il est capable de voir et de manipuler (débrancher par exemple). Quelques exemples : \textit{Livebox}, \textit{STB}\footnote{Set Top Box : Boitier connecté à une télévision pour fournir les services de partages de contenus, de télévisions, VOD...}, Ordinateurs, Tablettes, Disques durs réseaux.

Un équipement \textit{peut} avoir un \textit{numéro de série} unique. Cela représente a priori son seul moyen d'identification. Mais d'autres heuristiques peuvent être faites avec les autres concepts.

\subsubsection{Applications}
Nous désignons par \textit{application} un programme instancié sur un des équipements. À un \textit{équipement} est associé plusieurs \textit{applications}. Les \textit{applications} permettent d'effectuer des tâches sur leurs équipements respectifs. Une \textit{application} en exécution est considéré comme \textit{active} ou à l'inverse \textit{inactive}. Quelques exemples : Partage de contenu, Agent d'administration, Gestion d'impression.

Une \textit{application} a un \textit{nom}, une \textit{description} textuelle et potentiellement un \textit{type}. Il est nécessaire de pouvoir identifier une application pour un équipement donné par son \textit{nom}, l'unicité sur son \textit{type} ou si disponible un identifiant de processus.

\subsubsection{Interfaces réseaux}
Les \textit{interfaces réseaux} sont moins sujets à interprétation que les deux concepts précédents. Une \textit{interface réseau} représente un moyen de connexion que possède un \textit{équipement}. Nous supposons que cette interface est une interface \textit{MAC}\footnote{Permettant des interfaces \textit{IP} et \textit{Zigbee} potentiellement}. Elle possède un \textit{nom} système unique pour l'équipement, un type (wifi, ethernet), ainsi qu'une adresse \textit{MAC} considéré unique sur le réseau et d'autres adresses comme \textit{IPv4} ou \textit{v6}.

L'identification de ces interfaces peut être faite via ses adresses, bien que seule l'adresse \textit{MAC} garantisse l'unicité.

\subsubsection{Concepts supplémentaires possibles}
Ayant une modélisation des concepts principaux du réseau local domestique, nous pouvons maintenant ajouter des concepts supplémentaires. Bien que nous ne les utiliserons pas dans la suite de ce chapitre, il est intéressant d'en voir quelques exemples.

Les \textit{services} sont des compositions d'\textit{applications}. Ils permettent de fournir des expériences à l'utilisateur de plus haut niveau. À ces \textit{services}, nous pouvons attacher des contrats de services.

Enfin, pour développer la topologie du réseau, les \textit{liens} relient deux interfaces réseaux. Les \textit{chemins} composent des \textit{liens} réseaux. Et les \textit{canaux de transport} peuvent utiliser un \textit{chemin} pour aussi relier deux \textit{applications}. Toutefois, la collecte de ces informations est soumise à des contraintes de mise en œuvre plus technique comme l'utilisation de protocoles \textit{LLTD} ou \textit{IEEE 1905.1}.

\subsection{Les données volatiles intéressantes}
Les données \textit{volatiles} sont les données temps-réel que nous souhaitons archiver en vue d'une analyse a posteriori. En premier lieu, nous explorons les historiques de métriques. Ces données sont en général utilisées sur des graphiques afin de détecter des comportements anormaux.

Les métriques de charge seront pour nous les données à surveiller. En effet, la surveillance de la santé du réseau local passe avant tout par la surveillance de ses paramètres reflétant sa charge. Pour les équipements, nous surveillons la charge processeur utilisé (en \%) et mémoire occupée (en ko). Pour les interfaces réseaux, le débit est un bon indicateur de sa saturation. Toutefois, même si la donnée est difficile à obtenir, la capacité maximale de l'interface permet de savoir la charge totale de l'interface. Si nécessaire, d'autres métriques plus spécifiques pourront être enregistrées, par exemple : la latence ou la gigue d'une interface réseau .

Nous souhaitons aussi être capable de traquer les changements de structure du réseau réseau local. Pour cela, nous pouvons surveiller et historiser les changements d'adresse \textit{IP} des interfaces, ou de statut (actif/inactif) des équipements, applications et interfaces.

Avec ces données combinées avec les capacités d'interrogations que nous avons, nous devons être capable de surveiller le réseau local domestique et si nécessaire aider l'établissement de diagnostiques.

\subsection{Les données disponibles}
Dans notre mise en œuvre, nous utilisons le protocole \textit{UPnP} largement répandu dans le réseau local actuellement. Ce protocole permet d'annoncer un \textit{Device}\footnote{Attention au vocabulaire utilisé par le protocole. Un \textit{device} UPnP est en réalité une \textit{application} déployée sur un \textit{équipement}.} sur le réseau. Ce \textit{Device} expose des \textit{Services} qui contiennent des \textit{Actions} qu'il est possible d'invoquer à distance. Les \textit{Devices} répondent à des profils la plupart du temps standards.

L'annonce des \textit{Device} \textit{UPnP} sur le réseau produit un flux de donnée \begin{center}\textit{UPnPStatus}(uuid, ip, type, friendlyName, status, $\t$).\end{center} Ce flux indique que le \textit{device} exposé avec l'IP \textit{ip}, ayant pour identifiant unique \textit{uuid}, pour profil \textit{type} et pour description textuelle \textit{friendlyName} a changé son statut au temps $\t$ pour \textit{status} (\textit{active}/\textit{inactive}).

Nous avions exploré en section~\ref{sec:rw:supervision:administration} que les protocoles et agents d'administrations pouvaient être de bons interlocuteurs pour fournir des données. Dans le monde \textit{UPnP}, il existe le profil \textit{DeviceManagement} capable de fournir des données intéressantes. Toutefois, ce profil n'est pas déployé sur tous les équipements. Le tableau~\ref{tab:valid:domvision:upnpdm} indique les données que nous exploitons. Les flux produits auront en plus des paramètres les attributs \textit{uuid} correspondant à l'identifiant \textit{UPnP} de l'agent, ainsi que le \textit{timestamp} d'émission $\t$.

\begin{table}
\centering
\begin{tabular}{|m{0.25\textwidth}|>{\ttfamily}m{0.25\textwidth}|m{.4\textwidth}|} \bottomrule
\rowcolor{hypcolor} Nom du flux & \rm Paramètre & Description du paramètre\\ \hline
\multicolumn{3}{|c|}{/UPnP/DM/DeviceInfo/PhysicalDevice/} \\\hline
Serial & {SerialNumber} & Numéro de série de l'équipement\\\hline
\multirow{3}{*}{NetworkInterface} & {SystemName} & Nom de l'interface réseau\\\cline{2-3}
& {MACAddress} & Adresse MAC de l'interface\\\cline{2-3}
& InterfaceType & Type de l'interface \\\hline
\multicolumn{3}{|c|}{/UPnP/DM/Configuration/} \\\hline
\multirow{2}{*}{IPInterface} & SystemName & Nom de l'interface réseau \\\cline{2-3}
& IPv4Address & Adresse \textit{IPv4} de l'interface \\ \hline
\multicolumn{3}{|c|}{/UPnP/DM/Monitoring/} \\\hline
\multirow{2}{*}{OperatingSystem} & CPUUsage & Charge actuelle du processeur\\\cline{2-3}
& MemoryUsage & Charge actuelle de la mémoire\\ \hline
\multirow{3}{*}{IPUsage} & SystemName & Nom de l'interface réseau\\ \cline{2-3}
& TotalBytesSent & Nombre d'octets envoyés \\\cline{2-3}
& TotalBytesReceived & Nombre d'octets reçus \\ \toprule
\end{tabular}
\caption{Listes des flux intéressant fournis par le profil UPnP-DM}\label{tab:valid:domvision:upnpdm}
\end{table}

\subsection{Les sources Astronef}
Nous avons conçu un composant spécifique pour la formation du flux \textit{UPnPStatus}. Ce composant n'a pas de paramètre et fournit directement l'interface \textit{Source} avec comme attributs ceux mentionnés. Son fonctionnement est événementiel.

Pour la communication sur \textit{UPnP-DM}, nous souhaitons un profil plus réutilisable. Nous avons conçu un composant générique capable d'interroger tous les nœuds de l'arbre et de fonctionner de manière événementielle et périodique. Le mécanisme d'événement ne provient pas du protocole car les paramètres que nous souhaitons observer ne sont pas en \textit{EventOnChange} (production d'une notification en cas de changement). Concernant l'accès à la donnée, le composant devra itérer si nécessaire pour récupérer toutes les instances (\#) d'un chemin et enverra l'ensemble des données dans un seul \textit{batch}. L'ensemble des paramètres de cette source est présenté dans la table~\ref{tab:valid:domvision:dmsource}.

\begin{table}[ht]
    \centering
    \begin{tabular}{cl}
        paramètre & description \\ \midrule
        path & Chemin d'accès aux données \\
        parameters & Liste CSV des paramètres souhaités \\
        period & Période de mise à jour du flux (optionnel) \\
        channel & Canal interne d'événement à souscrire (optionnel)
    \end{tabular}
    \caption{Paramètres du composant \textit{DMSource}}\label{tab:valid:domvision:dmsource}
\end{table}

Ainsi, pour les flux \textit{OperatingSystem} et \textit{IPUsage}, nous y accéderons périodiquement, et pour les autres, ils seront récupérés à chaque arrivée de \textit{Device} de ce profil\footnote{Nous aurions aussi pu faire une représentation à base de manipulation temporelle comme dans Asteroid.}. Pour cela, nous formons la requête $\sigma_{status=active\wedge type=dm}$ \textit{UPnPStatus}. Cette requête sera injecté dans le puit \textit{SourceNotifier} dont le but est de notifier les \textit{Sources} s'étant inscrit dans un canal d'événement. Ici, toutes les instances de \textit{DMSource} souhaitant être notifiées le seront dans le canal \textit{DM}.

Nous avons vu la représentation du système, les données intéressantes, et comment obtenir ces données. Nous présentons désormais des exemples de requêtes que nous posons au système d'observation pour démontrer son expressivité.