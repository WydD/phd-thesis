\section{Conclusion}\label{sec:valid:domvision:conclusion}
Le système d'observation produit par Astral-Astronef-Asteroid a montré une grande capacité d'adaptation. L'utilisation de notre solution dans un nouvel environnement passe par plusieurs étapes. Tout d'abord, l'utilisateur doit définir sa représentation du système et les données qu'il souhaite archiver. Ensuite des composants sont créés pour dialoguer avec les fournisseurs de données.

À partir des flux de données disponibles et des capacités d'expressions d'Astral, nous sommes arrivés à entretenir le catalogue représentant le système. De même, nous avons réussi à archiver les données volatiles que nous souhaitions. Ces données ont été extraites par l'utilisation de requêtes Astral. Par la suite, nous avons créé des alertes impliquant dans le cas le plus complexe des jointures entre les historiques du SGBD et les flux temps-réel. Enfin, nous avons pu historiser le catalogue descriptif du système.

Il est important de noter que toutes les notions et exemples présentés dans cette section ont été implémentés et expérimentés dans la pratique sur un réseau domestique d'expérimentation.

Toutefois, nous avons dû écrire des composants pour pouvoir mettre à jour les données du schéma descriptif. Les composants n'ont pas une grande complexité, mais cela rentre en conflit avec la création d'un système d'observation déclaratif. De plus, lorsque nous sommes dans l'incapacité d'identifier correctement un objet observé, la spécification des heuristiques ainsi que leurs limites n'est pas clair. Il devient impératif pour les prochains développements de cette approche d'intégrer une gestion automatique et plus déclarative des mises à jour.

Nous n'avons cependant toujours pas analysé l'aspect performance d'Astronef-Asteroid. Le chapitre suivant présente quelques expérimentations que nous avons pu faire pour mesurer l'efficacité de notre solution.
