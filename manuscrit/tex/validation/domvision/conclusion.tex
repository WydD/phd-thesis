\section{Conclusion}\label{sec:valid:domvision:conclusion}
Le système d'observation produit par Astral-Astronef-Asteroid a montré une grande capacité d'adaptation. L'utilisation de notre solution dans un nouvel environnement passe par plusieurs étapes. Tout d'abord, l'utilisateur doit définir sa représentation du système et les données qu'il souhaite archiver. Ensuite des composants sont créés pour dialoguer avec les fournisseurs de données.

À partir des flux de données disponibles et des capacités d'expressions d'Astral, nous avons pu entretenir le catalogue représentant le système. De même, nous avons archivé les données volatiles que nous souhaitions. Ces données ont été extraites par l'utilisation de requêtes Astral. Par la suite, nous avons créé des alertes impliquant dans le cas le plus complexe des jointures entre les historiques du SGBD et les flux temps-réel. Enfin, nous avons historisé le catalogue descriptif du système.

Nous avons été confrontés aux limites de notre approche lors de l'écriture des composants de mise à jour du schéma descriptif. Il est difficile d'évaluer la pertinence et les limitations des moyens d'identifications alternatifs ce qui peut être source d'inconsistances.

Il est important de noter que toutes les notions et exemples présentés dans cette section ont été implémentés et mis en pratique dans la pratique sur un réseau domestique d'expérimentation comme présenté dans le papier de démonstration~\cite{Petit:domvision}. De plus, ce projet a été intégré à un intergiciel plus large de gestion de service dans le réseau domestique présenté dans~\cite{Kaed:insight}. DomVision sert ici de fournisseur de données pour permettre de gérer la qualité de service des différentes applications déployées dans ce réseau.

Nous n'avons cependant toujours pas analysé l'aspect performance d'Astronef-Asteroid. Le chapitre suivant présente quelques expérimentations que nous avons pu faire pour mesurer l'efficacité de notre solution.
