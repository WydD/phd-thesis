\section{Conclusion}\label{sec:valid:domvision:conclusion}
Le système d'observation produit par Astral-Astronef-Asteroid a montré une haute capacité d'adaptation. L'utilisation dans un nouvel environnement de notre solution passe par plusieurs étapes. Tout d'abord, l'utilisateur doit définir sa représentation du système et les données qu'il souhaite archiver. Ensuite des composants sont créés pour dialoguer avec les fournisseurs de données. 

À partir des flux de données disponibles et des capacités d'expressions d'Astral, nous sommes arrivés à entretenir le catalogue représentant le système, à travers la création de composants dédiés pour la mise à jour. De même, nous avons réussi à archiver les données que nous souhaitions, après manipulation par Astral, pour identifier par le SGBD les données ou pour les transformer. Par la suite, nous avons créé des alertes impliquant dans le cas le plus complexe des jointures entre les historiques du SGBD et les flux temps-réel.

Pour démontrer la flexibilité de l'architecture, et pour montrer que nous pouvons introduire les points de vues utilisateurs dans l'observation, nous avons intégré un nouvel opérateur de préférences dans notre solution. Cet opérateur nous a permis de sélectionner les données les plus intéressantes selon le profil de l'utilisateur. Cette intégration s'est faite via la spécification de 6 courtes règles ce qui rend l'ajout de composants facile.

Il est important de noter que toutes les notions et exemples présentés dans cette section ont été implémentés et expérimentés dans la pratique sur un réseau local domestique d'expérimentation.

Toutefois, nous avons du écrire des composants pour pouvoir mettre à jour les données du schéma descriptif. Les composants sont de taille raisonnables (<300 lignes de code chacun) mais, dans notre vision déclarative, nous considérons qu'il n'est pas correct d'écrire un code algorithmique pour spécifier notre observation. Ainsi, il devient impératif pour les prochains développement de notre approche d'intégrer une gestion automatique des mises à jours.

Nous n'avons cependant toujours pas analysé l'aspect performance d'Astronef-Asteroid. Le chapitre suivant présente quelques expérimentations que nous avons pu faire pour mesurer l'efficacité de notre solution.