\section{Choix des fondations de l'algèbre}\label{sec:valid:expressivite:modele}
Lors de l'établissement des premières définitions d'Astral dans la section~\ref{sec:contrib:astral:definitions}, nous avons fait des choix concernant le temps, les entités et les équivalences de requêtes. Cette section discute de la validité de ces choix par en montrant que des choix différents auraient menés à des ambiguïtés sémantiques. Nous présentons d'abord l'hypothèse de continuité du temps. Nous analysons ensuite nos choix en terme de gestion des ordres. Enfin, nous détaillons les équivalences de requêtes.

\subsection{Continuité du temps}
La définition~\ref{def:timestamp} présente un \textit{timestamp} en tant qu'élément d'un espace continu. Nous avons vu que dans la littérature, le temps était souvent considéré comme un entier, ou au mieux dans un espace isomorphe à $\N$. Ce choix permet de mieux gérer les différences d'interprétation du temps par l'implémentation.

En effet, chaque système informatique est limité par un \textit{chronon} : la plus petite différence de timestamp observable. Pour certains système, ce \textit{chronon} est d'une milliseconde, d'autres d'une seconde et d'autres d'un \textit{tick} processeur. Ne pas se restreindre à un \textit{chronon} particulier nous permet de manipuler facilement les \textit{timestamps} issus de deux systèmes de datation (synchronisés) sans ambiguïté. De façon plus formelle, nous aurions pu définir le temps comme un ensemble isomorphe à $\N$ quelconque mais les jointures et autres opérations binaires auraient été plus complexe.

Nous remarquons aussi que ce choix nous a permis de clarifier l'opérateur $\mathcal{RS}$ de STREAM~\cite{Arasu:stream}. Cet opérateur est décrit par la phrase \enquote{\it à chaque timestamp, envoyer l'ensemble de la relation}. Nous avions potentiellement une ambiguïté sachant que l'interprétation de \textit{timestamp} pouvait dépendre du système dans lequel il était exécuté. Cet opérateur est remplacé par $\RS{r}$ dans Astral avec $r$ une période de temps explicite.

\subsection{Hypothèse de la cohérence temporelle}
L'hypothèse de la cohérence temporelle~\ref{hyp:ordres} affirme que les n-uplets doivent arriver dans un flux de manière ordonnée selon leurs \textit{timestamps}. Cette hypothèse a suscité beaucoup d'intérêt de la part de la communauté. La question ouverte est de savoir s'il est nécessaire que des opérateurs existent dans l'algèbre, ou est-ce à l'implémentation de le garantir d'une manière ou d'une autre ? Nous pouvons par exemple noter Aurora~\cite{Abadi:aurora} qui avait défini des opérateurs spécifiques au réordonnement d'un flux grâce à une mémoire tampon de $n$ n-uplets.

En supposant que l'hypothèse n'est pas vérifiée, nous avons la fonction position-\textit{batch} (def~\ref{def:tau}) qui n'est plus croissante. Ainsi, sa pseudo-inverse n'existe plus, ce qui fait qu'il devient impossible de définir les séquences de fenêtres telles que nous les avons faites. Il est, en effet, facile de voir que la fenêtre explicitant les $50$ derniers n-uplets. La notion de dernier est sémantiquement lié aussi à son \textit{timestamp} ce qui induit des confusions. D'un point de vue implémentation, la croissance du temps fait qu'une fois qu'un \textit{timestamp} $t$ est présent dans un n-uplet, nous sommes garanti que toutes les données inférieures à $t$ sont arrivées. Ainsi le \textit{scheduler} peut décider d'exécuter un opérateur en garantissant son résultat.

Notre constat est le suivant : pour avoir une algèbre sans ambiguïté sémantique, nous devons supposer que les \textit{timestamps} sont croissants. C'est à l'implémentation de garantir cette contrainte, et si elle n'est pas vérifiée, il devient très difficile de prévoir les conséquences que cela va induire.

\subsection{Sémantiques d'ordres}
L'évaluation d'une relation temporelle à un \textit{batch} donné n'est pas un ensemble d'n-uplet, c'est une séquence. Ce choix est central car il intervient dans la majorité des définitions. Que ce soit dans la définition du produit cartésien~\ref{def:produit} ou des \textit{streamers}~\ref{def:streamers}, nous explicitons la sémantique que nous choisissons pour l'ordre des n-uplets résultants.

Dans la littérature, les flux sont souvent étendus du modèle $\mathcal{SEQ}$~\cite{Seshadri:seq} décrivant les manipulations de séquences. Il est acquis que les flux sont des ensembles totalement et strictement ordonnés. Or, lorsque nous transformons ce flux en relation temporelle, cet ordre est rarement abordé. Pourtant, nous avons vu dans la définition des \textit{streamers}~\ref{def:streamers} qu'il faut assigner un ordre strict aux flux produits. 

Dans Astral, nous considérons que les relations instantanées doivent avoir un ordre pour obtenir un flux correctement ordonné après l'application d'un \textit{streamer}. Ainsi, nous n'avons pas d'ambiguïté sémantique lié à cet aspect. Cette formalisation plus stricte nous a amené à découvrir le théorème~\ref{thm:asymetrie} qui dévoile l'asymétrie du produit cartésien, contrairement aux affirmations actuelles.

Il est important de noter que ce théorème est vrai à cause du choix de l'équivalence de requête (def~\ref{def:equivalence}). Cette équivalence inclue deux notions, celle des entités initialisé qui prend tout son sens dans le cadre des transpositions, et celle des inclusions et équivalences de séquences. Nous avons en effet définit l'inclusion des relations instantanées comme une définition d'inclusion de suite, ce qui nécessite une équivalence de l'ordre. Une définition différente de l'équivalence de requête permettrait d'obtenir une symétrie du produit. Dans la pratique, il est en effet possible que l'utilisateur souhaite obtenir ses n-uplets dans un ordre quelconque, ce qui pourrait permettre des optimisations supplémentaires.

Nous avons présenté comment les choix que nous avons fait pour les fondations de l'algèbre permettent d'avoir une sémantique claire et sans ambiguïté. Nous comparons maintenant l'expressivité d'Astral comparé aux algèbres existantes.