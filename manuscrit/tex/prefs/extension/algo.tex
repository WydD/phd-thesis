\section{Mise en œuvre}\label{sec:ext:prefs:algo}
Cette section présente les algorithmes que nous avons développés pour les opérateurs \textbf{Best} et \textbf{KBest}. Ces opérateurs nécessitent de connaître la hiérarchie des n-uplets déduite d'un ensemble de préférences. Cette hiérarchie est déduite du \textit{graphe de préférence} \textit{GP}. Nous présentons ci-après deux algorithmes pour la gestion du graphe : l'un est calculé sur l'état courant de la relation temporelle d'entrée, l'autre travaille sur les différences de la relation temporelle de manière incrémentale.

\subsection{Le graphe de préférence}
L'ordre de préférence entre deux n-uplets $t_1$ et $t_2$ selon une règle $\varphi$ est construit à partir d'une fonction $Com\-pare(t_1,t_2,\varphi_k)$ qui retourne $\emptyset$ si $t_1$ et $t_2$ sont incomparables, 1 si $t_1$ est préféré à $t_2$ et $-1$ sinon. L'algorithme~\ref{algo:comparT} étend la comparaison à une théorie $\Gamma$. Cette dernière utilise un ensemble de règles pour calculer l'ordre de préférence, mais ne calcule pas la clôture transitive.

\textbf{Graphe de préférence : }
Les opérateurs \textbf{Best} et \textbf{KBest} sont appliqués sur un ensemble de n-uplets $TS$ et ont besoin du graphe de préférence relatif à $TS$.
Pour son implémentation, nous avons adopté la structure $Graph(Next, Prec, Src)$ défini comme suit:
 $Next$ associe chaque n-uplet à la liste des n-uplets \textbf{qu'il domine}.
 $Prec$ associe chaque n-uplet à la liste des n-uplets \textbf{qui le dominent}.
 $Src$ est l'ensemble des n-uplets non dominés, représentant la source du graphe.
Afin de fournir de bonnes performances, les implémentations de ces structures sont des \textit{hash-sets} ou \textit{hash-maps}.

La construction et la mise à jour du graphe utilisent des méthodes \textit{Graph.Insert} et \textit{Graph.Delete} qui fonctionnent comme suit. Pour insérer un n-uplet dans un graphe, la méthode \textit{Graph.Insert} itère sur les nœuds du graphe pour mettre à jour \textit{Next}, \textit{Prec} et l'ensemble \textit{Src}. Comme le coût de l'insertion et de suppression dans une structure hachée peut être considéré comme $\mathcal O(1)$, le coût global de l'insertion d'un n-uplet dans le graphe est de $\mathcal O(\abs G)$. Pour la suppression d'un n-uplet $s$ du graphe, la méthode \textit{Graph.Delete} itère sur les nœuds connectés à $s$. Le coût est de $\mathcal O(\mathrm{deg}(s))$.

Pour une théorie de préférences $\Gamma$ donnée et un ensemble de n-uplets $TS$, la construction du graphe de préférence complet est réalisée par l'algorithme~\ref{algo:create}.

\subsection{Calcul de Best et KBest}
Par définition, le graphe inclut l'ensemble $Src$ qui contient les n-uplets les plus préférés. L'ensemble $Src$ est le résultat de l'opérateur \textit{Best}. Toutefois, l'algorithme peut-être optimisé en évitant de construire complètement \textit{GP}. La méthode $Graphe.Insert$ peut être optimisée pour tenir compte de cela. 
La complexité est alors réduite à $\mathcal O(\abs{Src})$. La complexité de $Best(R)(b)$ devient $\mathcal O(NS)$ où $N = \abs{R(b)}$ et $S=\abs{Best(R)(t)}$.

L'algorithme principal utilisé pour calculer $KBest$ à partir du \textit{GP} est un tri topologique de Kahn limité aux $k$ premiers résultats. Voir l'algorithme~\ref{alg:kbest}.

\subsection{Évaluation incrémentale du GP}
Cette section présente un algorithme pour le calcul incrémental de \textit{GP}. Le fait que les requêtes avec préférences sur les flux s'expriment sur des séquences de fenêtres motive cette méthode. Il est nécessaire de construire le \textit{GP} pour l'ensemble des n-uplets de la fenêtre \textit{courante}. Comme deux fenêtres consécutives peuvent se superposer, le nouveau \textit{GP} peut être construit par mise à jour incrémentale du graphe \textit{courant}. 

Supposons que $\delta_R^{-}$ contienne les n-uplets \textit{sortants} de la fenêtre et $\delta_R^{+}$ contienne ceux qui \textit{rentrent} dans la nouvelle fenêtre. Il n'y a pas d'intersection entre ces deux ensembles. Ces ensembles sont utilisés par l'algorithme~\ref{algo:update} pour construire le \textit{GP} de la nouvelle fenêtre à partir du \textit{GP} de la précédente. 

Il est important de noter que l'approche incrémentale est intéressante si la différence entre deux graphes successifs est faible comparée à la taille totale du graphe. Dans un tel cas, une grande proportion du \textit{GP} est réutilisée. Dans le cas contraire, la création du \textit{GP} \textit{de zéro} est meilleure. Le tableau~\ref{tab:valid:perfs:prefs:complexity} résume l'ensemble des complexités algorithmiques.

\begin{table}[p]
\noindent
\begin{minipage}{0.55\textwidth}
\small
\begin{algorithm}[H]\caption{Calcule KBest$(R)(t)$}\label{alg:kbest}
\dontprintsemicolon
\KwData{La structure GP, $k$ le nombre de n-uplet demandé}
Res $\gets $ \textbf{new TreeSet}$()$ \tcp{Ensemble ordonné}
\If{$k < \abs{Src}$}{\tcp{Src contient plus de $k$ n-uplets}
	$N \gets \abs{Src} - k$ \;\tcp{L'ordre position de Src est utilisé}
	\ForEach{$s\in Src$}{\tcp{ pour garder les $k$ n-uplets les plus récents}
		\lIf{$N = 0$}{Res.add$(s)$\;}
		\lElse{$N \gets N-1$\;}
	}
	\Return{Res}
}
NextLvl $\gets Src$; $id \gets 0$\;
PrecCount $\gets $ \textbf{new HashMap}()\;
\While{$id<k$ \textbf{and} $id < \abs{Src}+$Prec.count()}{
	\tcp{Buffer contient les n-uplets du même niveau}
	\If{Buffer = $\emptyset$}{
		\ForEach{$t\in $NextLvl}{
			Buffer.push$(t)$\;
		}
		NextLvl.clear()\;
	}
	$t\gets$ Buffer.pop()\;
	\ForEach{$s\in Next$.get$(t)$}{\tcp{Pour tout n-uplet dominé par $t$}
		$n\gets $PrecCount.get$(s)$\;
		\If{$n = $\textbf{null}}{$n=$Prec.get$(s)$.size()\;}
		\If{$n=1$}{
			NextLvl.add$(s)$\;\tcp{$s$ fait parti du prochain niveau}
		} \lElse {
			PrecCount.put$(s,n-1)$\;\tcp{Il n'y a plus de nœud dans cet ensemble}
		}
	}
	Res.add$(t)$\;
}
\Return{Res}
\end{algorithm}
\end{minipage}
\begin{minipage}{0.45\textwidth}
\small

\begin{algorithm}[H]\caption{ComparT$(t_1,t_2,\Gamma)$}\label{algo:comparT}
\dontprintsemicolon
%\begin{algorithmic}
%\KwIn{$t_1$ and $t_2$ two tuples from $R$}
\KwData{$\Gamma = \{\varphi_1,...,\varphi_k\}$ une théorie}
\KwResult{$\{1,-1,\emptyset\}$, \\ \quad resp. $\{t_1 >_{\Gamma} t_2$, $t_1 <_{\Gamma} t_2$, inc.$\}$}
\ForEach {$\varphi_k \in \Gamma$}{
	$r\gets Compare(t_1,t_2,\varphi_k)$\;
\lIf{$r \neq \emptyset$}{
\Return{$r$}
}
}
\Return{$\emptyset$}
%\end{algorithmic}
\end{algorithm}

\vspace{1cm}
\begin{algorithm}[H]\caption{Créer GP}\label{algo:create}
\dontprintsemicolon
\KwIn{$TS$ un ensemble de n-uplets}
\KwData{La structure de GP}
\lForEach{$s\in TS$}{
	$Graph.$insert$(s)$\;
}
\end{algorithm}

\vspace{1cm}
\begin{algorithm}[H]\caption{GP incrémental}\label{algo:update}
\dontprintsemicolon
\KwIn{$\delta_R^{-}(t,i)$ and $\delta_R^{+}(t,i)$}
\KwData{La structure du GP}
\lForEach{$s\in \delta_R^{-}(t,i)$}{
	$Graph.$Delete$(s)$\;
}
\lForEach{$s\in \delta_R^{+}(t,i)$}{
	$Graph.$Insert$(s)$\;
}
\end{algorithm}

\vspace{1cm}
\begin{tabular}{rcc}
& Créer GP & GP Incrémental\\ \noalign{\hrule height 1pt}
Best \quad &\quad $\mathcal O(N.S)$\quad & $\mathcal O(\Delta.N)$ \\
KBest \quad & $\mathcal O(N^2)$ & \quad $\mathcal O((\Delta+k)N)$ \quad\\ \noalign{\hrule height 1pt}
\end{tabular}
\caption{Complexité de Best/KBest}\label{tab:valid:perfs:prefs:complexity}
\end{minipage}
\end{table}
