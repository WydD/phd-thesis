\begin{savequote}[6cm]
<< The shape's fine, just make the whole thing... you know, cooler. 

\quad It needs to be about 20\% cooler. >>
\qauthor{Rainbow Dash}
\end{savequote}
\chapter{Extension à la gestion des préférences}\label{chap:prefs}
\chaptertoc

Les systèmes devenant de plus en plus complexes, ils produisent de plus en plus de données. De nombreux efforts ont été consacrés ces dernières années à la personnalisation des réponses lors de l'accès aux bases de données. Dans notre cadre, un de nos objectifs est de pouvoir adapter les résultats en fonction de l'utilisateur. Dans ce chapitre, nous explorons une extension à notre approche pour ajouter des opérateurs capables d'adapter les résultats d'interrogations aux préférences utilisateurs. Ainsi, l'utilisateur n'obtient que les données qui l'intéressent.

Dans la section~\ref{sec:ext:prefs:algebre}, nous détaillons les fondements théoriques à la gestion de préférences contextuelles. Une fois que les opérateurs sont décrits dans l'algèbre Astral, nous pouvons présenter les algorithmes utilisés pour les mettre en œuvre dans la section~\ref{sec:ext:prefs:algo}. Enfin, dans la section~\ref{sec:ext:prefs:integration} nous présentons l'intégration faite dans Astronef pour que l'utilisateur puisse personnaliser ses résultats d'interrogation. Cette intégration est accompagnée d'une évaluation de performances pour sélectionner le meilleur algorithme de calcul.

\input{tex/prefs/extension/algebre}
\section{Mise en œuvre}\label{sec:ext:prefs:algo}
Cette section présente les algorithmes que nous avons développés pour les opérateurs \textbf{Best} et \textbf{KBest}. Ces opérateurs nécessitent de connaître la hiérarchie des n-uplets déduite d'un ensemble de préférences. Cette hiérarchie est déduite du \textit{graphe de préférence} \textit{GP}. Nous présentons ci-après deux algorithmes pour la gestion du graphe : l'un est calculé sur l'état courant de la relation temporelle d'entrée, l'autre travaille sur les différences de la relation temporelle de manière incrémentale.

\subsection{Le graphe de préférence}
L'ordre de préférence entre deux n-uplets $t_1$ et $t_2$ selon une règle $\varphi$ est construit à partir d'une fonction $Com\-pare(t_1,t_2,\varphi_k)$ qui retourne $\emptyset$ si $t_1$ et $t_2$ sont incomparables, 1 si $t_1$ est préféré à $t_2$ et $-1$ sinon. L'algorithme~\ref{algo:comparT} étend la comparaison à une théorie $\Gamma$. Cette dernière utilise un ensemble de règles pour calculer l'ordre de préférence, mais ne calcule pas la clôture transitive.

\textbf{Graphe de préférence : }
Les opérateurs \textbf{Best} et \textbf{KBest} sont appliqués sur un ensemble de n-uplets $TS$ et ont besoin du graphe de préférence relatif à $TS$.
Pour son implémentation, nous avons adopté la structure $Graph(Next, Prec, Src)$ défini comme suit:
 $Next$ associe chaque n-uplet à la liste des n-uplets \textbf{qu'il domine}.
 $Prec$ associe chaque n-uplet à la liste des n-uplets \textbf{qui le dominent}.
 $Src$ est l'ensemble des n-uplets non dominés, représentant la source du graphe.
Afin de fournir de bonnes performances, les implémentations de ces structures sont des \textit{hash-sets} ou \textit{hash-maps}.

La construction et la mise à jour du graphe utilisent des méthodes \textit{Graph.Insert} et \textit{Graph.Delete} qui fonctionnent comme suit. Pour insérer un n-uplet dans un graphe, la méthode \textit{Graph.Insert} itère sur les nœuds du graphe pour mettre à jour \textit{Next}, \textit{Prec} et l'ensemble \textit{Src}. Comme le coût de l'insertion et de suppression dans une structure hachée peut être considéré comme $\mathcal O(1)$, le coût global de l'insertion d'un n-uplet dans le graphe est de $\mathcal O(\abs G)$. Pour la suppression d'un n-uplet $s$ du graphe, la méthode \textit{Graph.Delete} itère sur les n?uds connectés à $s$. Le coût est de $\mathcal O(\mathrm{deg}(s))$.

Pour une théorie de préférences $\Gamma$ donnée et un ensemble de n-uplets $TS$, la construction du graphe de préférence complet est réalisée par l'algorithme~\ref{algo:create}.

\subsection{Calcul de Best et KBest}
Par définition, le graphe inclut l'ensemble $Src$ qui contient les n-uplets les plus préférés. L'ensemble $Src$ est le résultat de l'opérateur \textit{Best}. Toutefois, l'algorithme peut-être optimisé en évitant de construire complètement \textit{GP}. La méthode $Graphe.Insert$ peut être optimisée pour tenir compte de cela. 
La complexité est alors réduite à $\mathcal O(\abs{Src})$. La complexité de $Best(R)(b)$ devient $\mathcal O(NS)$ où $N = \abs{R(b)}$ et $S=\abs{Best(R)(t)}$.

L'algorithme principal utilisé pour calculer $KBest$ à partir du \textit{GP} est un tri topologique de Kahn limité aux $k$ premiers résultats. Voir l'algorithme~\ref{alg:kbest}.

\subsection{Évaluation incrémentale du GP}
Cette section présente un algorithme pour le calcul incrémental de \textit{GP}. Le fait que les requêtes avec préférences sur les flux s'expriment sur des séquences de fenêtres motive cette méthode. Il est nécessaire de construire le \textit{GP} pour l'ensemble des n-uplets de la fenêtre \textit{courante}. Comme deux fenêtres consécutives peuvent se superposer, le nouveau \textit{GP} peut être construit par mise à jour incrémentale du graphe \textit{courant}. 

Supposons que $\delta_R^{-}$ contienne les n-uplets \textit{sortants} de la fenêtre et $\delta_R^{+}$ contienne ceux qui \textit{rentrent} dans la nouvelle fenêtre. Il n'y a pas d'intersection entre ces deux ensembles. Ces ensembles sont utilisés par l'algorithme~\ref{algo:update} pour construire le \textit{GP} de la nouvelle fenêtre à partir du \textit{GP} de la précédente. 

Il est important de noter que l'approche incrémentale est intéressante si la différence entre deux graphes successifs est faible comparée à la taille totale du graphe. Dans un tel cas, une grande proportion du \textit{GP} est réutilisée. Dans le cas contraire, la création du \textit{GP} \textit{de zéro} est meilleure. Le tableau~\ref{tab:valid:perfs:prefs:complexity} résume l'ensemble des complexités algorithmiques.

\begin{table}[p]
\noindent
\begin{minipage}{0.55\textwidth}
\small
\begin{algorithm}[H]\caption{Calcule KBest$(R)(t)$}\label{alg:kbest}
\dontprintsemicolon
\KwData{La structure GP, $k$ le nombre de n-uplet demandé}
Res $\gets $ \textbf{new TreeSet}$()$ \tcp{Ensemble ordonné}
\If{$k < \abs{Src}$}{\tcp{Src contient plus de $k$ n-uplets}
	$N \gets \abs{Src} - k$ \;\tcp{L'ordre position de Src est utilisé}
	\ForEach{$s\in Src$}{\tcp{ pour garder les $k$ n-uplets les plus récents}
		\lIf{$N = 0$}{Res.add$(s)$\;}
		\lElse{$N \gets N-1$\;}
	}
	\Return{Res}
}
NextLvl $\gets Src$; $id \gets 0$\;
PrecCount $\gets $ \textbf{new HashMap}()\;
\While{$id<k$ \textbf{and} $id < \abs{Src}+$Prec.count()}{
	\tcp{Buffer contient les n-uplets du même niveau}
	\If{Buffer = $\emptyset$}{
		\ForEach{$t\in $NextLvl}{
			Buffer.push$(t)$\;
		}
		NextLvl.clear()\;
	}
	$t\gets$ Buffer.pop()\;
	\ForEach{$s\in Next$.get$(t)$}{\tcp{Pour tout n-uplet dominé par $t$}
		$n\gets $PrecCount.get$(s)$\;
		\If{$n = $\textbf{null}}{$n=$Prec.get$(s)$.size()\;}
		\If{$n=1$}{
			NextLvl.add$(s)$\;\tcp{$s$ fait parti du prochain niveau}
		} \lElse {
			PrecCount.put$(s,n-1)$\;\tcp{Il n'y a plus de nœud dans cet ensemble}
		}
	}
	Res.add$(t)$\;
}
\Return{Res}
\end{algorithm}
\end{minipage}
\begin{minipage}{0.45\textwidth}
\small

\begin{algorithm}[H]\caption{ComparT$(t_1,t_2,\Gamma)$}\label{algo:comparT}
\dontprintsemicolon
%\begin{algorithmic}
%\KwIn{$t_1$ and $t_2$ two tuples from $R$}
\KwData{$\Gamma = \{\varphi_1,...,\varphi_k\}$ une théorie}
\KwResult{$\{1,-1,\emptyset\}$, \\ \quad resp. $\{t_1 >_{\Gamma} t_2$, $t_1 <_{\Gamma} t_2$, inc.$\}$}
\ForEach {$\varphi_k \in \Gamma$}{
	$r\gets Compare(t_1,t_2,\varphi_k)$\;
\lIf{$r \neq \emptyset$}{
\Return{$r$}
}
}
\Return{$\emptyset$}
%\end{algorithmic}
\end{algorithm}

\vspace{1cm}
\begin{algorithm}[H]\caption{Créer GP}\label{algo:create}
\dontprintsemicolon
\KwIn{$TS$ un ensemble de n-uplets}
\KwData{La structure de GP}
\lForEach{$s\in TS$}{
	$Graph.$insert$(s)$\;
}
\end{algorithm}

\vspace{1cm}
\begin{algorithm}[H]\caption{GP incrémental}\label{algo:update}
\dontprintsemicolon
\KwIn{$\delta_R^{-}(t,i)$ and $\delta_R^{+}(t,i)$}
\KwData{La structure du GP}
\lForEach{$s\in \delta_R^{-}(t,i)$}{
	$Graph.$Delete$(s)$\;
}
\lForEach{$s\in \delta_R^{+}(t,i)$}{
	$Graph.$Insert$(s)$\;
}
\end{algorithm}

\vspace{1cm}
\begin{tabular}{rcc}
& Créer GP & GP Incrémental\\ \noalign{\hrule height 1pt}
Best \quad &\quad $\mathcal O(N.S)$\quad & $\mathcal O(\Delta.N)$ \\
KBest \quad & $\mathcal O(N^2)$ & \quad $\mathcal O((\Delta+k)N)$ \quad\\ \noalign{\hrule height 1pt}
\end{tabular}
\caption{Complexité de Best/KBest}\label{tab:valid:perfs:prefs:complexity}
\end{minipage}
\end{table}

\section{Intégration de nouveaux composants}\label{sec:contrib:astronef:integration}
Astronef est basé sur l'architecture de composants à services. Ainsi, comme nous l'avons présenté précédemment, les composants sont nativement fournis avec des fabriques. Ces fabriques sont enregistrés dans le registre de service. La recherche d'un composant et sa création se fait donc via le motif d'interaction des services.

Toutefois, l'intégration des composants d'opérateurs nécessite aussi l'apport de ses connaissances en terme de règles logiques. L'intergiciel expose donc un service \textit{KnowledgeBase} capable d'ajouter des règles  à sa base de connaissance (sous forme de fichier ou de chaîne de caractère). Ainsi, il devient possible d'étendre les possibilités de la construction de requêtes.
\subsection{Construction du composant}
La construction du composant doit simplement être construit dans la technologie qui nous permet d'instancier l'architecture (en l'occurence iPojo/OSGi). Et il doit implémenter les services nécessaire à son exploitation. Par la suite, il doit spécifier les propriétés de configurations qu'il supporte, et évidemment respecter et correctement manipuler l'API fournie par Astronef pour manipuler les structures. Ceci est important pour notamment correctement manipuler le \textit{scheduler} en indiquant si le composant doit s'abonner aux modifications du résultat intermédiaire et autres.

\subsection{Règles logiques}
La seule règle obligatoire pour exploiter un nouveau composant est le fait de fournir au moins une règle \textbf{implrules} où le nom du composant (sa classe d'implémentation par défaut) est indiqué. Si ce composant implémente un macro-bloc, alors il faudra définir potentiellement un nouveau nom de nœud en plus des règles \textbf{macrobloc}.

Mais si ce composant implémente un nouvel opérateur que nous souhaitons utiliser dans l'expression de requête. Alors il est strictement \textbf{nécessaire} de définir sa sémantique en terme de types supportés et d'attributs fournis. Sans ces deux règles, il sera impossible de construire la requête. De plus, si nous possèdons la connaissance suffisante, nous pouvons indiquer son comportement face à la projection, la sélection ou d'autres optimisations logiques.

\section{Conclusion}
Après 20 années de recherche, la gestion de flux de données devient désormais suffisamment mature pour être appliqué massivement. Plusieurs produits commerciaux sont d'ailleurs maintenant utilisés en production. Toutefois, nous pouvons nous rendre compte que la complexité théorique de ces systèmes a été sous-estimé. De nombreux modèles ont été décrit pour représenter les flux de données et leurs traitements. Ces modèles sont encore remis en questions aujourd'hui au fur et à mesure des applications concrètes. 

Nous avons vu que le problème d'avoir une bonne connaissance du modèle et du comportement théorique des SGFD est crucial. En l'état, l'intégration des supports persistants reste ad-hoc et assisté par l'utilisateur. Un fonctionnement intégré avec une modélisation générique capable de gérer les deux modes d'interrogations de façon unifiée est donc indispensable pour manipuler correctement flux et relations persistantes. Similairement, les contributions sur l'optimisation de traitement des requêtes sont encore principalement ponctuelles. Afin d'appliquer un traitement efficace pour toute requête, il est nécessaire d'avoir une bonne connaissance théorique du traitement.

Notre contribution technique se concentrera sur trois points principaux :
\begin{itemize}
 \item[\textbf{Modélisation}] : Création d'Astral, algèbre de traitement des requêtes continues sur flux et relations temporelles. Nous accorderons de l'importance sur la prise en compte des problèmes relevés en section~\ref{sec:rw:sgfd:modeles:synthese}. Cette algèbre sera présenté dans le chapitre~\ref{chap:contrib:astral}
 \item[\textbf{Exécution}] : Mise en œuvre de l'intergiciel Astronef pour construire et exécuter efficacement une requête exprimée avec l'algèbre Astral. Ainsi, à partir d'une requête algébrique, il est possible de sélectionner le plan de requête qui semble le plus efficace grâce aux connaissances accumulés. Cette mise en œuvre sera développée dans le chapitre~\ref{chap:contrib:execution}.
 \item[\textbf{Persistance}] : Conception de l'extension Asteroid permettant l'intégration des requêtes continues sur flux et des requêtes sur support relationnel persistant. Ceci permettra de gérer la représentation du système observé ainsi que l'historisation des données dynamiques. Le support mathématique de cette intégration sera supporté par Astral et sa mise en œuvre par Astronef. Cette intégration sera effectuée dans le chapitre~\ref{chap:contrib:persistance}.
\end{itemize}

Grâce à ces contributions, il deviendra possible de mettre en œuvre un système d'observation générique applicable sur tout type de données. L'utilisateur devra exprimer des requêtes dans le langage algébrique Astral. Une fois ces requêtes écrites, nous serons garanti de leur mise en œuvre. Le tableau~\ref{tab:rw:contrib} résume l'ensemble des points d'analyses que nous nous étions fixés en section~\ref{sec:rw:supervision:criteres}.
\begin{table}[!ht]
\criteretabDonnee
    {Relationnel dérivé. Nous réutilisons les principes utilisés dans la gestion de flux et des bases de données.}
    {\good Modèle entité-relation augmenté pour supporter les flux.}
    {\good Requêtes sur tout type de données (flux, relations).}
\criteretabTraitement
    {\good Continue, Instantannée, Mixte}
    {\good Utilisation des requêtes continues des SGFD en tant qu'intégrateur.}
    {\meh Astral : langage de requête algébrique. Un langage purement déclaratif reste toutefois dérivable de ces fondations théoriques.}
    {\good Relationnel avec support \textbf{intégré} du dynamisme des données.}
\criteretabAdaptabilite
    {\good Spécification du modèle du système ainsi que des requêtes d'intégration (algébriques).}
    {\meh Pour l'analyse, la gestion de données multi-dimensionnelles des entrepôts utilisés est utilisé. Pour l'interrogation continue, utilisation d'un opérateur de préférences sur les flux.}
    {\good Infrastructure générique capable de supporter l'ajout d'opérateurs avec plusieurs implémentations.}
    {\good Héritage de l'efficacité des flux de données. Sélection du meilleur plan d'exécution pour chaque requête. Héritage des supports de grands volumes grâce aux entrepôts.}
\caption{Résumé de notre contribution selon nos critères}\label{tab:rw:contrib}
\end{table}
