\section{Conclusion}\label{sec:contrib:astronef:conclusion}
Nous avons présenté dans ce chapitre l'intergiciel \textit{Astronef}. Son architecture basé sur le modèle de composants orientés services permet une grande flexibilité. Nous pouvons en effet ajouter, supprimer ou remplacer chaque composant de l'intergiciel par d'autres. Cette flexibilité nous ouvre les portes pour une optimisation de traitement très poussé. Puisque nous avons plusieurs requêtes, ainsi que plusieurs composants dont la sémantique est équivalente, il devient intéressant de sélectionner le meilleur plan d'exécution.

Notre approche à base de règle a permit une mise en œuvre intuitive et rapide. Tout d'abord, nous restructurons l'expression algébrique pour qu'elle soit plus efficace. Cette restructuration est toutefois limité car elle s'appuie sur des heuristiques. Une version plus élaborée sera d'utiliser les capacités d'un assistant de preuve. En utilisant les définitions exactes d'Astral, nous pourrions être capable de vérifier si deux requêtes sont équivalentes. Ce qui pourrait nous élargir le champ de recherche.

Puis nous sélectionnons les meilleurs composants pour exécuter cette nouvelle requête. Contrairement au domaine de l'optimisation des SGBD, nous n'avons pas eu recours à une évaluation du coût de la requête, ce qui pourrait être une amélioration notable.

Toutefois, notre implémentation permet une sélection d'un plan d'exécution qui est intuitivement efficace. Son évaluation sera abordé dans le chapitre~\ref{chap:valid:perfs}. De plus, l'intégration de nouveaux composants se fait très rapidement. D'un point de vue architecture, notre modèle sous-jacent permet une telle opération. Il nous suffit de fixer la sémantique algébrique du composant selon plusieurs règles et l'intégration sera complète.

Dans le chapitre suivant, nous profitons de l'extensibilité de cet intergiciel pour mettre en œuvre une extension d'\textit{Astronef} afin d'effectuer le couplage avec un SGBD relationnel.