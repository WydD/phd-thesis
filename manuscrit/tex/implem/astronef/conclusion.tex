\section{Conclusion}\label{sec:contrib:astronef:conclusion}
Nous avons présenté dans ce chapitre l'intergiciel \textit{Astronef}. Son architecture basée sur le modèle de composants orientés services permet une grande flexibilité. Nous pouvons en effet ajouter, supprimer ou remplacer chaque composant de l'intergiciel par d'autres. De plus, nous sommes capables de décrire la sémantique des composants opérateurs grâce à Astral. Ainsi, nous pouvons aligner l'expression algébrique avec son implémentation. Toutefois, pour une expression, nous avons plusieurs requêtes, ainsi que plusieurs composants dont la sémantique est équivalente. Il devient intéressant de sélectionner le meilleur plan d'exécution.

Notre approche à base de règle a permis une mise en œuvre intuitive et efficace. Tout d'abord, nous réécrivons l'expression algébrique pour qu'elle soit plus performante. Ces réécritures sont basées sur les connaissances théoriques accumulées avec Astral que nous pouvons directement traduire en terme de règles. Puis nous sélectionnons les meilleurs composants pour exécuter cette nouvelle requête. Contrairement au domaine de l'optimisation des SGBD, nous n'avons pas recours à une évaluation du coût de la requête principalement, car les données ne sont pas encore présentes avant l'exécution de la requête. L'évaluation des performances d'Astronef est abordée dans le chapitre~\ref{chap:valid:perfs}. 

De plus, l'intégration de nouveaux composants se fait rapidement. D'un point de vue architectural, notre modèle sous-jacent permet une telle opération. Il nous suffit de spécifier la sémantique algébrique du composant selon plusieurs règles et l'intégration est complète. Nous abordons cet aspect lors de l'extension de notre approche dans le chapitre~\ref{chap:prefs} sur la personnalisation des résultats.

Il serait intéressant d'étendre ces travaux pour passer au-delà de l'approche par règle afin d'éviter la spécification d'heuristiques. Par exemple, l'utilisation d'assistants de preuves pour permettre des réécritures non triviales ainsi que la spécification d'un modèle de coût permettrait d'affiner les recherches.

Dans le chapitre suivant, nous profitons de l'extensibilité de cet intergiciel pour mettre en œuvre le couplage avec un SGBD relationnel à l'intérieur d'Astronef.
