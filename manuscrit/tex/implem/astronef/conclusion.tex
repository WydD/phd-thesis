\section{Conclusion}\label{sec:contrib:astronef:conclusion}
Nous avons présenté dans ce chapitre l'intergiciel \textit{Astronef}. Il est le moteur d'exécution de requêtes \textit{Astral}. Son architecture basée sur le modèle de composants orientés services permet une grande flexibilité. Nous pouvons en effet ajouter, supprimer ou remplacer chaque composant de l'intergiciel par d'autres. De plus, nous sommes capables de décrire la sémantique des composants opérateurs grâce à Astral. Ainsi, nous pouvons aligner l'expression algébrique avec son implémentation. Toutefois, pour une expression, il existe de nombreuses possibilités pour son évaluation : différentes expressions algébriques sont possibles, différents composants peuvent être utilisés. Il devient intéressant de sélectionner le meilleur plan d'exécution.

Notre approche à base de règle a permis une mise en œuvre intuitive et efficace. Tout d'abord, nous réécrivons l'expression algébrique pour qu'elle soit plus performante. Ces réécritures sont basées sur les connaissances théoriques accumulées avec Astral que nous pouvons directement traduire en terme de règles. Puis nous sélectionnons les meilleurs composants pour exécuter cette nouvelle requête. L'évaluation des performances d'Astronef est abordée dans le chapitre~\ref{chap:valid:perfs}. 

De plus, l'intégration de nouveaux composants se fait rapidement par les règles et par l'architecture à composants orientés services. En effet, il nous suffit de spécifier la sémantique algébrique du composant selon plusieurs règles et le composant peut être exploité. Nous abordons cet aspect lors de l'extension de notre approche dans le chapitre~\ref{chap:prefs} sur la personnalisation des résultats.

Dans le chapitre suivant, nous profitons de l'extensibilité de cet intergiciel pour mettre en œuvre le couplage avec un SGBD relationnel à l'intérieur d'Astronef.
