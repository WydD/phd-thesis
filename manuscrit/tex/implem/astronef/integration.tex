\section{Intégration de nouveaux composants}\label{sec:contrib:astronef:integration}
Astronef est basé sur l'architecture de composants orientés services. Ainsi, nous pouvons apporter de nouveaux composants. Toutefois, l'intégration des composants opérateurs nécessite aussi l'apport de ses connaissances en terme de règles logiques. L'intergiciel expose un service \textit{KnowledgeBase} capable d'ajouter des règles à sa base de connaissances (sous forme de fichier ou de chaînes de caractères). Ainsi, le constructeur de requête est lui aussi extensible.

Afin d'être compatible, le nouveau composant doit fournir la fabrique dans la même technologie que les opérateurs originels (en l'occurrence iPojo/OSGi). Il doit naturellement aussi fournir les services nécessaires à son exploitation. Enfin, il doit spécifier les propriétés de configurations qu'il supporte, et évidemment respecter et correctement manipuler les services d'Astronef pour manipuler les structures de données ou le \textit{Scheduler}.

La seule règle obligatoire pour exploiter un nouveau composant est de fournir au moins une règle \textbf{implrules} où le nom du composant (sa classe d'implémentation par défaut) est indiqué. Si ce composant implémente un macrobloc, alors il faut définir potentiellement un nouveau nom de nœud en plus des règles \textbf{macrobloc}.

Mais si ce composant implémente un nouvel opérateur que nous souhaitons utiliser dans l'expression de requête. Alors, il est strictement \textbf{nécessaire} de définir sa sémantique en terme de types supportés et d'attributs fournis. Sans ces deux règles, il est impossible de construire la requête. De plus, si nous possédons la connaissance suffisante, nous pouvons indiquer son comportement face à la projection, la sélection ou d'autres optimisations logiques.
