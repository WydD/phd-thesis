\section{Optimisation physique}\label{sec:contrib:astronef:physique}
Comme présentée dans la figure~\ref{fig:contrib:astronef:optimisation}, l'optimisation physique travaille sur l'arbre produit par l'optimisation logique. Les composants opérateurs implémentent différentes sémantiques d'exécution. Cette sémantique peut être traduite avec Astral. Une fois ces liens entre théorie et implémentation établis, nous avons plusieurs plans d'exécutions envisageables. Dans cette section, nous présentons les règles permettant de sélectionner les composants les plus efficaces.

\subsection{Macroblocs}
Afin d'exécuter des sous-parties de la requête de façon plus optimale, nous utilisons le concept de macroblocs. Un macrobloc est une composition d'opérateurs pouvant être réunis en un seul bloc, considéré plus efficace. Par exemple, la combinaison agrégation-fenêtre $\G S[W]$ largement étudié dans la littérature est un macrobloc. Sachant que des composants sont capables d'exécuter cet opérateur composite, nous pouvons effectuer cette transformation. Ici, l'hypothèse~\ref{hyp:macro} permet d'appliquer des macroblocs dès que possible.
\begin{hyp}[Heuristique des macros-blocs]\label{hyp:macro}
    Plus le nombre de composants opérateurs utilisés pour exécuter une requête est petit, plus son exécution est efficace.
\end{hyp}

Cette hypothèse est basée sur l'idée qu'un opérateur composite restreint ses capacités en terme de puissance d'expression. Ainsi, il devient plus spécialisé et efficace pour l'exécution de sa tâche plutôt que deux (ou plus) opérateurs génériques. De plus, moins il y a d'opérateurs, plus les travaux de planifications du \textit{Scheduler} sont allégés.

La mise en œuvre des macroblocs se fait via l'utilisation d'un prédicat spécifique pour réécrire un arbre ou un sous-arbre en un nouveau bloc. Il est important de voir que la nature du nœud peut se changer en un opérateur non standard de l'algèbre.
\begin{regle}[Regroupement de macroblocs]
La réécriture d'un groupe de nœud $[AIn,BIn,CIn]$ en un macrobloc $[AOut,BOut,COut]$ est géré par le prédicat :
\begin{center} \textbf{macrobloc}($[AIn,BIn,CIn],[AOut,BOut,COut]$).\end{center}
L'application de ce prédicat est faite de manière itérative.
\end{regle}
\begin{example}
    Reprenons la composition agrégation-fenêtre $\G S[W]$ dont il existe plusieurs implémentations. Nous allons combiner les deux opérateurs \textbf{aggregate} et \textbf{window} pour former un nouveau nœud \textbf{windowaggregate}. Celui-ci peut avoir plusieurs implémentations.
    \begin{lstlisting}
macrobloc(
    [aggregate, ArgAgg, [
        [window, ArgWindow, C]
    ]], 
    [windowaggregate, ArgWinAgg, C]
):-
    map_get(ArgWindow, "description", [D]), !,
    map_put(ArgAgg, ["description", [D]], ArgWinAgg). 
    \end{lstlisting}
\end{example}
Si un bloc est créé, il doit exister un composant capable de l'implémenter tel qu'il a été formé. Ainsi, il est nécessaire de vérifier des conditions supplémentaires si les capacités des composants sont limitées. En reprenant notre exemple, si nous ne possédons qu'un composant capable d'exécuter le nœud \textbf{windowaggregate} avec une description temporelle. Alors, nous devons le vérifier au moment de la formation du bloc.

\subsection{Encapsulation d'opérations de n-uplets}
Les opérateurs de n-uplets tels que présentés dans la définition~\ref{def:operateurtuple} sont des opérateurs dont l'évaluation peut se faire de manière indépendante, n-uplet par n-uplet. Plusieurs opérateurs de ce type peuvent regroupés dans un seul macrobloc.
\begin{defi}[Opérateurs de n-uplets]\label{def:operateurtuple}
    Soit $TS$ une séquence de n-uplets,

    Un opérateur de n-uplet $\Lambda$ est défini par la fonction partielle de n-uplet vers n-uplet : $\lambda$.
    $$\Lambda(TS) = \{\lambda(s) / s \in TS \}$$
\end{defi}

Ces opérateurs ont la particularité de vérifier le corollaire~\ref{cor:defunary} ce qui fait qu'ils sont applicables sur des flux ou des relations. De plus, ces opérateurs sont composables. La proposition~\ref{prop:composition:operateurtuple} montre que l'évaluation de celle-ci est égale à la composition des traitements n-uplet par n-uplet. La démonstration de cette proposition se fait trivialement par récurrence en composant les fonctions $\lambda$.
\begin{prop}[Composition d'opérateurs de n-uplets]\label{prop:composition:operateurtuple}
    La composition d'opérateurs de n-uplet est égale à l'opérateur de n-uplet défini par la composition de leurs fonctions partielles.
\end{prop}

Ainsi, il est possible de regrouper tous les opérateurs de n-uplets en un seul composant efficace. La règle pour appliquer cette opération utilise le prédicat \textbf{macrobloc}. Afin de simplifier les règles de type, d'implémentation et de macrobloc, nous créons un prédicat pour déclarer les opérateurs de n-uplets.
\begin{regle}[Déclaration d'opérateurs de n-uplets]
La déclaration que le nœud de nature $A$ avec la configuration $B$ peut se traiter n-uplet par n-uplet avec le composant $Impl$ se fait via le prédicat :
\begin{center}\textbf{unaryimpl}($A$,$B$,$Impl$)\end{center}
\end{regle}

Une fois ces composants déclarés, une règle interne réécrira l'ensemble des opérateurs de n-uplets sous un nœud \textbf{unary} comprenant la propriété \textit{operations} contenant la liste des opérations à appliquer.
\begin{example}
Soit la requête $\sigma_{a > 30}\Pi_{id,a} R$. Son expression avant l'encapsulation est :
\begin{lstlisting}
[sigma, {attributes: ["id","a"], type: "relation", 
		  condition: "a>30", conditionAttributes:"a"}, [
	[pi, {attributes: ["id","a"], type: "relation"}, [
		[source, {id:"R", attributes: ["id","a","b"], type:"relation"},[]]
	]]
]]
\end{lstlisting}

En supposant qu'il existe à l'intérieur d'Astronef les déclarations suivantes.
\begin{lstlisting}
unaryimpl(sigma,_, "SelectUnaryOperation").
unaryimpl(pi,_, "ProjectUnaryOperation").
\end{lstlisting}

Nous obtenons après encapsulation, le résultat suivant :
\begin{lstlisting}
[unary, {
	operations: [
		{impl: "ProjectUnaryOperation", attributes: ["id","a"]},
		{impl: "SelectUnaryOperation", attributes: ["id","a"], 
				condition: "a>30", conditionAttributes:"a"}], 
	attributes: ["id","a"], type: "relation"}, [
	  [source, {id:"R", attributes: ["id","a","b"], type:"relation"},[]]
]]
\end{lstlisting}
\end{example}


\subsection{Sélection des composants}
Nous avons maintenant un arbre dont la structure est jugée comme efficace. Nous en venons désormais à la dernière règle : la sélection du meilleur composant pour exécuter chaque nœud. Ce choix pourra être seulement basé sur la nature du nœud dans le cas où il n'existe pas d'autres alternatives. Mais dans la plupart des cas, il est nécessaire de sélectionner selon sa configuration.

\subsubsection{Calcul incrémental}
Dans la section~\ref{sec:rw:sgfd:optimisation:flux}, nous avons présenté le calcul incrémental et son intérêt dans l'optimisation du traitement des flux. Ce principe reste applicable naturellement dans Astral, il est toujours possible de définir à partir d'une relation temporelle $R$ les deux $\Delta_R^+$ et $\Delta_R^-$. Ces deux objets indiquent les différences de $R$ à chaque \textit{batch} (resp. ajout et suppression). 

Certains composants implémentant un opérateur particulier sont capables de traiter une relation en utilisant uniquement ces $\Delta_R$. Et certains opérateurs fournissent sans coût supplémentaire ces $\Delta_R$. Ainsi, il devient important de sélectionner les composants les plus capables d'exploiter ce mode. Il est important que garder à l'esprit que dans certains cas, les tailles des $\Delta_R$ sont similaires aux tailles des états de $R$ (quand les $R$ états de $R$ n'ont pas d'éléments en commun). Dans ce cas, il n'est peut-être pas optimal d'utiliser le mode incrémental\footnote{Mais ce choix est difficile à mettre en œuvre car il est souvent nécessaire d'évaluer les cardinalités, ce qui nécessite des modèles statistiques.}.

\subsubsection{Le prédicat de sélection}
La sélection du meilleur composant est importante, car les performances peuvent être différentes en fonction des implémentations. Par exemple, les opérateurs de séquences de fenêtres $[L]$ (dernier n-uplet) et $[B]$ (dernier batch) peuvent se définir avec une description de séquence de fenêtre générique. Toutefois, leur implémentation ne nécessite pas d'utiliser un opérateur générique, car leur calcul est très simple. Utiliser un composant qui calcule une description de fenêtre linéaire générique serait inutile et contre performant. 

\begin{regle}[Sélection d'implémentations]
La sélection du composant nommé $Impl$ pour le nœud $[A,B,C]$ auxquels sont ajoutées les propriétés optionnelles $Props$ est décidé par le prédicat :
\begin{center} \textbf{implrules}($[A,B,C]$,$Impl$,$Props$).\end{center}
L'application de ce prédicat est faite de manière récursive.
\end{regle}

Les propriétés optionnelles permettent d'indiquer certaines informations au composant qui utilise ce nœud. Par exemple, il est possible d'indiquer que cette implémentation fournit une vue incrémentale des modifications.
\begin{example}
	Nous souhaitons implémenter les règles pour choisir les meilleures fenêtres. Nous utilisons un prédicat capable d'identifier les fenêtres particulières $[L]$ et $[B]$. Si la description correspond à ces fenêtres, alors \textit{LastBatchWindow} est sélectionnée, sinon \textit{WindowImpl} est utilisé par défaut.
\begin{lstlisting}
implrules([window,{description: [D]},_], "LastBatchWindow", 
        {incremental: 1}):- % support de l'incremental
    lastbatchorlasttuple(D), !.
implrules([window,_,_], "WindowImpl", {incremental: 1}):- !.
\end{lstlisting}
\end{example}

\begin{example}
Le streamer $\IS$ est capable de supporter les deux modes. Il est important de noter que le calcul incrémental pour les streamers $\IS$ et $\DS$ est très important, car leur coût devient quasi nul en utilisant directement les $\Delta_R$. Ces deux modes sont gérés par deux implémentations différentes, ce qui se traduit par :
\begin{lstlisting}
implrules([streamer,{stype: "Is"}, [[_,{incremental:1},_]]], 
    "DynamicIsImpl"):- !. % Si Is et incremental
implrules([streamer,{stype: "Is"},_], "IsImpl"):- !. %Par defaut
\end{lstlisting}
\end{example}

L'arbre fourni par l'application finale de la sélection des composants est ensuite donné à un service capable d'instancier les différents composants qui constituent le plan de requête. Il est intéressant de noter que ce service est lui-même écrit en \textit{Prolog} par facilité d'implémentation.

Nous avons vu comment, par le renseignement de règles logiques, nous pouvons construire un plan de requête efficace. Ce plan est construit par sa restructuration logique et par la sélection des meilleurs composants pour mettre en œuvre cet arbre. Une évolution dans la même lignée que le calcul incrémental serait de raffiner les structures utilisées dans les résultats intermédiaires. Nous détaillons désormais les capacités de cette base de règle en terme d'extensibilité.
