\section{Conclusion}\label{sec:contrib:asteroid:conclusion}
Dans ce chapitre, nous avons présenté l'intégration d'un support persistant relationnel. Dans le cadre de l'observation de système, nous exploitons le support persistant pour la gestion du catalogue du système ainsi que les historiques de flux. L'analyse des quatre dynamiques de données a mis en avant une méthode pour concevoir le schéma physique de la base de données. Astral permet d'unifier les deux mondes pourtant régis par des dynamiques, des modes d'interrogation et des concepts différents. Ainsi, cette solution gère l'hétérogénéité en terme d'évolution des données que nous nous sommes fixés lors de cette thèse.

La mise en œuvre de l'intégration du support persistant s'appuie fortement sur les connaissances dont nous disposons avec Astral en terme de modélisation des sources de données. Nous avons conçu plusieurs composants capables de refléter différentes sémantiques pour les modes de collecte de données, comme en terme de persistance de celles-ci. La flexibilité du moteur de règles développé dans Astronef a permis de mettre en œuvre des optimisations non triviales du plan de requête.

Toutefois, notre approche est améliorable. En effet, dans le cadre de la persistance des données du catalogue, des composants spécifiques doivent être mis en place. L'absence de gestion déclarative pour ce point rend cette tâche délicate. 

Nous nous sommes basés sur des heuristiques et sur l'application itérative de règles pour résoudre l'optimisation du plan de requête. Une généralisation de l'approche pourrait optimiser des plans plus complexes, notamment dus à l'ordre des jointures. Par exemple, si nous souhaitons optimiser $(R_1 \Join R_2) \Join R_3$ avec $R_2$ et $R_3$ issus d'un même SGBD. Même en sachant qu'en Astral la jointure soit associative, il est difficile de spécifier une règle permettant $R_1 \Join (R_2 \Join R_3)$ avec $R_2\Join R_3$ poussé au niveau SGBD. Le domaine des SGBD a rencontré le même problème ce qui a permis l'introduction de la programmation dynamique dans l'optimisation de requêtes. Nous pourrions améliorer nos performances en utilisant un tel procédé.

Nous avons désormais un système de gestion de données persistantes et temps réel. Nous validons notre approche en déployant ce système pour l'observation du réseau domestique.