\section{Réécriture du plan d'exécution}\label{sec:contrib:asteroid:reecriture}
Notre objectif est de pouvoir interroger tout flux ou relation de façon unifiée grâce au langage Astral. Nous avons décrit dans la section précédente les composants \textit{dbsource} et \textit{dbjoin} permettant d'intégrer les relations persistantes dans Astronef. Nous devons indiquer à Astronef la manière de réécrire une requête Astral pour sélectionner un plan de requête efficace pouvant mêler relations persistantes et requêtes continues. Ainsi, l'objectif est de distribuer l'exécution de la requête entre Astronef et le SGBD.

\subsection{Approche de la distribution de l'exécution}
Notre heuristique de distribution que nous choisissons est que pour une requête relationnelle le SGBD sait calculer de façon plus efficace une requête.

\begin{hyp}[Heuristique de l'utilisation prioritaire du SGBD]\label{hyp:sgbd}
    Plus les opérations sont déléguées au SGBD plus le plan de requête est efficace dans le cas général.
\end{hyp}

Ce choix est argumenté par l'efficacité du SGBD grâce au précalcul des index et autres algorithmes optimisés pour ce support. Ce qui fait que les opérations de sélection ou de jointures sont très efficaces par rapport à un calcul à la volée. Cette heuristique s'inspire de l'hypothèse~\ref{hyp:optimpush} permettant de pousser les projections et sélections au plus proche des sources.

Afin d'appliquer cette transformation, nous adoptons une approche récursive. Nous possédons le composant \textit{dbsource} capable de représenter toute expression $Q$ relationnelle (exprimée en \textit{SQL}) en relation temporelle. Ainsi, si nous avons la requête $S[B] \Join \sigma_{...} R$ avec $S$ un flux et $R$ une relation persistante. Nous réécrivons premièrement $R$ en composant \textit{dbsource}. Puis, nous transformons la configuration de \textit{dbsource} pour que son expression \textit{SQL} intègre l'opérateur $\sigma_{...}$ et ainsi de suite jusqu'à ne plus pouvoir réécrire.

\subsubsection{Initialisation de la distribution}
Pour initialiser la réécriture avec les composants Asteroid, nous utilisons un sucre syntaxique permettant de traduire l'appel d'une relation \textit{Name} en composant \textit{dbsource} avec ses paramètres de configurations reflétant une mise à jour par \textit{trigger}. Après l'application de cette règle, nous avons un arbre dont toutes les interactions avec le SGBD se font via ces nœuds \textit{dbsource}.
\begin{lstlisting}
sugar(
    [entity,{name: Name},[]],
    [dbsource, {
        'attributes': Attributes,
        'query': Query,
        'mode': 'trigger', 
        'tables': [Name],
    },[]]
):-
    dbattributes(Name,Attributes), % Recupere les attributs de cette relation
    concat(["SELECT * FROM ", Name], Query). % Forme la requete SQL
\end{lstlisting}

\subsection{Règles de macroblocs pour le placement d'opérateurs}
Nous souhaitons transformer un bloc \lstinline|[A,B,[[dbsource,Config,[]],...]]| avec \lstinline|[A,B,_]| opérateur supporté par \textit{dbsource} en un seul \lstinline|[dbsource,NewConfig,[]]|. Ceci correspond à la spécification de macroblocs. Il existe deux types de transformations : celles modifiant la requête, celles modifiant le mode de mise à jour.

\subsubsection{Réécriture de la requête}
Voyons tout d'abord la modification de requête dans \textit{dbsource}. Nous définissons un prédicat capable de transformer les requêtes des sources en une nouvelle. Naturellement, la règle \textbf{macrobloc} doit transmettre les attributs et conserver les anciens paramètres de configuration.
\begin{regle}[Modification de la requête de dbsource]
    Soit $[A,B,C]$ un nœud dont tous les fils sont de type \textit{dbsource} avec pour requêtes \textit{Queries},

    La transformation du macrobloc $[A,B,C]$ en $[dbsource,Config,[]]$ est assuré par le prédicat suivant :
    \begin{center} \textbf{dboperator}($[A,B,C], Queries, NewQuery$). \end{center}
    Avec l'attribut \textit{query} de $Config$ égal à $NewQuery$.
\end{regle}

\begin{example}
    Nous souhaitons pousser une sélection sur le SGBD. Cette opération est faite par l'application de la clause \sql{WHERE} sur la requête \textit{SQL} actuelle. Par mesure de simplification, nous pouvons considérer la requête d'entrée comme une sous-requête à placer dans \textit{FROM}. Nous obtenons la règle de transformation suivante :
    \begin{lstlisting}
dboperator([sigma,B,_], [Query], NewQuery):- !,
    map_get(B, 'condition', Cond),
    conditionsql(Cond, Sql), % Traduit la condition en SQL
    concat(['SELECT * FROM (', Query, ') v WHERE ', Sql], NewQuery).
    \end{lstlisting}

    Si le SGBD possède un optimiseur suffisamment puissant, il est capable de traiter efficacement cette nouvelle requête.
\end{example}

Ainsi, il suffit de spécifier l'ensemble des règles permettant de traduire la sémantique Astral en \textit{SQL}. En l'état, nous avons implémenté les règles pour les opérateurs $\Pi$, $\sigma$, $\rho$, $\null_a G_b$, $\Join$ et $\cup$. Notons qu'il n'est pas évident de transformer les opérateurs binaires. Notamment, il est difficile de transformer la requête suivante $(\D^{f_1} R_1)\Join (\D^{f_2} R_2)$ (avec $f_1\neq f_2$) en $\D^{f} R$. Nous nous limitons au cas où les modes appliqués aux composants \textit{dbsource} sont égaux (typiquement \textit{trigger} ou \textit{oneshot}).

\subsubsection{Changement du mode de mise à jour}
Nous nous intéressons maintenant à l'application de l'opération $\D^f_c$ au composant \textit{dbsource} ce qui correspond à un changement de mode. L'application du prédicat \textbf{dbsourcemode} permet de traduire la sémantique de l'opérateur $\D^f_c$ par le mode de \textit{dbsource}.
\begin{regle}[Modification du mode de dbsource]
    Soit $[timetransform,B,[dbsource, Config, []]]$ un nœud de manipulation temporelle dont la transformation temporelle est de type $T$ et dont les paramètres est $Parameters$

    La transformation du macro-bloc $[timetransform,B,[dbsource, Config, []]]$ en $[dbsource,NewConfig,[]]$ est assuré par le prédicat suivant :
    \begin{center} \textbf{dbsourcemode}($T, Parameters, Config, NewConfig$).\end{center}
\end{regle}
\begin{example}
    Nous souhaitons effectuer une requête instantanée sur le SGBD et d'en produire la relation temporelle associée. Nous écrivons la règle suivante :
    \begin{lstlisting}
dbsourcemode("freeze", {'at': Time}, Config, NewConfig):- !,
    map_merge(Config, {
        'at': Time,
        'mode': "oneshot"
    }, NewConfig).
    \end{lstlisting}
\end{example}

En écrivant une règle pour chaque sémantique d'exécution spécifiée par la table~\ref{tab:contrib:asteroid:dbsource:modes}, nous pouvons au mieux réécrire le plan pour que les composants \textbf{dbsource} se mettent à jour lorsqu'il est nécessaire.

Une application de cet ensemble de règles est la capacité à traduire une requête Astral relationnelle pure en \textit{SQL}. Notamment, si toutes les sources issues du SGBD sont utilisés sous forme de requêtes instantanées (ou si l'opérateur $\D^{(t_0,0)}$ est appliqué en tête de la requête) : nous sommes capables d'exécuter une requête instantanée Astral par le SGBD. Ainsi, même pour faire une analyse a posteriori avec des requêtes instantanées, il reste possible d'écrire ses requêtes uniquement en Astral. Malheureusement, le surcoût induit par la réécriture fait que du point de vue des performances, il n'est pas intéressant de faire ainsi.

\subsection{Cas de la jointure hybride SGFD-SGBD}\label{sec:contrib:asteroid:reecriture:join}
Grâce à la flexibilité de \textit{dbsource}, nous avons pu réécrire le plan de requête pour placer au plus les opérateurs relationnels sur le SGBD. Toutefois, nous sommes aussi en possession d'un composant opérateur que nous pouvons exploiter : \textit{dbjoin}. Nous avons défini la sémantique de l'opérateur comme capable de supporter toute requête $R_1\ssjoin R_2$ avec $R_1$ relation temporelle quelconque et $R_2$ exprimée en algèbre relationnelle sur les relations de la base de données. Ce composant est un macrobloc que nous pouvons former grâce à une règle. Afin d'illustrer l'application successive des règles, voici l'ensemble des résultats intermédiaires appliqués lors du raisonnement :
\begin{align*} 
R_1\ssjoin_c R_2 &= \sigma_c \left(R_1 \Join \D^{\mathrm{change}}_{R_1} R_2\right) & \text{(via \textbf{sugar})}\\
& =  R_1 \Join_c \D^{\mathrm{change}}_{R_1} R_2 & \text{(via \textbf{pushselection})}\\
& =  R_1 \Join_c \D^{\mathrm{change}}_{R_1} \textbf{dbsource}_{trigger}^{Q} & \text{(via \textbf{dboperator})}\\
& =  R_1 \Join_c \textbf{dbsource}_{notify(R_1)}^{Q} & \text{(via \textbf{dbsourcemode})}
\end{align*}

Enfin, l'application d'une dernière règle va nous permettre de transformer cette jointure en : $\textbf{dbjoin}^Q_c (R_1)$. Cette règle est exprimée grâce au prédicat \textbf{macrobloc} :

\begin{lstlisting}
macrobloc(
    [join,JoinConfig,[
        [A,LeftConfig,C],
        [dbsource,DBConfig,[]]
    ]],
    [dbjoin,NewConfig,[[A,LeftConfig,C]]]
):- % Si...
    map_get(DBConfig, "mode", "notify"), % Le mode de dbsource est notify
    map_get(DBConfig, "dependentRId", RId), % Et notify pointe sur...
    map_get(LeftConfig, "rid", RId), % ... l'entite de gauche,
    !, % alors...
    map_get(DBConfig, "query", Query), % Recuperation de la requete
    map_get(JoinConfig, "attributes", Attr), % Recupere les attributs
    map_get(JoinConfig, "condition", "1==1", Cond), % Condition (1=1 par defaut)
    conditionsql(Cond, CondSQL), % Transformation en SQL
    map_merge(JoinConfig, { % Ajout des nouveaux parametres de configuration
        'query': Query,
        'condition': CondSQL, 
        'attributes': Attr,
    }, NewConfig).
\end{lstlisting}

Toutefois, l'utilisation de cette règle ne semble pas optimale dans tous les cas. Pour effectuer l'opération de jointure (semi-sensible) entre une relation temporelle et un SGBD, il existe deux plans de requête :
\begin{itemize}
    \item[\textbf{P1}] Une jointure relationnelle usuelle (par exemple, jointure hachée) est appliquée entre la relation temporelle et un nœud \textit{dbsource} configuré en mode \textit{notify}.
    \item[\textbf{P2}] Un \textit{dbjoin} configuré avec la bonne requête \textit{SQL} est appliqué sur la relation temporelle.
\end{itemize}

En effet, si les relations de la base de données ne sont pas mises à jours, alors il devient raisonnable de dire que la création d'une vue matérialisée en mémoire du résultat intermédiaire peut améliorer les performances avec le plan \textit{P1}. Cette vue devient un cache local. L'opérateur de jointure peut créer des accès plus optimisés comme des tables de hachages pour minimiser son coût. Toutefois, le coût de construction de cette vue est non négligeable, car \textit{dbjoin} exécute une requête sur le SGBD ce qui peut prendre beaucoup de temps.

A contrario, si nous appliquons un \textit{dbjoin} avec le plan \textit{P2}, la sélection des n-uplets fait que la taille du résultat est minimisée comparé au plan \textit{P1}. Les résultats intermédiaires peuvent être optimisés par le SGBD. Toutefois, cette requête est exécutée pour chaque n-uplet ce qui peut avoir un coût important. Même avec des stratégies de cache évoluées du côté du SGBD, il est possible d'avoir un état de la relation entrante différent pour chaque requête ce qui le rend difficile d'exploiter des caches.

Notre analyse empirique nous indique que la fréquence des mises à jour est déterminante pour privilégier \textit{P1} (basse fréquence) ou \textit{P2} (haute fréquence). Ce point est validé et affiné par les expérimentations que nous menons dans la section~\ref{sec:valid:perfs:couplage}.