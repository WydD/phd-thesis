\begin{savequote}[6cm]
<< I was elected because I can think outside the box. 
Which means... \textbf{*bunk*} I can also think inside a chimney! >>
\qauthor{Pinkie Pie}
\end{savequote}

\chapter{Asteroid : Intégration des supports persistants relationnels}\label{chap:contrib:asteroid}
\chaptertoc

Nous sommes désormais en possession d'un intergiciel capable de mettre en œuvre une requête exprimée en Astral. Or, cet intergiciel a pour propriété d'être extensible et Astral est capable de supporter l'hétérogénéité des données en terme de mode d'interrogation. Nous pouvons donc mettre en œuvre à la fois théoriquement et de façon pratique un système de gestion de données capable de se coupler avec un SGBD relationnel. Dans ce chapitre, nous allons présenter \textit{Asteroid}\footnote{Astronef Extension for Relations Inside Databases} qui étend l'intergiciel Asteroid pour mettre en œuvre ce couplage.

La section~\ref{sec:contrib:asteroid:theorie} présente les fondamentaux théoriques qui permettrons le couplage d'un SGBD relationnel avec un formalisme tel qu'Astral. Cette section détaillera de plus l'influence de l'évolution des données sur le schéma utilisé dans le SGBD couplé. Ensuite, nous détaillons en section~\ref{sec:contrib:asteroid:composants} les différents composants Astronef qui nous permettent d'instancier ce couplage. Enfin, nous présentons les règles de réécritures nécessaires pour restructurer le plan de requête suivant ces différents composants de couplage en section~\ref{sec:contrib:asteroid:reecriture}. Puis, nous concluons en section~\ref{sec:contrib:asteroid:conclusion}.

\section{Formalisation théorique}\label{sec:contrib:asteroid:theorie}

\subsection{Dynamique des données}
Dans un système, nous avons vu que les données peuvent être persistantes ou temps réel. Nous remarquons aussi que les données évoluent suivant des schémas différents qui vont influencer la manière de les manipuler par la suite. Notamment, cela a un impact non négligeable sur le schéma utilisé dans le SGBD relationnel pour le support de persistance.

Par définition, la persistance d'une donnée implique le fait de stocker l'information sur un support. Sa mise à jour sur ce support est une opération considérée comme lente\footnote{Cette lenteur a permit la création des SGFD à la fin des années 90.}. Ainsi, il sera difficile de supposer possible le fait d'avoir une représentation physique acceptable de toutes les données du système.

Les données persistantes sont considérés en quatre dynamiques divisées en deux catégories. Tout d'abord, celles que nous qualifions de meta-données qui sont rassemblés dans des catalogues (donc, des relations persistantes). Celle-ci est composé de deux classes de dynamiques :
\begin{itemize}
	\item[\textbf{Statique}] Cette meta-donnée ne changera jamais par essence. Son utilisation en interrogation continue sera similaire à une relation temporelle $R$ figée : $R^{t_0}$. Le numéro de série d'un équipement est une information qui par nature est immuable.
	\item[\textbf{Stable}] Cette meta-donnée est considéré la plupart du temps comme figée. Elle n'est toutefois pas immuable par essence et peut subir des modifications. Bien que son utilisation soit avant-tout une interrogation instantannée, son utilisation en interrogation continue est une relation temporelle $R$ n'ayant subit aucune manipulation temporelle. Un paramètre de configuration d'un équipement du réseau local est considéré comme stable.
\end{itemize}

La deuxième catégorie rassemble les données évoluant en temps réel. Ces données sont issus de flux de données. Elles peuvent posséder deux dynamiques :
\begin{itemize}
	\item[\textbf{Périodique}] L'historique de cette donnée forme un flux régulier. Son interrogation continue passe par l'application d'une fenêtre dont le contenu n'est pas limité à un \textit{batch}. En effet, la régularité induite par cette donnée implique qu'il est plus important d'observer son évolution plutôt que sa valeur présente. Le relevé des débits d'une carte réseau constitue une donnée périodique.
	\item[\textbf{Imprévisible}] Cette donnée n'a pas de motif d'évolution défini a priori. Son comportement incontrôlable fait que chaque nouvelle donnée du flux a son importante. Son utilisation en requête continue est donc faite par l'application d'une fenêtre $[B]$ décrivant le dernier \textit{batch}. La notification de l'arrivée d'un équipement sur le réseau est imprévisible.
\end{itemize}

Le principe important est le fait que ces classes de dynamiques sont manipulables grâce à l'algèbre. Il est possible de figer une donnée à un instant grâce à la manipulation temporelle. Il est possible de former un flux de changement à partir d'une relation stable. De plus, elle traduit une certaine qualité de la donnée car si nous utilisons une donnée d'une classe comme une autre alors nous perdrons de la qualité.
\begin{example}
	Si nous récupérons les notifications d'arrivée des équipements sur le réseau de manière périodique, nous perdons de la qualité en terme de ponctualité. De même si nous considérons un paramètre de configuration comme statique. A l'inverse, nous introduisons du bruit si nous interrogeons de manière périodique la configuration du routeur de la passerelle d'accès à internet.
\end{example}

Il est important de voir que ces classes nous permettent d'imaginer les mécanismes les plus adaptés pour collecter les données. Toutefois, si un mécanisme n'est pas disponible et qu'un autre est utilisé\footnote{\textit{push} absent $\im$ remplacement par un \textit{pull} régulier}, cela nous permet d'en analyser rapidement les conséquences. La figure~\ref{fig:contrib:asteroid:theorie:dynamics} montre des transformations possibles entre les dynamiques grâce à l'algèbre Astral. Les méta-données sont représenté par des relations temporelles $R$ et les données temps-réelles sont des flux non-partitionnables $S$.

\begin{figure}[ht]
    \centering
\tikzstyle{dynamics}=[ellipse,minimum width=3cm,minimum height=1cm,draw=blue!50,fill=blue!20,thick]
\begin{tikzpicture}[>=stealth,->,shorten >=2pt,thick,bend angle=20, node distance=7cm]
\node (relation) {Méta-données};
\node (flux) [below of=relation,node distance=3cm] {Temps-réel};

\node[dynamics] (static) [right of=relation,node distance=4cm]{Statique};
\node[dynamics] (stable) [right of=static] {Stable};
\node[dynamics] (periodic) [below of=static,node distance=3cm] {Périodique};
\node[dynamics] (event) [right of=periodic] {Imprévisible};

\tikzstyle{every node}=[auto]
\path (stable)      edge    node[above]{$R^{t_0}$} (static);
\path (event)       edge    node[near end,above,sloped]{$S[B]^{\tau_S(0)}$} (static);
\path (periodic)    edge[bend left]    node{$S[B]^{\tau_S(0)}$} (static);
\path (static)      edge[bend left]    node[right]{$\RS{r}(R)$} (periodic);
\path (stable)      edge    node[near end,below,sloped]{$\RS{r}(R)$} (periodic);
\path (event)       edge    node{$\RS{r}(S[B])$} (periodic);

\path (stable)      edge[bend left]    node{$\IS(R)$} (event);
\path (event)      edge[bend left]    node{$S[B]$} (stable);
\end{tikzpicture}
\caption{Transformations des différentes dynamiques en Astral}\label{fig:contrib:asteroid:theorie:dynamics}
\end{figure}

\subsection{Modèle physique de la persistance}

\subsection{Représentation dans Astral}
\section{Composants Astronef}\label{sec:contrib:asteroid:composants}


\section{Plan d'exécution}\label{sec:contrib:asteroid:reecriture}

\section{Conclusion}
Après 20 années de recherche, la gestion de flux de données devient désormais suffisamment mature pour être appliqué massivement. Plusieurs produits commerciaux sont d'ailleurs maintenant utilisés en production. Toutefois, nous pouvons nous rendre compte que la complexité théorique de ces systèmes a été sous-estimé. De nombreux modèles ont été décrit pour représenter les flux de données et leurs traitements. Ces modèles sont encore remis en questions aujourd'hui au fur et à mesure des applications concrètes. 

Nous avons vu que le problème d'avoir une bonne connaissance du modèle et du comportement théorique des SGFD est crucial. En l'état, l'intégration des supports persistants reste ad-hoc et assisté par l'utilisateur. Un fonctionnement intégré avec une modélisation générique capable de gérer les deux modes d'interrogations de façon unifiée est donc indispensable pour manipuler correctement flux et relations persistantes. Similairement, les contributions sur l'optimisation de traitement des requêtes sont encore principalement ponctuelles. Afin d'appliquer un traitement efficace pour toute requête, il est nécessaire d'avoir une bonne connaissance théorique du traitement.

Notre contribution technique se concentrera sur trois points principaux :
\begin{itemize}
 \item[\textbf{Modélisation}] : Création d'Astral, algèbre de traitement des requêtes continues sur flux et relations temporelles. Nous accorderons de l'importance sur la prise en compte des problèmes relevés en section~\ref{sec:rw:sgfd:modeles:synthese}. Cette algèbre sera présenté dans le chapitre~\ref{chap:contrib:astral}
 \item[\textbf{Exécution}] : Mise en œuvre de l'intergiciel Astronef pour construire et exécuter efficacement une requête exprimée avec l'algèbre Astral. Ainsi, à partir d'une requête algébrique, il est possible de sélectionner le plan de requête qui semble le plus efficace grâce aux connaissances accumulés. Cette mise en œuvre sera développée dans le chapitre~\ref{chap:contrib:execution}.
 \item[\textbf{Persistance}] : Conception de l'extension Asteroid permettant l'intégration des requêtes continues sur flux et des requêtes sur support relationnel persistant. Ceci permettra de gérer la représentation du système observé ainsi que l'historisation des données dynamiques. Le support mathématique de cette intégration sera supporté par Astral et sa mise en œuvre par Astronef. Cette intégration sera effectuée dans le chapitre~\ref{chap:contrib:persistance}.
\end{itemize}

Grâce à ces contributions, il deviendra possible de mettre en œuvre un système d'observation générique applicable sur tout type de données. L'utilisateur devra exprimer des requêtes dans le langage algébrique Astral. Une fois ces requêtes écrites, nous serons garanti de leur mise en œuvre. Le tableau~\ref{tab:rw:contrib} résume l'ensemble des points d'analyses que nous nous étions fixés en section~\ref{sec:rw:supervision:criteres}.
\begin{table}[!ht]
\criteretabDonnee
    {Relationnel dérivé. Nous réutilisons les principes utilisés dans la gestion de flux et des bases de données.}
    {\good Modèle entité-relation augmenté pour supporter les flux.}
    {\good Requêtes sur tout type de données (flux, relations).}
\criteretabTraitement
    {\good Continue, Instantannée, Mixte}
    {\good Utilisation des requêtes continues des SGFD en tant qu'intégrateur.}
    {\meh Astral : langage de requête algébrique. Un langage purement déclaratif reste toutefois dérivable de ces fondations théoriques.}
    {\good Relationnel avec support \textbf{intégré} du dynamisme des données.}
\criteretabAdaptabilite
    {\good Spécification du modèle du système ainsi que des requêtes d'intégration (algébriques).}
    {\meh Pour l'analyse, la gestion de données multi-dimensionnelles des entrepôts utilisés est utilisé. Pour l'interrogation continue, utilisation d'un opérateur de préférences sur les flux.}
    {\good Infrastructure générique capable de supporter l'ajout d'opérateurs avec plusieurs implémentations.}
    {\good Héritage de l'efficacité des flux de données. Sélection du meilleur plan d'exécution pour chaque requête. Héritage des supports de grands volumes grâce aux entrepôts.}
\caption{Résumé de notre contribution selon nos critères}\label{tab:rw:contrib}
\end{table}
