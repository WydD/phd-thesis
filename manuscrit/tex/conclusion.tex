\begin{savequote}[.5\textwidth]
<< Dear Princess Celestia,

\quad Sometimes you can feel like what you have to offer is too little to make a difference, but today, I learned that everypony's contribution is important, no matter how small. >>
\qauthor{Fluttershy}
\end{savequote}

\chapter{Conclusion et perspectives}\label{chap:conclusion}
\chaptertoc


Devant la popularisation de la technologie au grand public, nous sommes de plus en plus confronté à des systèmes complexes. Afin de pouvoir les maîtriser et les dépanner, nous avons décidé de nous pencher sur l'interrogation de données. En l'état, il existe plusieurs approches capables d'établir un tel système d'observation. Chaque approche possède un avantage propre : la collecte de données, la gestion de la sémantique des données, l'analyse de grand volumes, et le traitement des données en temps réel. Notre orientation s'est faite vers l'approche de la gestion de flux de données car elle est la seule à gérer de façon native le traitement de données en requête continues. D'une manière plus globale, cette thèse a pour but d'enrichir nos connaissances sur la gestions de données issus de systèmes hétérogènes. Les objectifs plus détaillés de la thèse sont rappelés en section~\ref{sec:conclusion:objectifs}. La section~\ref{sec:conclusion:contributions} résume l'ensemble des contributions que nous établies. Enfin, la section~\ref{sec:conclusion:perspectives} présente les perspectives de recherches.

\section{Rappel des objectifs}\label{sec:conclusion:objectifs}
Nous rappelons brièvement ici les objectifs de cette thèse. Ils se résument en quatre points majeurs :
\begin{itemize}
	\item[$\bullet$] \textbf{Hétérogénéité des systèmes}. Nous souhaitons concevoir une solution générique d'observation de système. Ainsi, tout système, que ce soit un réseau de capteur, un \textit{data-center} ou une maison ubiquitaire, nous devons être capable de l'observer.
	\item[$\bullet$] \textbf{Évolution des données}. Contrairement à la gestion de données persistants, le système en observation est dynamique et produit de nouvelles données au fur et à mesure du temps. Ces données sous forme de flux peuvent être stockées pour analyse a posteriori, traitées en temps réel ou croisés avec des données passés ou consolidés pour fournir de larges capacités d'interrogations.
	\item[$\bullet$] \textbf{Hétérogénéité des données}. Les données qui émanent du système sont de tous types et représentent différents fragments du système. Le schéma de ces données diffère en fonction des sources de données. Ainsi, il est nécessaire d'avoir une intégration des données et une capacité de traitement puissante.
	\item[$\bullet$] \textbf{Adaptabilité}. Le système d'observation doit être capable de s'adapter efficacement à son environnement. De plus, chaque utilisateur possède son interprétation du système, et nous devons pouvoir refléter cette personnalisation. Mais cela passe aussi par la rapidité à écrire les processus d'intégrations, et par l'extensibilité de la solution d'interrogation. Le support de l'ajout de fonctions tierces pour les utilisateurs experts permet de fournir des capacités d'interrogation plus spécialisées.
\end{itemize}

Dans cette thèse, nous avons principalement mis en avant la gestion de l'évolution des données et l'adaptabilité grâce aux travaux sur les flux de données. La gestion de l'hétérogénéité des systèmes et des données est assurée par l'expressibilité que nous obtenons grâce au support relationnel et aux requêtes continues.

\section{Résumé des contributions}\label{sec:conclusion:contributions}
La contribution de cette thèse sur l'observation de systèmes se découpe en trois parties. Premièrement, nous avons proposé \textbf{une algèbre} de gestion des requêtes continues et instantanées sur flux et relations. Deuxièmement, nous avons proposé un \textbf{intergiciel extensible} capable d'évaluer des requêtes exprimées grâce à l'algèbre. Enfin, nous avons présenté l'extension de cet intergiciel pour intégrer les supports persistants dans l'expressivité des requêtes.

\subsection{Langage de requête formel pour une interrogation universelle}
Dans la littérature, les langages de requêtes existants dans le cadre de la gestion de flux de données ont montré des lacunes en terme de clarté sémantique. Deux requêtes exécutées sur deux systèmes peuvent donner des interprétations différentes. Notre approche a été de redéfinir entièrement une algèbre de gestion des flux de données avec comme objectif d'être indépendant du système d'implémentation pour avoir des expressions de requêtes claires et sans possibilité d'interprétations divergentes.

Le langage algébrique \textit{Astral} présenté dans le chapitre~\ref{chap:contrib:astral} est un dérivé de l'algèbre relationnelle pour les flux de données. Les connaissances concernant la manipulation, mais aussi les équivalences de requêtes peuvent être réutilisées. Comme beaucoup d'autres langages de l'état de l'art, cette algèbre sépare les notions de flux et de relation temporelles. Cette approche permet de clarifier les requêtes en interdisant par exemple les opérations flux$\to$flux sans passer par le domaine relationnel via un opérateur de fenêtre.

Astral possède trois avantages : des fondations solides, une expressivité accrue et une intégration instantanée — continue. En effet, les définitions fondamentales d'Astral formalisent les notions d'ordres et d'équivalences de requêtes de manière déterministes. Ces fondations ont permis la spécification d'opérateurs sans ambiguïtés, mais aussi elles ont permis de prouver des résultats non triviaux comme l'asymétrie de la jointure, ou l'équivalence de requête à temps de départs différents (la transposabilité).

L'expressivité d'Astral a permis de rassembler les propositions actuelles en une algèbre intégrée. Par exemple, une grand quantité de sémantiques d'exécution de l'opérateur de fenêtre sont possibles dans notre modèle. De plus, la formalisation de l'opérateur de manipulation temporelle permettant de sélectionner un état passé d'une relation temporelle a permis l'intégration des requêtes continues et instantanées. Cette intégration est notre pierre angulaire pour pouvoir gérer l'hétérogénéité de dynamisme des données du système. La validation de ces points forts a été démontrée dans le chapitre~\ref{chap:validation:expressivite}.

\subsection{Intergiciel d'évaluation de requête extensible}
Nous sommes désormais capables d'écrire toutes requêtes sur des flux ou relations. Toutefois, nous devons avoir un système capable d'évaluer ces expressions Astral. L'intergiciel \textit{Astronef}, présenté en chapitre~\ref{chap:contrib:astronef} permet de mettre en œuvre une telle évaluation. Son architecture interne est dirigée par les architectures à composants orientés services. Cette approche permet d'ajouter, enlever, reconfigurer des composants à tout moment. Cela rend l'approche extensible en terme d'architecture.

Astral est un langage indépendant du système d'implémentation et Astronef définit ses composants en terme algébrique. En effet, chaque composant opérateur doit définir son équivalent algébrique en fonction de sa configuration. Ainsi, un moteur de règle permet de transformer une expression algébrique en plan d'exécution.

Cette transformation se fait en deux étapes, comme dans les SGBD. Tout d'abord, l'expression est réécrite pour être plus optimale en terme de taille de résultats intermédiaires. Cette réécriture utilise les résultats d'équivalences de requêtes données par les preuves faites avec Astral. Ensuite, à partir de la nouvelle expression, un ensemble de règles permet de sélectionner un plan de requête efficace par l'utilisation d'heuristiques (détection de motifs). La validation de cette construction de plan de requête a été expérimentée dans le chapitre~\ref{chap:valid:perfs}. 

\subsection{Intégration du support persistant à l'intergiciel}
Le langage Astral nous permet d'intégrer les requêtes continues et requêtes instantanées. Nous avons développé une extension à Astronef capable de coupler un SGBD aux traitements des requêtes continues : \textit{Asteroid}.

Cependant, l'intégration des données persistantes dans le cadre de l'observation de système n'est pas anodine. En effet, nous avons remarqué que les données persistantes et les données temps réel pouvaient avoir des motifs d'évolutions différents (statique, stable, périodique et imprévisible). Ces dynamiques permettent de conditionner la place de la donnée dans le schéma de la base de données ainsi que les traitements associés. Ainsi, nous avons développé un schéma en deux sections, les données représentant le système (schéma descriptif, contenant des données stables et statiques), et les archives de flux (schéma historique, contenants des données périodiques et imprévisibles).

Ainsi, nous avons développé les notions théoriques élémentaires pour manipuler le support persistant. Nous avons intégré ces notions dans Astronef par le développement de plusieurs composants. Les premiers sont des puits d'écritures capables d'insérer des nouvelles données dans le schéma descriptif (composants dédiés) ou historiques (composants génériques).

Ensuite, nous avons décrit le comportement d'une source capable de représenter toute requête relationnelle sous forme de relation temporelle Astronef. Cette relation temporelle subit des mises à jour régulières suivant un mode de rafraichissement spécifique (périodique, événementiel, \textit{trigger}). Cette source est configurable et grâce à la formalisation Astral ainsi qu'aux règles Astronef, nous pouvons effectuer le placement du plus grand nombre d'opérateurs sur le SGBD pour en exploiter ses capacités. Un dernier composant a été décrit pour permettre d'effectuer par un SGBD une jointure entre une relation temporelle et une relation classique. Le choix du meilleur plan a été validé dans la section~\ref{sec:valid:perfs:couplage}.

L'ensemble du système d'observation a été mis en œuvre et expérimenté sur le réseau local domestique afin d'en explorer son expressivité. L'instance de ce système, baptisé \textit{DomVision}, a été détaillée dans le chapitre~\ref{chap:valid:domvision}. Nous démontrons notamment que nous pouvons effectuer des intégrations complexes de données hétérogènes en termes de schéma comme en terme de dynamique. Nous arrivons de plus à former des flux d'alertes capables de comparer une valeur actuelle (temps réel) avec une valeur moyenne calculée sur l'historique (persistante).

\subsection{Gestion des préférences}
Le système observé va être surveillé par plusieurs utilisateurs dont les intérêts peuvent diverger. Afin de pouvoir gérer les points de vue de chacun, nous introduisons un moyen de personnaliser les résultats d'une requête dans le chapitre~\ref{chap:prefs}. Nous intégrons ainsi deux nouveaux opérateurs \textbf{Best} et \textbf{KBest} capables d'effectuer cette tâche.

Chaque utilisateur exprime ses préférences contextuelles dans un profil et ces opérateurs adapteront les résultats à ce profil. Comme notre solution d'observation est capable d'interroger des données venant de flux ou de relations persistantes, alors nous sommes capables de gérer les préférences sur ces deux supports de manière intégrée. Nous avons créé deux implémentations pour ces opérateurs en exploitant ou non l'évaluation incrémentale des données. Enfin, une évaluation de performances a permis de montrer les conditions où l'évaluation incrémentale est plus efficace. Cet ajout de fonctionnalité démontre de plus l'extensibilité d'Astronef à intégrer un nouvel opérateur.

\section{Perspectives de recherches}\label{sec:conclusion:perspectives}
Avec l'avènement des mouvements tels que le \textit{BigData} ou l'\textit{Internet des objets}, la gestion des données a subi un renouveau depuis quelques années. Nous avons pu apporter notre contribution toutefois, il reste de nombreux défis scientifiques à relever en continuité de notre travail. Nous avons identifié quatre pans de recherches à explorer sur le court et long terme.

\subsection{De la représentation des données d'un système}
Dans notre analyse de l'état de l'art, nous avons vu que peu d'approches permettaient de \textit{modéliser} un ensemble de flux de données. Dans le chapitre~\ref{chap:contrib:asteroid}, nous avons présenté le schéma physique d'Asteroid. Ce schéma est découpé en deux sections. Tout d'abord, le schéma descriptif représentant par un schéma normalisé le modèle du système. Ensuite, le schéma historique permet d'archiver une liste d'historiques. Cette approche est avant tout dirigée par l'implémentation.

En effet, cette modélisation nous permet d'ajouter des historiques de flux à volonté, toutefois, le lien entre les flux et les concepts est établi par un unique identifiant. De façon similaire, nous avons considéré que les flux de données étaient des flux indépendants, sans liens entre eux a priori.

Or, dans la conception d'une base de données, le développeur identifie ses données, établit des dépendances fonctionnelles, et construit un schéma respectant ces dépendances (entité relation ou modèle \textit{UML}). Dans notre cas, nous n'avons pu opérer de la sorte pour deux raisons : (1) la notion de dépendance fonctionnelle n'a pas été définie pour les flux (2) nous ne concevons pas les sources de données, nous subissons ce que le système nous fournit.

Dans Asteroid, nous avons séparé la gestion des données volatiles de la description du système. Cette séparation s'est faite à cause des dynamiques différentes. Est-il toutefois possible de créer une modélisation pour gérer nativement ces classes de dynamiques ? Il serait ainsi possible d'intégrer des sources ayant des classes de dynamiques conflictuelles. Par exemple, une donnée de lieu peut être statique sur certains objets et volatile sur d'autres. Cet exemple produirait un historique de flux et une entrée dans le schéma descriptif du système qu'il faudrait gérer à la main lors de l'interrogation du système.

En dehors des approches de base de données classiques, le domaine de la gestion de contexte, comme nous l'avons montré, fournit plusieurs outils pour effectuer des représentations de données hétérogènes. Être capable de formaliser un contexte uniforme permettrait d'utiliser notre approche dans des applications plus larges (\textit{context-aware softwares}, \textit{service level checking},...).

\subsection{De l'expressivité des langages de requêtes}
Astral est un langage algébrique capable d'interroger, de manière instantanée ou continue, des flux ou des relations (temporelles). Ce langage est très expressif et permet de manipuler de manière claire et déterministe nos entités comme nous l'avons démontré dans le chapitre~\ref{chap:validation:expressivite}.

Néanmoins, nous n'avons en l'état que très peu d'outils pour analyser l'expressivité d'un langage de requête dans le cadre des flux en dehors de la comparaison avec les outils actuels. Pour l'algèbre relationnelle, nous savons sa limitation grâce à la logique du premier ordre (en excluant la récursion). Dans le cas d'Astral, nous n'avons pas de moyens en l'état pour quantifier grâce à des opérateurs logiques la classe des résultats possibles.

Une fois cette expressivité connue, de la même façon que le \textit{SQL} a été défini pour l'algèbre relationnelle, nous pouvons établir un langage déclaratif aussi expressif qu'Astral. L'expression en termes déclaratifs permet aux experts métiers du système, non-spécialiste de la gestion de flux de données, de manipuler aisément les données. Sachant que nous sommes garantis d'une mise en œuvre efficace à partir de l'expression algébrique d'une requête, il nous faut désormais faciliter l'accès à ces capacités.

\subsection{De la performance de l'observation}
Nous avons présenté avec Astronef, dans le chapitre~\ref{chap:contrib:astronef}, une méthode générique, automatique et extensible pour évaluer de manière efficace les requêtes continues. Nous avons prouvé son efficacité dans le chapitre~\ref{chap:valid:perfs}. Toutefois, les limites de cette approche peuvent être rapidement atteintes. En effet, lorsque deux règles sont applicables, alors l'une est privilégiée à l'autre.

De même, lors de la jointure de plusieurs relations, l'optimisation de requête classique de SGBD a montré que l'approche par règle n'est pas suffisante. La recherche de solutions optimales par algorithmes de programmation dynamique permet en général la résolution de ce problème. Toutefois, il est nécessaire de prédire les tailles des relations. Ces approches sont difficiles et deviennent importantes lorsque les expressions de requêtes sont générées grâce à un langage déclaratif.

Afin d'envisager l'observation de plusieurs millions d'entités, il est nécessaire d'établir des solutions de stratégies globales. Nous avons établi dans cette thèse des règles capables de rendre l'évaluation d'\textbf{une} requête efficace. Dans le cadre où des centaines de requêtes sont établies sur plusieurs nœuds distribués, une optimisation locale ne suffit plus. Ainsi, il est nécessaire de concevoir des algorithmes de partage de requête et de calcul à l'exécution de nouveaux plans de requêtes distribués. Notre formalisation de l'équivalence de requête transposée nous permet de mettre en pratique le partage automatique de plan. Toutefois, de nouvelles approches sont à envisager pour pouvoir effectuer un passage à l'échelle des infrastructures de gestions de flux de données.

\subsection{De l'analyse et la compréhension d'un système}
Dans cette thèse, nous avons conçu une méthodologie générique pour gérer les données d'un système. Cette gestion avait pour but de mieux comprendre le système et éventuellement en cas de problème, de pouvoir le diagnostiquer. Nous avons pu mettre en pratique cette solution sur un problème en production chez \textit{Orange France}.

Il a été observé sur quelques centaines de \textit{Livebox} en France un problème récurrent de coupure de service VoIP pour raisons inconnues. Nous avons été contactés pour aider à la résolution de ce problème. Les experts métiers nous ont donné accès à l'ensemble des données de configuration de l'appareil (un accès \textit{pull} et un accès événementiel), soit environ 10~000 paramètres. Nous avons installé notre prototype sur le réseau d'une \textit{Livebox} présentant le problème pour tracer les événements ainsi que les changements des données de configuration, collectées toutes les deux minutes (minimum supportable).

Nous avons pu effectuer plusieurs analyses a posteriori grâce au stockage des données sur Asteroid. Nous n'avons toutefois pas pu résoudre le problème. Nous avons pu observer les conséquences du problème en remarquant que la VoIP devenait inactive. Toutefois, nous n'avons pas pu \textbf{comprendre} le problème, ni en trouver la \textbf{cause}. À cet échec nous pouvons conclure trois possibilités : notre approche n'est pas correcte, nous n'avons pas su traiter les données correctement ou les données qui nous auraient permis de conclure n'étaient pas accessibles. Nous ne pouvons répondre à de telles affirmations. Toutefois, nous avons pu exploiter les capacités que pouvaient nous fournir nos sources de données, ce qui a permis la compréhension d'autres phénomènes en dehors du problème en question.

La compréhension d'un système est un domaine à part entière. Ce domaine permet de fournir des méthodes et des outils, issus des analyses statistiques et des intelligences artificielles multiagents, pour mieux appréhender la complexité d'un système. Nous avons fourni un outil pour observer les données d'un système. Un croisement avec ces approches permettrait d'obtenir une vraie démarche de diagnostic.

