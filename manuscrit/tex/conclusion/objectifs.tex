\section{Rappel des objectifs}\label{sec:conclusion:objectifs}
Nous rappelons brièvement ici les objectifs de cette thèse. Ils se résument en quatre points majeurs :
\begin{itemize}
	\item[$\bullet$] \textbf{Hétérogénéité des systèmes}. Nous souhaitons concevoir une solution générique d'observation de système. Ainsi, tout système, que ce soit un réseau de capteur, un \textit{data-center} ou une maison ubiquitaire, nous devons être capable de l'observer.
	\item[$\bullet$] \textbf{Évolution des données}. Contrairement à la gestion de données persistants, le système en observation est dynamique et produit de nouvelles données au fur et à mesure du temps. Ces données sous forme de flux peuvent être stockées pour analyse a posteriori, traitées en temps réel ou croisés avec des données passés ou consolidés pour fournir de larges capacités d'interrogations.
	\item[$\bullet$] \textbf{Hétérogénéité des données}. Les données qui émanent du système sont de tous types et représentent différents fragments du système. Le schéma de ces données diffère en fonction des sources de données. Ainsi, il est nécessaire d'avoir une intégration des données et une capacité de traitement puissante.
	\item[$\bullet$] \textbf{Adaptabilité}. Le système d'observation doit être capable de s'adapter efficacement à son environnement. De plus, chaque utilisateur possède son interprétation du système, et nous devons pouvoir refléter cette personnalisation. Mais cela passe aussi par la rapidité à écrire les processus d'intégrations, et par l'extensibilité de la solution d'interrogation. Le support de l'ajout de fonctions tierces pour les utilisateurs experts permet de fournir des capacités d'interrogation plus spécialisées.
\end{itemize}

Dans cette thèse, nous avons principalement mis en avant la gestion de l'évolution des données et l'adaptabilité grâce aux travaux sur les flux de données. La gestion de l'hétérogénéité des systèmes et des données est assurée par l'expressibilité que nous obtenons grâce au support relationnel et aux requêtes continues.
