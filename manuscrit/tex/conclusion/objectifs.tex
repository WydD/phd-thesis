\section{Rappel des objectifs}\label{sec:conclusion:objectifs}
Actuellement, il existe plusieurs approches capables d'établir un tel système d'observation. Chaque approche possède un avantage propre : la collecte de données par les systèmes d'administration, la gestion de la sémantique des données par l'informatique contextuelle, l'analyse de grands volumes par les entrepôts de données, et le traitement des données en temps réel par la gestion de flux de données. Notre orientation s'est faite vers ce dernier domaine, car elle est la seule à gérer de façon déclarative le traitement de données en requête continue. D'une manière plus globale, cette thèse a pour but d'enrichir nos connaissances sur la gestion de données issues de systèmes dynamiques hétérogènes.

Nous souhaitons concevoir une solution générique d'observation de système. Nous souhaitons résoudre deux points importants : gérer l'hétérogénéité des données issues du système observé ; et permettre le système d'observation à s'adapter facilement.

\vspace{1ex}\noindent\textbf{Système observé : hétérogénéité des données}

\vspace{1ex}
Nous souhaitons pouvoir observer différents types de systèmes. Cela implique que la description des systèmes en terme de schéma conceptuel de données est hétérogène. Les données qui émanent du système sont de tous types et représentent différents fragments du système. Le schéma de ces données diffère en fonction des sources de données. Ainsi, il est nécessaire d'avoir une intégration des données et une capacité de traitement puissante.

Le système en observation est dynamique et ses données évoluent au fur et à mesure du temps. Il existe deux catégories de données : les données persistantes et les données temps réel. Ces dernières peuvent être stockées pour analyse a posteriori, traitées en temps réel, croisées avec des données passées ou consolidées. Ainsi, il est important de permettre d'interroger les différentes données.

\vspace{1ex}\noindent\textbf{Système d'observation : adaptabilité aux besoins}

\vspace{1ex}
Le système d'observation doit être capable de s'adapter efficacement à son environnement. Afin de fournir une solution flexible, il est préférable d'avoir un langage déclaratif. Ce langage doit toutefois avoir un pouvoir d'expression permettant de gérer l'hétérogénéité du dynamisme des données.

De plus, chaque utilisateur possède son interprétation du système, et nous devons pouvoir faciliter cette personnalisation. Il est également important que le système d'observation soit extensible. Par exemple, le support de l'ajout de fonctions tierces pour les utilisateurs experts permet de fournir des capacités d'interrogation plus spécialisées.

Dans cette thèse, nous avons principalement mis en avant la gestion de l'évolution des données et l'adaptabilité grâce aux travaux sur les flux de données. La gestion de l'hétérogénéité des schémas conceptuels est assurée par la capacité à interroger aussi bien les données temps réel que persistantes.
