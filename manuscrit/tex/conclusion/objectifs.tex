\section{Rappel des objectifs}\label{sec:conclusion:objectifs}
Nous rappelons brièvement ici les objectifs de cette thèse. Ils se résument en 4 points majeurs :
\begin{itemize}
	\item[$\bullet$] \textbf{Hétérogénéité des systèmes}. Nous souhaitons concevoir une solution générique d'observation de système. Ainsi, tout système, que ce soit un réseau de capteur, un \textit{data-center} ou une maison ubiquitaire, nous devons être capable de l'observer.
	\item[$\bullet$] \textbf{Évolution des données}. Contrairement à la gestion de données persistants, le système sous observation est dynamique et produit de nouvelles données au fur et à mesure du temps. Ces données sous forme de flux peuvent être stockées pour analyse a posteriori, traitées en temps réel ou croisés avec des données passés ou consolidés pour fournir des larges capacités d'interrogations.
	\item[$\bullet$] \textbf{Hétérogénéité des données}. Les données qui émanent du système sont de tous types et représentent différents fragment du système. Le schéma de ces données diffèrent en fonction des sources de données. Ainsi, il est nécessaire d'avoir une intégration des données et une capacité de traitement puissante.
	\item[$\bullet$] \textbf{Adaptabilité}. Le système d'observation doit être capable de s'adapter efficacement à son application. De plus, chaque utilisateur possède son interprétation du système, et nous devons pouvoir refléter cette personnalisation. Mais cela passe aussi par la rapidité à écrire les processus d'intégrations, et par l'extensibilité de la solution d'interrogation. Le support de l'ajout de fonction tierces pour les utilisateurs experts permet de fournir des capacités d'interrogation plus spécialisés pour le système en question.
\end{itemize}