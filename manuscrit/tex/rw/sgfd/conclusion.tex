\section{Conclusion}\label{sec:rw:sgfd:synthese}
Après 20 années de recherche, la gestion de flux de données devient désormais suffisamment mature pour être appliquée massivement. Plusieurs produits commerciaux sont d'ailleurs maintenant utilisés en production. Toutefois, nous pouvons nous rendre compte que la complexité théorique de ces systèmes a été sous-estimée. De nombreux modèles ont été décrits pour représenter les flux de données et leurs traitements. Ces modèles sont encore remis en questions aujourd'hui au fur et à mesure du déploiement d'applications concrètes. Toutefois, les modèles sont bien souvent liés au système d'implémentation et certaines sémantiques d'exécutions ne sont pas claires. Ainsi, l'exécution de la même requête sur deux systèmes différents peut provoquer des interprétations différentes.

Nous avons vu que l'infrastructure des SGFD a reçu beaucoup d'attention en terme d'architecture, de tolérance aux fautes, d'intégration et de support du passage à l'échelle. Néanmoins, la compréhension de la sémantique des composants du SGFD est encore limitée. Ainsi, il est difficile d'intégrer deux SGFD, car ils ne partagent pas le même langage et que l'interprétation des langages peut être conflictuelle.

L'intégration des supports persistants reste ad hoc et souvent assistée par l'utilisateur. Les expressions de requêtes hybrides entre flux et relations persistantes sont faites par l'écriture de deux requêtes pour chaque système. Il est nécessaire d'avoir un langage capable d'exprimer les deux types d'interrogations, mais aussi de faire la jointure entre les deux mondes.

Les contributions sur l'optimisation de traitement des requêtes sont nombreuses. Mais peu de travaux~\cite{Galpin:snee,Kramer:semantics} proposent des optimisations de plan de requêtes similaires aux SGBD (optimisation logique puis physique). Dû au manque de connaissances sur les équivalences de requêtes, seules les règles simples telles l'application de projections au plus près des sources sont faites dans l'optimisation logique. Il est nécessaire d'avoir une \textit{recherche d'optimisation plus approfondie} sans intervention de l'utilisateur, ce qui est limité en l'état.

\chapter*{Présentation de la contribution}
Notre contribution se focalise sur trois points principaux :
\begin{itemize}
 \item[\textbf{Modélisation}] : Création d'Astral, algèbre de traitement des requêtes continues sur flux et relations temporelles. En particulier, les définitions sont indépendantes du système d'implémentation. Cela permet l'optimisation et la médiation de systèmes. Cette algèbre est présentée dans le chapitre~\ref{chap:contrib:astral}. Son expressivité ainsi que la démonstration d'équivalences de requêtes sont présentées dans le chapitre~\ref{chap:validation:expressivite}.
 \item[\textbf{Exécution}] : Mise en œuvre de l'intergiciel Astronef pour construire et exécuter efficacement les requêtes exprimées avec l'algèbre Astral. Ce moteur intègre un constructeur de plan de requête sélectionnant un assemblage de composants efficace pour exécuter une expression algébrique. Cette mise en œuvre est développée dans le chapitre~\ref{chap:contrib:astronef}.
 \item[\textbf{Persistance}] : Conception de l'extension Asteroid permettant l'intégration des requêtes continues sur flux et des requêtes sur support relationnel persistant. Ceci permet de gérer la représentation du système observé ainsi que l'historisation des données dynamiques. Il devient possible d'interroger via le même langage, les données temps réel et persistantes. Le support formel de cette intégration est effectué par Astral et sa mise en œuvre par Astronef. Ces travaux sont présentés dans le chapitre~\ref{chap:contrib:asteroid}.
\end{itemize}

Ces contributions sont validées dans les chapitres~\ref{chap:valid:domvision} et~\ref{chap:valid:perfs}. De plus, nous proposons une extension permettant de \textbf{personnaliser} les résultats des requêtes dans le chapitre~\ref{chap:prefs}. Ainsi, face à la masse de données auquel l'utilisateur est confronté, il sera en mesure de réduire le volume de données à restituer selon ses préférences.

Grâce à l'ensemble de ces contributions, il devient possible de mettre en œuvre un système d'observation générique applicable sur un large ensemble de données. L'utilisateur exprime des requêtes dans le langage algébrique Astral. Les requêtes sont ensuite exécutées par Astronef-Asteroid. 

\begin{figure}[ht]
	\fbox{\begin{minipage}{\linewidth}
	\vskip 5cm
	\centering TODO
		\vspace{5cm}
	\end{minipage}}
	\caption{Présentation de la contribution de cette thèse}
\end{figure}

%\begin{table}[!ht]
%\criteretabDonnee
%    {Relationnel dérivé. Nous réutilisons les principes utilisés dans la gestion de flux et des bases de données.}
%    {\meh Modèle entité relation auquel sont attachés des flux.}
%    {\good Requêtes sur tout type de données (flux, relations).}
%\criteretabTraitement
%    {\good Continue, Instantannée, Mixte}
%    {\good Utilisation des requêtes continues des SGFD en tant qu'intégrateur.}
%    {\meh Astral : langage de requête algébrique. Un langage purement déclaratif reste toutefois dérivable de ces fondations théoriques.}
%    {\good Relationnel avec support \textbf{intégré} du dynamisme des données.}
%\criteretabAdaptabilite
%    {\good Spécification du modèle du système ainsi que des requêtes d'intégration (algébriques).}
%    {\meh Pour l'analyse, la gestion de données multidimensionnelles des entrepôts utilisés est utilisée. Pour l'interrogation continue, utilisation d'un opérateur de préférences sur les flux.}
%    {\good Infrastructure générique capable de supporter l'ajout d'opérateurs avec plusieurs implémentations.}
%    {\good Héritage de l'efficacité des flux de données. Sélection du meilleur plan d'exécution pour chaque requête. Héritage des supports de grands volumes grâce aux entrepôts.}
%\caption{Résumé de notre contribution selon nos critères initiaux}\label{tab:rw:contrib}
%\end{table}
