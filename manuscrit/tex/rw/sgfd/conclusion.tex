\section{Conclusion}\label{sec:rw:sgfd:synthese}
Après 20 années de recherche, la gestion de flux de données devient désormais suffisamment mature pour être appliquée massivement. Plusieurs produits commerciaux sont d'ailleurs maintenant utilisés en production. Toutefois, nous pouvons nous rendre compte que la complexité théorique de ces systèmes a été sous-estimée. De nombreux modèles ont été décrits pour représenter les flux de données et leurs traitements. Ces modèles sont encore remis en questions aujourd'hui au fur et à mesure du déploiement d'applications concrètes. Toutefois, les modèles sont bien souvent liés au système d'implémentation et certaines sémantiques d'exécutions ne sont pas claires. Ainsi, l'exécution de la même requête sur deux systèmes différents peut provoquer des interprétations divergentes.

Nous avons vu que l'infrastructure des SGFD a reçu beaucoup d'attention en terme d'architecture, de tolérance aux fautes, d'intégration et de support du passage à l'échelle. Néanmoins, la compréhension de la sémantique des composants du SGFD est encore limitée. Ainsi, il est difficile d'intégrer deux SGFD, car ils ne partagent pas le même langage et que l'interprétation des langages peut être conflictuelle.

L'intégration des supports persistants reste ad hoc et souvent assistée par l'utilisateur. Les expressions de requêtes hybrides entre flux et relations persistantes sont faites par l'écriture de deux requêtes pour chaque système. Il est nécessaire d'avoir un langage capable d'exprimer les deux types d'interrogations, mais aussi de faire la jointure entre les deux mondes.

Les contributions sur l'optimisation de traitement des requêtes sont nombreuses. Mais peu de travaux~\cite{Galpin:snee,Kramer:semantics} proposent des optimisations de plan de requêtes similaires aux SGBD (optimisation logique puis physique). Dû au manque de connaissances sur les équivalences de requêtes, seules les règles simples telles l'application de projections au plus près des sources sont faites dans l'optimisation logique. Il est nécessaire d'avoir une \textit{recherche d'optimisation plus approfondie} sans intervention de l'utilisateur, ce qui est limité en l'état.
