\chapter{Gestion de données pour l'observation}
\minitoc

La supervision est un domaine très actif du fait de ses implications concrètes. Plusieurs solutions ont déjà été proposées pour résoudre les différentes problématiques détaillées en section~\ref{sec:intro:problematique}. En dehors des systèmes d'observations ad-hoc qui ne peuvent pas répondre à l'hétérogénéité conséquente à laquelle cette thèse fait face, plusieurs approches ont étés développées. Ce chapitre présente les quatre grandes familles que nous avons pu identifier.

L'analyse se fera sur les trois critères qui ont une importance cruciale pour l'établissement d'une solution générique, à savoir : le modèle de données utilisé, la capacité de traitement possible et enfin la capacité à s'adapter à l'application finale.

\TODO{plan}

\section{Critères d'analyse}
