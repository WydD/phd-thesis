\section{Critères d'analyse}\label{sec:rw:supervision:criteres}
L'analyse est établie sur les trois critères sont important pour l'établissement d'une solution générique. Les représentations abstraites des données utilisées sont détaillées en section~\ref{sec:rw:supervision:criteres:structure}. La capacité de traitement des différentes approches est présentée en section~\ref{sec:rw:supervision:criteres:traitement}. Enfin, l'adaptation à l'application finale est développée en section~\ref{sec:rw:supervision:criteres:adaptation}.

\subsection{\critereA}\label{sec:rw:supervision:criteres:structure}
Cette thèse suppose que les données sont accessibles au dessus d'une couche d'abstraction. Cependant, aucune hypothèse n'a été formulée sur la structure de ces données ou leurs formats. Ce premier critère d'analyse se focalise donc sur le formalisme utilisé lors de l'acquisition des données. Trois sous-critères sont décrits : la structure abstraite, la gestion de l'hétérogénéité sémantique et enfin le support de l'évolution des données.

\subsubsection{\critereAA}
Un modèle de données abstrait permet de décrire une représentation abstraite d'un ensemble de données. Des exemples classiques de ce type de modèles sont : relationnel, objet, sémantique, à dimension ou encore hiérarchique. Ce modèle abstrait permet par la suite de définir tout autant la capacité de représentation des données que ses capacités intrinsèques en terme de traitement. Ainsi, le choix de cette structure est central pour la gestion de l'hétérogénéité des données.

\subsubsection{\critereAB}
Afin d'être capable de représenter la structure du système, il est nécessaire de définir un modèle de données sémantique. La structure abstraite n'étant qu'un support à cette représentation. La structuration du modèle définira ainsi la sémantique accordé au système. Par exemple, en gestion de base de données, le modèle entité-relation détermine la représentation logique du système. Elle est par la suite traduite de façon physiques en tables. En système d'information, la représentation d'un système passe souvent par un modèle objet. Cette capacité à représenter le système est critique pour manipuler les concepts et leurs liens de manière claire.

\subsubsection{\critereAC}
La valeur obtenue à un instant particulier d'une donnée n'est pas toujours pertinante. En effet, pour plusieurs types d'informations, la pertinence de la donnée est liée à son évolution au fur et à mesure du temps. Les différents types de dynamiques ont été présentés en section~\ref{sec:intro:problematique:data}. La capacité à ce modèle de représenter les différentes évolutions sera important à l'intégration des différentes sources de données hétérogènes.

\subsection{\critereB}\label{sec:rw:supervision:criteres:traitement}
Afin d'évaluer la capacité du système de supervision à fournir les réponses les plus précises à l'utilisateur, il faut établir des processus de traitements sur les données. Suivant les systèmes, cette capacité à collecter et transformer les données des sources n'a pas la même puissance. Ainsi, il est nécessaire d'analyser plus en profondeur les capacités de ce procédé. Nous détaillerons ceci en quatre critères principaux : les types d'interrogation possibles, l'intégration de sources, le langage d'expression et enfin le pouvoir expressif de ce langage.

\subsubsection{\critereBA}
La création des processus de traitements de données est assimilable à une interrogation que pose l'utilisateur sur l'ensemble des données. Il existe différentes natures d'interrogations (ou requêtes). Celles-ci reflètent l'hétérogénéité de la dynamique des données.
\begin{itemize}
    \item \textbf{Interrogation instantanée} : il constitue le paradigme usuel d'interrogation utilisé notamment dans les applications de gestion de base de données. L'utilisateur pose une question sur un ensemble de données considérées figées, du moins le temps du calcul de la réponse. Le système fournit une réponse représentatif d'un état à un instant donnée. Un exemple simple étant : \enquote{\it quel est l'ensemble actuel des capteurs de température de mon système}. La réponse sera \enquote{\it à cet instant, les capteurs 1, 4 et 42 sont des capteurs de température du système}. La mention \enquote{\it à cet instant} est très importante, car si un nouveau capteur arrive dans le système, la réponse deviendra erronée. Il est bien évidemment possible d'imaginer revenir dans le passé et poser une question sur un état antérieur. La consultation ponctuelle de l'ensemble de données par l'utilisateur est donc une application de cette interrogation.
    \item \textbf{Interrogation continue} : principal acteur des systèmes événementiels, ce paradigme considère que les données sont en constante évolution. L'utilisateur obtiendra ainsi une réponse qui évoluera au cours du temps, sous forme de flux ou de mise à jour d'état. Un exemple pouvant être : \enquote{\it le flux de température moyenne sur une minute du capteur 42}. La réponse sera un flux continu d'information qui, toutes les minutes, reportera une nouvelle valeur moyenne pour ce capteur. Ainsi, la formation de processus de collecte ou de formation d'alerte sont des applications de ce type d'interrogation.
\end{itemize}
Il est important de noter que ces deux grands paradigmes d'interrogation peuvent se combiner. Par exemple, il est possible d'effectuer un appel à une interrogation instantanée à l'intérieur d'un processus continu. De façon similaire, l'appel régulier d'une interrogation instantanée forme une réponse continue. Il est donc nécessaire que le système d'observation soit capable de manipuler naturellement ces deux types d'interrogations pour manipuler correctement le dynamisme des données.

\subsubsection{\critereBB}
Pour permettre une grande compréhension du système et de ses interactions, il est nécessaire de corréler les différentes sources d'informations en une seule base d'information intégrée. En effet, chaque source de donnée peut-être considérée comme un fragment de cet ensemble intégré. Ainsi, le système doit se doter de fonctionnalités d'agrégations de plusieurs sources. Celles-ci doivent être suffisamment génériques pour permettre de supporter toutes sources et tout type d'intégration.

\subsubsection{\critereBC}
L'expression des requêtes possibles se fait à travers d'un langage. Son paradigme sous-jacent définira la manière et la facilité d'adaptation à un système en particulier. Ce langage peut être dans le cas le plus extrême : un langage de programmation impératif bas niveau (comme le C par exemple). Dans ce cas, l'approche sera très algorithmique, permettant une meilleure gestion des performances, mais une utilisation plus difficile et technique par la suite. À l'autre extrême, le langage peut être issu de la programmation logique permettant des performances moins contrôlées, mais une gestion globale plus déclarative, permettant une grande flexibilité.

\subsubsection{\critereBD}
Le langage utilisé dans le système d'observation n'a pas forcément pour vocation a être un langage \textit{Turing-complet}. Ainsi, son expressivité sera peut-être limitée. Il est important d'être capable d'énumérer ce qui pourra être, ou non, possible d'exprimer en terme de processus. Par exemple, les opérations de manipulation de données pourront être limités dans la logique du premier ordre, ou dans le relationnel.

\subsection{\critereC}
Un critère de qualité d'un système d'observation est sa capacité à se modeler à l'application finale. Au plus, la portée de l'observation sera générique au plus ce critère sera important. Cette section décrit les différentes facettes de l'adaptabilité. Tout d'abord d'un point de vue de gestion de l'hétérogénéité décrite précédemment avec l'adaptation au système et la gestion des différentes perspectives. Ensuite, nous présenterons des critères permettant l'adaptabilité aux plus stricts besoins tels que l'intégration de fonctions métiers et la gestion de performances.

\subsubsection{\critereCA}\label{sec:rw:supervision:criteres:adaptation}
Comme décrite précédemment, la gestion de l'hétérogénéité des différents types de systèmes est centrale pour l'établissement d'une solution générique. Ce critère devra décrire comment la supervision spécifie sa représentation du système. Il sera considéré comme meilleur si le nombre et la complexité des procédures pour s'adapter au système visé est faible. Car si un système est complet mais dont l'adaptation est longue et complexe, il sera difficile à mettre en application.

\subsubsection{\critereCB}
Suivant les observateurs, le système de supervision n'a pas le même intérêt. Ainsi, il est nécessaire que le système d'observation ai prévu de quoi adapter le résultat des interrogations en fonction du point de vue de l'expert métier qui l'utilisera.

\subsubsection{\critereCC}
Il existe certains traitements propres à chaque expertise qui peuvent être difficiles ou impossibles à mettre en place. Il est ainsi nécessaire que le système de supervision soit capable d'intégrer des routines procédurales spécifiques de traitement de données. Cette extensibilité permet aussi bien l'intégration de tous types de besoins que d'améliorer les performances de traitements récurrents.

\subsubsection{\critereCD}
Afin d'être déployable dans le plus grand nombre de contextes différents, il est nécessaire que le système de supervision soit efficace. De plus, ce critère améliore la qualité des réponses aux différentes requêtes de l'utilisateur. En effet, l'amélioration des performances inclura la réduction des coûts de temps de traitement. La réponse sera ainsi disponible plus rapidement à l'utilisateur et reflétera une vision plus à jour des données. Le critère de qualité se mesure sur la capacité à traiter la charge d'un système (en terme de temps de calcul ou de nombre de données supportable). Une classe de complexité algorithmique pourra être citée pour mieux évaluer la charge.
