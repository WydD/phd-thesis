\section{Critères d'analyse}\label{sec:rw:supervision:criteres}
L'analyse est établie sur les trois critères sont important pour l'établissement d'une solution générique. Les représentations abstraites des données utilisées sont détaillées en section~\ref{sec:rw:supervision:criteres:structure}. La capacité de traitement des différentes approches est présentée en section~\ref{sec:rw:supervision:criteres:traitement}. Enfin, l'adaptation à l'application finale est développée en section~\ref{sec:rw:supervision:criteres:adaptation}.

\subsection{\critereA}\label{sec:rw:supervision:criteres:structure}
Cette thèse suppose que les données sont accessibles au dessus d'une couche d'abstraction. Cependant, aucune hypothèse n'a été formulée sur la structure de ces données ou leurs formats. Ce premier critère d'analyse se focalise donc sur le formalisme utilisé lors de l'acquisition des données. Trois sous-critères sont décrits : le modèle de données, le schéma contextuel et enfin le support de l'évolution des données.

\subsubsection{\critereAA}
Un modèle de données permet de décrire une représentation abstraite d'un ensemble de données. Des exemples classiques de ce type de modèles sont : relationnel, objet, sémantique, à dimensions ou encore hiérarchique. Ce modèle permet par la suite de définir tout autant la capacité de représentation des données que ses capacités intrinsèques en terme de traitement. Ainsi, le choix de ce modèle est central pour la gestion de l'hétérogénéité des données.

\subsubsection{\critereAB}
Afin d'être capable de représenter le système, il est nécessaire de définir un schéma de données. Ce schéma détermine la sémantique accordé au système. Par exemple, en gestion de base de données, un modèle entité-relation détermine la représentation logique du système. Elle est par la suite traduite de façon physique en tables. En système d'information, la représentation d'un système passe souvent par un modèle objet. Cette capacité à représenter le système est critique pour manipuler clairement les concepts et leurs liens.

\subsubsection{\critereAC}
La valeur obtenue à un instant particulier d'une donnée n'est pas toujours pertinente. En effet, pour plusieurs types d'informations, la pertinence de la donnée est liée à son évolution au fur et à mesure du temps. La capacité du modèle de données à représenter et à traiter cette évolution est important à l'intégration des différentes sources de données hétérogènes.

\subsection{\critereB}\label{sec:rw:supervision:criteres:traitement}
Afin d'évaluer la capacité du système d'observation à fournir les réponses les plus précises à l'utilisateur, il faut analyser les processus de traitements des données. Suivant les systèmes, cette capacité à collecter et transformer les données des sources n'a pas la même puissance expressive. Nous détaillons ceci en quatre critères principaux : les types d'interrogation possibles, l'intégration de sources, le langage d'expression et enfin le pouvoir expressif de ce langage.

\subsubsection{\critereBA}
La création des processus de traitements de données est assimilable à une interrogation que pose l'utilisateur sur l'ensemble des données. Il existe différentes natures d'interrogations (ou requêtes). Celles-ci reflètent l'hétérogénéité de la dynamique des données.
\begin{itemize}
    \item \textbf{Interrogation instantanée} : C'est la manière usuelle d'interroger dans les applications de gestion de base de données. L'utilisateur pose une question sur un ensemble de données considérées figées, du moins le temps du calcul de la réponse. Le système fournit une réponse représentative d'un état à un instant donnée. Un exemple simple étant : \enquote{\it quel est l'ensemble actuel des équipements actifs de mon système}. La réponse à cette requête pourrait être \enquote{\it à cet instant, les équipements Box, PC1 et STB sont connectés et actifs}. La mention \enquote{\it à cet instant} est très importante, car si un nouvel équipement arrive dans le système, une nouvelle évaluation donnerait une réponse différente. Ainsi, cette interrogation correspond à la consultation ponctuelle de l'ensemble des données disponibles.
    \item \textbf{Interrogation continue} : C'est l'élément principal des systèmes événementiels, où les données sont considérés en constante évolution. L'utilisateur obtient ainsi une réponse qui évoluera au cours du temps, sous forme de flux ou de mise à jour d'état. Un exemple pouvant être : \enquote{\it le flux de la charge processeur moyenne sur une minute du PC1}. La réponse forme un flux continu d'information qui, toutes les minutes, reporte une nouvelle valeur moyenne pour ce capteur. Ainsi, la formation de processus de collecte ou de formation d'alerte suivent ce type d'interrogation.
\end{itemize}
Il est important de noter que ces deux grands paradigmes d'interrogation peuvent se combiner. Par exemple, il est possible d'effectuer un appel à une interrogation instantanée à l'intérieur d'un processus continu. De façon similaire, l'appel régulier d'une interrogation instantanée forme une réponse continue. Il est donc nécessaire que le système d'observation soit capable de manipuler naturellement ces deux types d'interrogations pour manipuler correctement le dynamisme des données.

\subsubsection{\critereBB}
Pour permettre une bonne compréhension du système et de ses interactions, il est nécessaire d'intégrer les différentes sources d'informations en une seule base d'information. En effet, chaque source de donnée peut-être considérée comme un fragment de cet ensemble. Ainsi, le système doit se doter de fonctionnalités d'agrégations de plusieurs sources. Le système d'observation doit être suffisamment génériques pour permettre l'intégration de toutes les sources de données disponibles.

\subsubsection{\critereBC}
L'expression des traitements possibles se fait à travers d'un langage. Son paradigme sous-jacent définit la manière et la facilité d'adaptation à un système en particulier. Ce langage peut être dans le cas le plus extrême : un langage de programmation impératif bas niveau (comme le C par exemple). Dans ce cas, l'approche est très algorithmique, permettant une meilleure gestion des performances, mais une utilisation plus difficile et technique par la suite. À l'autre extrême, le langage peut être issu de la programmation logique permettant des performances moins contrôlées, mais une gestion globale déclarative, permettant une grande flexibilité.

\subsubsection{\critereBD}
Le langage utilisé dans le système d'observation peut avoir une capacité d'expression limitée. Il est important d'être capable d'énumérer ce qui est possible, ou non, d'exprimer en terme de processus. Les classes de logiques ou les équivalences à d'autres langages permettent de caractériser ces limitations. Par exemple, les opérations de manipulation de données pourrait être limitées par la logique du premier ordre, ou par le calcul relationnel.

\subsection{\critereC}
Un critère de qualité d'un système d'observation est sa capacité à s'adapter à l'application finale. Plus la portée de l'observation est générique plus ce critère est important. Cette section décrit les différentes facettes de l'adaptabilité. Le premier point de vue est celui de la gestion de l'hétérogénéité décrite précédemment avec l'adaptation au système et la gestion des différentes perspectives. Ensuite, nous présentons des critères permettant l'adaptabilité aux besoins tels que l'intégration de fonctions métiers et la gestion de performances.

\subsubsection{\critereCA}\label{sec:rw:supervision:criteres:adaptation}
La gestion de l'hétérogénéité des différents types de systèmes est centrale pour l'établissement d'une solution générique. La solution générique est considéré comme meilleure si le nombre et la complexité des procédures nécessaires à l'adaptation au système visé sont faible. Car si un système est complet mais nécessite une adaptation longue et complexe, il devient difficile à mettre en œuvre.

\subsubsection{\critereCB}
Suivant les observateurs, le système d'observation n'a pas le même intérêt. Ainsi, il est nécessaire que le système d'observation soit prévu pour adapter le résultat des interrogations en fonction du point de vue de l'expert métier qui l'utilise.

\subsubsection{\critereCC}
Il existe certains traitements propres à chaque expertise qui peuvent être difficiles ou impossibles à mettre en place de façon générique. Il est ainsi nécessaire que le système de supervision soit capable d'intégrer des routines spécifiques de traitement de données. Cette extensibilité permet aussi bien l'intégration de tous types de besoins que l'amélioration des performances de traitements récurrents.

\subsubsection{\critereCD}
Afin d'être déployable dans le plus grand nombre de contextes différents, il est nécessaire que le système d'observation soit efficace. De plus, ce critère améliore la qualité des réponses aux différentes requêtes de l'utilisateur. En effet, l'amélioration des performances permet la réduction des coûts de temps de traitement. La réponse devient ainsi disponible plus rapidement et reflète une vision plus à jour des données. Le critère se mesure sur la capacité à traiter la charge d'un système en terme de nombre de sources ou en terme de débit supportés.

Nous avons détaillé les différents critères d'analyse de l'état de l'art. Nous commençons notre étude par l'analyse des systèmes d'administration.