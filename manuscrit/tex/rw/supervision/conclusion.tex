\section{Conclusion}
Ce chapitre a dressé un état de l'art des différents systèmes capables d'offrir une solution générique de supervision. Il en ressort qu'aucun système ne supporte entièrement les critères de qualité que nous nous sommes fixés. Le tableau~\ref{tab:rw:supervision:bilan} résume les 11 points d'analyse en colorant les différentes points suivant leurs conformités. 

Il en ressort que les systèmes d'administrations sont avant tout des systèmes qui fonctionnent grâce au support des standards et à leur simplicité d'implémentation. L'architecture avec gestionnaire adaptable grâce à des langages impératifs permets une grande flexibilité pour s'adapter aux cas d'usages. De son côté, l'informatique contextuelle fournit des outils permettant de modéliser et manipuler proprement les concepts du système grâce aux ontologies et aux raisonnements logiques. Il en sort une claire séparation des domaines de compétences, mais une complexité de traitement. Les entrepôts de données quant à eux se distinguent par des capacités d'analyses très poussées, ainsi qu'un procédé d'intégration, très complexe et lourd malheureusement, mais très complet. Enfin, la gestion de flux de données est une base solide pour gérer les données dynamiques. L'intégration et l'adaptation au système est fait entièrement de manière déclarative en fait une solution performante et viable.

À la vue de l'état de l'art, voici les points qui vont être critique sur notre établissement de notre contribution :
\begin{itemize}
    \item La gestion de flux est un bon socle pour gérer les données dynamique grâce aux requêtes continues.
    \item Elle ne suffit pas pour constituer un système d'observation complet, notamment à l'absence de schéma contextuel et de requêtes instantanées.
    \item Les entrepôts et bases de données sont capables de répondre aux requêtes instantanées.
    \item Les ETL sont trop complexes à manipuler pour intégrer les données, alors que les SGFD sont plus déclaratifs.
\end{itemize}
Il devient clair que les systèmes de gestions de flux de données forment un bon candidat comme fondation pour un architecture d'observation de système. Il nous faut approfondir l'état de l'art technique sur ce domaine pour modeler notre contribution. Le point majeur est d'apporter les capacités des systèmes de gestions de données relationnels. En effet, en apportant le support persistant à la gestion de flux de données, en clarifiant et augmentant son langage, nous obtenons un outil plus apte à répondre à notre problématique. Ainsi, l'héritage du relationnel permet une structure sémantique correcte, ainsi que des capacités d'analyses plus évoluées. Enfin et surtout, les données serait intégrés malgré leur hétérogénéité profonde. Le chapitre suivant détaille l'état de l'art technique de la gestion de flux de données afin de pouvoir effectuer ces améliorations.

\begin{sidewaystable}[ht]
\centering
\begin{tabular}{@{{\vrule width 1pt}\ \ }>{\raggedleft}m{3cm}@{\ \ {\vrule width 1pt}\ \ }M{4.2cm}|M{4.2cm}|M{4.2cm}|M{4.2cm}@{\ \ {\vrule width 1pt}}} \bottomrule
\head Critère & \head Système d'administration & \head Gestion de contexte & \head Entrepôts de données & \head Gestion de flux de données \\  \toprule \bottomrule
\critereAA & Hiérarchique & Triplets & Relationnel & Relationnel dérivé \\ \hline
\critereAB & \meh Structure hiérarchique sans contraintes & \good Ontologies & \good Modèle relationnel normalisé & \bad Pas de structure \\ \hline
\critereAC & \meh Notifications & \bad Ajout du temps en propriété & \meh CDC & \good Flux natif \\ \toprule \bottomrule
\critereBA & \meh Instantanée, continu en ad-hoc & \bad Instantané principalement & \meh Instantané. ETL en pseudo-continu & \bad Continu uniquement \\ \hline
\critereBB & \good Standardisation, union de modèles & \meh Fusion d'ontologies non standardes & \good Processus ETL (complexe) & \good Union et jointures de flux \\ \hline
\critereBC & \bad Impératif principalement & \good Logique & \meh Déclaratif (SQL) et Procédural (ETL) & \good Déclaratif principalement\\ \hline
\critereBD & \meh Procédures à écrire soi-même & \good Logique du premier ordre & \good Relationnel multidimensionnel et Algorithmie dédiée & \meh Relationnel avec support du dynamisme\\ \toprule \bottomrule
\critereCA & \good Support des standards & \meh Spécification longue des domaines & \bad Spécification du schéma, des ETL, autre (complexe) & \good Écriture de requêtes \\ \hline
\critereCB & \bad Aucune & \good Séparation par les domaines & \good Données multidimensionnelles & \bad Aucune \\ \hline
\critereCC & \good Modèle extensible, fonctions métiers dans le gestionnaire & \meh Capteurs virtuels & \good Opérateurs ETL, procédures SQL, algorithmes & \meh Sources et puits mais pas les opérateurs  \\ \hline
\critereCD & \good Large échelle & \bad Complexité très haute & \meh Réactivité lente, Support de grande quantité & \good Support de haut débits\\ \toprule 
\end{tabular}
\caption{Récapitulatif de l'état de l'art des systèmes génériques de supervision}\label{tab:rw:supervision:bilan}
\end{sidewaystable}