\section{Définitions générales}
\subsection{N-uplets et identifiants}
\begin{defi}[n-uplet]
    Un n-uplet est une fonction partielle de l'ensemble des attributs vers l'espace des valeurs. Le domaine de cette fonction est appellé le schéma du n-uplet.
\end{defi}

\begin{defi}[Identifiant physique]
    L'identifiant physique d'un n-uplet $s$ est un élément de l'espace des identifiants $\I$ isomorphe à $\N$. Le nom d'attribut de cet identifiant sera noté $\varphi$ et $s(\varphi)$ désigne la valeur de l'identifiant de $s$.
\end{defi}

\begin{defi}[Séquence d'n-uplet]
    Un ensemble dénombrable d'n-uplets $TS$ est une séquence si et seulement si : 
    \begin{itemize}
     \item Tout n-uplet de $TS$ partage le même schéma $A$ contenant $\varphi$.
     \item $\forall s,s' \in TS^2$, $s\neq s' \equ s(\varphi) \neq s'(\varphi)$
    \end{itemize}

    Une séquence est donc naturellement totalement ordonné par son identifiant physique.
\end{defi}


\begin{defi}[Timestamp]
    L'espace temps $\T$ est un corps totalement ordonné et isomorphique à $\R$. 

    Un \textit{timestamp} est un élément de $\T$.
\end{defi}


\begin{defi}[Identifiant de batch]
    Un identifiant de batch est un élément de l'ensemble $\TN$ totalement ordonné par lexicographie.
\end{defi}

\begin{defi}[Flux]
    Un flux est un couple $(S,\BS)$ tel que :
    \begin{itemize}
        \item $S$ est une séquence d'n-uplet potentiellement infinie possédant un schéma contenant l'attribut spécial $\t$.
        \item $\BS$ est une fonction $S\mapsto \TN$ définissant le batch d'appartenance d'un n-uplet
    \end{itemize}
\end{defi}

