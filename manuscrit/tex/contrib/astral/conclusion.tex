\section{Conclusion}\label{sec:contrib:astral:conclusion}
Dans ce chapitre, nous avons présenté un nouveau modèle de gestion de flux de données. Celui-ci s'établit comme un modèle entièrement déterministe car toutes les définitions permettent d'obtenir une réponse claire et mathématique à une requête. Ainsi, une implémentation de système pourra décrire exactement son exécution sur ce modèle théorique.

Si nous reprenons les points que nous avions décrit en section~\ref{sec:rw:sgfd:modeles:synthese}. Nous pouvons dors et déjà répondre à plusieurs des problèmes qui avaient été identifié. Nous avons une clareté de la gestion de l'ordre. Ce point nous a conduit à reformuler certaines définitions classiques (telles que la jointure ou l'union) et en découvrir des comportements encore non-détaillés dans la littérature (telle que l'asymétrie de ces opérateurs). Nous avons définit strictement la notion d'équivalence de requête, autant à \textit{timestamp} de départ fixe, que différent. Ce qui nous permettra par la suite, de formuler mathématiquement des preuves exactes.

Quant au couplage relationnel, nous avons mis en place l'opérateur de manipulation temporelle applicable sur les relations temporelles. Ce qui permet de transformer une requête continue en requête statique, ou autre. Ceci va être la pierre angulaire du couplage possible avec les relations issues d'un système de gestion de base de données. 

Nous allons désormais présenter la mise en œuvre de cette algèbre dans le chapitre suivant.
