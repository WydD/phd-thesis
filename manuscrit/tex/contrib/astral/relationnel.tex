\section{Héritages du modèle relationnel}
La définition de relation temporelle que nous avons exposé comporte la notion de séquence d'n-uplet. Cette notion est certes proche des relations classiques mais possède un point majeur supplémentaire étant l'ordre. Dans cette section, nous verrons comment réutiliser les opérateurs de l'algèbre relationnelle.

\subsection{Opérateurs unaires simples}
Tout d'abord explorons le domaine des opérateurs unaires relationnels : sélection, projection et renommage. Comme ces opérateurs sont agnostiques de l'ordre dans lesquels sont les n-uplets, le principe de l'héritage est d'appliquer les définitions sur l'évaluation instantanée de la relation.

Par exemple, notons la sélection relationnelle classique $\Sigma$. Alors, pour un batch $b$ quelconque, l'expression suivante : $\Sigma(R(b))$, exprime bien la sélection des n-uplets. Ainsi l'application de l'opérateur relationnel standard sur le batch présent permet de définir la sélection (def~\ref{def:selection}). L'identifiant physique n'est pas altéré donc l'ordre ne l'est pas non plus.
\begin{defi}[Sélection]\label{def:selection}
Soit $R$ une relation temporelle,

Soit $c$ une expression booléenne applicable sur tout n-uplet de $R$,

Alors la sélection est définie comme suit :
$$\sigma_{c}(R) : b \mapsto \{s\in R(b), c(s)\} = \Sigma_c(R(b))$$
\end{defi}

Nous pouvons remarquer d'ores et déjà que la définition d'inclusion de requête est directement appliquable à la sélection (en prenant pour fonction d'extraction l'identité).
\begin{prop}[Inclusion de la sélection]
Soit $R$ une relation temporelle, et $c$ une condition de sélection, alors $$\sigma_c R \subseteqq R$$
\end{prop}

La projection et le renommage se définissent de façon similaire. Toutefois, il existe des cas pouvant altérer l'identifiant physique. Par exemple, la projection sur des attributs ne comprennant pas $\varphi$ le supprimerait. Nous instaurons donc des règles supplémentaires (def~\ref{projection}) pour éviter ces cas. De façon similaire, nous pourrions définir l'opérateur d'évaluation d'expressions $e_f^c$ permettant d'évaluer une expression $f$ dont le résultat serait placé dans l'attribut $c$.
\begin{defi}[Projection et renommage]\label{def:projection}
La projection $\Pi_p$ et le renommage $\rho_{b/a}$ sont défini par héritage de l'algèbre relationnelle à l'exception de ces deux règles :
\begin{itemize}
\item Une projection $\Pi_p$ est strictement égale à $\Pi_{p\cup \{\varphi\}}$
\item Le renommage $\rho_{b/\varphi}$ correspond a une copie de $\varphi$ dans $b$.
\end{itemize}
\end{defi}

Nous avons donc réussi à appliquer les définitions des 3 premiers opérateurs de l'algèbre relationnelle dans notre contexte. Il nous faut maintenant explorer les opérateurs binaires, en commençant par la jointure.

\subsection{Produit cartésien}
La contrainte de l'ordre tend à se faire pesante dans le cadre des opérations binaires. En effet, il nous faut établir un ordre strict sur la séquence d'n-uplets résultants du produit des deux relations. Il est important de noter que cette notion de séquence d'n-uplet est primordiale même pour les relations temporelles (voir notamment les \textit{streamers} def~\ref{def:streamers}).
\begin{example}\label{ex:asymetrie}
Soit $CPU$ une relation $(appId, cpu, \t)$ comportant des relevés fait par l'application $appId$ de la charge $cpu$ d'un processeur au temps $\t$. Soit $Devices$ la relation $(deviceId, appId)$ listant les applications $appId$ exécutés sur le dispositif $deviceId$. Voici un exemple de données :
\begin{center}
\begin{tabular}{ccc}
& deviceId & appId \\ %\hline 
\cline{2-3} & 1 & 2 \\
\textbf{Devices} &2 & 23 \\
&3 & 23 \\
&4 & 12 \\
\end{tabular} \quad \quad \quad
\begin{tabular}{cccc}
& appId & cpu & $\t$ \\% \hline 
\cline{2-4} & 12 & 12 & 21 \\
\textbf{CPU}& 2 & 11 & 32 \\
& 2 & 14 & 48 \\
&12& 13 & 54 \\
\end{tabular}
\end{center}

Supposons que l'utilisateur souhaite obtenir la charge \textit{CPU} des équipements. L'opération demandé est donc une jointure entre ces deux relations. Toutefois, deux solutions sont envisageables.
\begin{center}
\begin{tabular}{cccc} 
        deviceId & cpu & $\t$ \\ \hline 
        1&  11&  32  \\
        1&  14&  48  \\
        4&  12&  21 \\
        4&  13&  54\\
\end{tabular}
\quad \quad \quad
\begin{tabular}{cccc}
        deviceId & cpu & $\t$ \\ \hline 
        4&  12&  21\\
        1&  11&  32\\
        1&  14&  48\\
        4&  13&  54\\
\end{tabular}
\end{center}

Dans le premier cas, les n-uplets sont listés par dispositifs, puis par \textit{timestamp}. Dans le second cas, les n-uplets sont listés par \textit{timestamp}. Si ce résultat est transformé en flux, il peut y avoir des impacts sémantiques lourds : fenêtres positionnelles ou \textit{load-shedding} différents. Mais de plus, le coût d'aggrégation éventuel sera lui aussi impacté (tri par groupement déjà effectué).
\end{example}

Ainsi, il est important de clarifier l'ambiguïté latente à la gestion de l'ordre dans les opérations binaires. Intéressons nous au produit cartésien étant au centre des opérations binaires les plus utilisés. En effet, la jointure dite naturelle $\Join$ entre deux relations temporelles est définie comme le produit cartésien avec sélection sur l'égalité des attributs commun. Toute jointure est donc un opérateur composite centré sur le produit cartésien et sur des projections-renommage-sélection.

L'utilisation obligatoire de l'identifiant physique force la définition de l'ordre à tout niveau. Ainsi, le produit cartésien (def~\ref{def:produit}) est similaire au produit classique nonobstant l'utilisation d'une application $\Phi^\times$ à définir permettant la création du nouvel identifiant.
\begin{defi}[Produit Cartésien]\label{def:produit}
Soient $R_1$ et $R_2$ deux relations temporelles telles que $Attr(R_1) \cap Attr(R_2) = \{\varphi\}$, soit $b$ un identifiant de \textit{batch},

Soient $\I^{\times}$ un $\Phi$-espace et $\Phi^\times$ une application de $\I_{R_1}\times\I_{R_2}$ vers $\I^\times$,

Le produit cartésien de $R_1$ par $R_2$ au \textit{batch} $b$ est : $(R_1\times R_2)(b)=$
$$\bigcup_{\begin{array}{c}  r \in R_1(b)\\ s \in R_2(b)\end{array}} \{(\varphi, \Phi^\times(r(\varphi), s(\varphi))) \ \cup \ r[Attr(R_1)\backslash \varphi]\ \cup\ s[Attr(R_2)\backslash \varphi]\}$$
\end{defi}

Sauf mention contraire, dans Astral, nous considérons que $$\Phi^\times : \varphi_1, \varphi_2 \mapsto (\varphi_1,\varphi_2)\in \I^\times=\I_{R_1}\times \I_{R_2}$$ avec $\I^\times$ étant lexicographiquement ordonné (d'abord $R_1$ puis $R_2$). Aucun critère évident ne permet d'affirmer que cette application est meilleure qu'une autre. Ce choix est dirigé par le fait qu'il soit intuitif et qu'il reflete le comportement de l'algorithme usuel de boucles imbriqués (itération sur $R_1$ puis pour chaque n-uplet itération sur $R_2$). Les caractéristiques de cette fonction supplémentaire ont des implications concrêtes sur les propriétés du produit cartésien comme : la propriété d'asymétrie (théorème~\ref{thm:asymetrie}). 
\begin{thm}[Asymétrie du produit cartésien]\label{thm:asymetrie}
    Le produit cartésien ne peut être symétrique dans le cadre général.
\end{thm}

\begin{demo}[du théorème~\ref{thm:asymetrie}]
    Tout $\Phi$-espace est isomorphe à $\N$, par mesure de simplification, nous travaillerons donc dans cet espace. Supposons qu'il existe un ordre total et symétrique $<^2$ sur $\N^2$.

    Soient $a,b\in \N^2$ tels que $a \neq b$. Puisque l'ordre est total, alors $(a,b) <^2 (b,a)$ (ou inversement). Puisque l'ordre est symétrique alors, $(b,a) <^2 (a,b)$ ce qui est absurde.
\end{demo}

Ainsi, dans l'exemple~\ref{ex:asymetrie}, nous avions défini deux réponses à l'opération de jointure. Ces résultats correspondent aux opérations $Devices\Join CPU$ et $CPU \Join Devices$, ce qui illustre bien les problème d'asymétrie. Ce premier résultat est \textbf{majeur} car le choix de l'ordre des jointures est déterminant pour l'optimisation de requête. Toutefois, il est possible de redéfinir les ordres de jointures et obtenir $R_1 \times^1 R_2 = R_2 \times^2 R_1$ mais la définition de chacun des produits cartésiens n'est pas la même du fait d'un choix de $\Phi^\times$ différent. En pratique cela pourra se concrétiser par un tri a posteriori, ce qui peut introduire un surcoût.

\subsection{Union}
La définition de l'union de relations temporelles est aussi complexe à cause de l'identifiant physique encore. En effet, lors de l'union de deux séquences d'n-uplets, il n'est pas directement possible d'extraire une nouvelle séquence. Il est donc nécessaire réécrire sa définition. Le principe réside encore une fois dans une application particulière de réécriture $\Phi^\cup$ qui a pour but de réordonner.
\begin{defi}[Union relationnelle]
    Soient $R_1$ et $R_2$ deux relations temporelles avec le même schéma $A$,

    Soient $\I^{\cup}$ un $\Phi$-espace et $\Phi^\cup$ une application de $\I_{R_1}\cup\{\emptyset\}\times\{\emptyset\}\cup\I_{R_2}$ vers $\I^\cup$, soit $b$ un identifiant de \textit{batch},

    L'union de $R_1$ et $R_2$ au \textit{batch} $b$ est définie par : $(R_1\times R_2)(b)=$ 

        $$\begin{array}{c}
            \bigcup_{\scriptstyle r \in R_1(b)} \left\{ r[A\backslash \varphi] \cup (\varphi, \Phi^\cup(r(\varphi),\emptyset) \right\} \\
            \bigcup_{\scriptstyle s \in R_2(b)} \left\{ s[A\backslash \varphi] \cup (\varphi, \Phi^\cup(\emptyset,s(\varphi)) \right\}
        \end{array}$$
\end{defi}

Deux sémantiques principales peuvent s'appliquer dans le cadre de l'union. Tout d'abord, la sémantique générique que nous appliquons par défaut dans l'algèbre Astral : nous selectionnons d'abord les n-uplets de la séquence de gauche et ensuite ceux de la séquence de droite. $\I^\cup$ est donc égal à $(\I_{R_1}\cup \{\emptyset\})\times (\I_{R_2}\cup \{\emptyset\})$ avec un ordre naturel lexicographique ($\emptyset$ étant la valeure la plus petite possible) et l'application $\Phi^\cup$ est définie par : $$\Phi^\cup(\varphi_1,\varphi_2) = (\varphi_1,\varphi_2)$$

Nous remarquons que cette définition de l'union est elle aussi \textbf{asymétrique} car ce n'est pas une simple union ensembliste. Dans certains cas, il est possible d'avoir une union symétrique. Si l'union de $\I_{R_1}$ et $\I_{R_2}$ forment naturellement un $\Phi$-espace $\I$ et que pour tout \textit{batch} $b$, $R_1(b)$ et $R_2(b)$ ne partagent pas d'identifiant physiques, alors il est possible de définir une union naturelle qui conserve les identifiants physiques : $$\Phi^\cup(\varphi_1,\varphi_2) = \begin{cases} \varphi_1 & ,\ \varphi_1 \neq \emptyset \\\varphi_2 &,\ \varphi_2 \neq \emptyset\end{cases}$$
Ce cas se présente souvent lorsqu'une entité est partagée en multiples sous-entités (souvent appelé partitionnement). Ainsi, les identifiants physiques proviennent du même $\Phi$-espace et sont répartis sur plusieurs relations qui devront subir une union. Nous reverrons l'importance de ces définitions lors de l'établissement des fenêtres.

\textbf{Note} : La différence entre deux relations temporelles est délicate car pour être capable de retrancher un n-uplet d'une séquence, il est nécessaire de l'identifier exactement. L'identifiant physique $\varphi$ est effectivement présent pour ce point. La différence est donc une différence \textbf{ensembliste} pure. Toutefois, il est nécessaire de correctement gérer les identifiants physiques de la séquence à retrancher pour que les $\Phi$-espaces des deux séquences soient identiques et que les identifiants soient pertinents. Par exemple, $(R_1\cup R_2)-R_2$ donnera $R_1\cup R_2$ car les identifiants sont de natures différentes. Par contre, $(R_1\cup R_2)-(\Omega \cup R_2)=R_1$ (avec $\Omega$ la relation temporelle vide).

\subsection{Agrégation}
L'agrégation est une opération qui n'a pas été définie dans l'algèbre relationnelle standarde. Toutefois, vu son utilisation fréquente notamment dans le contexte des flux de données, il nous semble pertinent d'en exposer la définition exacte. L'opération consiste en l'application de fonctions d'agrégations sur des sous-groupes formés grâces à un regroupement par attributs égaux. Par exemple, la moyenne des valeurs de charge processeur calculée pour chaque identifiant d'équipement. 
\begin{defi}[Fonction d'agrégation]
    Une fonction d'agrégation $f$ est une application associant : une séquence d'n-uplet $S$ et un attribut $A$ à une valeur agrégée $f(S,A)$.
\end{defi}
\begin{example}
    La fonction d'agrégation de moyenne ($\avg$) est définie par $$\avg(S,A) = \frac{\sum_{s\in S} s(A)}{\# S}$$
\end{example}

Comme tout opérateur d'Astral, il est nécessaire de manipuler proprement l'identifiant physique. Celui-ci est défini par un agrégat particulier aux séquences d'n-uplets : 
$$\last(S,A) = s(A) \qquad \textrm{ avec } s\in S \textrm{ tel que }\pos_{S}(s) = \#S-1$$
Nous pouvons désormais définir l'opérateur d'agrégation comme l'application de fonctions d'agrégations aux sous-groupes créés et dont l'identifiant physique est calculé par l'application de la fonction $\last$.
\begin{defi}[Opérateur d'agrégation]
    Soit $R$ une relation temporelle de schéma $A$,

    Soient $a_1,...,a_n$, $n$ attributs de $A$,

    Soient $f_1,...,f_m$, $m$ fonctions d'agrégats, $b_1,...,b_m$, $m$ attributs de $A$, et $c_1,...,c_m$, $m$ attributs,

    L'opérateur d'agrégat $_{a_1,...,a_n}\G_{{f_1}_{b_1}^{c_1},...,{f_m}_{b_m}^{c_m}}$ au \textbf{batch} $b$ est égal à :
$$\left\{g(\sigma_{a_1=v_1,...,a_1=v_n} R(t,i),v_1,...,v_n), v_1 \in \mathrm{Dom}(a_1), ..., v_n \in \mathrm{Dom}(a_n)\right\} $$
$$\begin{array}{c}\textrm{avec } g(S,v_1,...,v_n) = \{(\varphi,\last(S,\varphi))\}\cup_{i=1}^n \{(a_i, v_i)\} \cup_{i=1}^m \{(c_i,f_i(S,b_i))\}\\ \textrm{ if } A \neq \emptyset,\ \emptyset\textrm{ else}\end{array}$$
\end{defi}

\begin{example}
    La requête continue calculant les valeurs moyennes sur la relation temporelle $CPU$ est donc $_{id}\G_{\avg_{cpu}^{moyenne}}(CPU)$.
\end{example}
