\section{De l'importance de l'observation de système}\label{sec:intro:contexte}
L'informatique a évolué de façon drastique au cours de ces dernières années. Grâce à l'amélioration des technologies de la micro-électronique, du réseau, et des techniques logicielles, il est désormais possible de concevoir des systèmes complexes et repartis tels que :
\begin{itemize}
 \item Un réseau de capteurs~\cite{Akyildiz:wsn,Szewczyk:monitoring}. Composé de nombreux micro-dispositifs autonomes, ou à grande durée de vie, il est capable de transmettre sans-fil des informations sur des mesures physiques. Ensembles, les capteurs permettent de créer un système de surveillance dans des applications telles que l'agriculture de précision~\cite{Jurdak:sumac}.
 \item Un environnement domestique intelligent~\cite{Harper:smarthome, Chan:smarthome, Coyle:assisted}. Grâce aux technologies sans-fils ou courant-porteur, des dispositifs sont capables de communiquer afin de fournir des services de haut-niveaux. De tels services peuvent être multimédia comme par exemples le partage de flux vidéos~\cite{Kang:upnpav} ou la visiophonie~\cite{Vilei:videophone}. Ils peuvent être orienté sur l'amélioration du confort de vie de l'usager comme la gestion automatique des lumières ou encore l'assistance aux personnes malades ou handicapées~\cite{Korhonen:health}.
 \item Un centre de traitement de données. La mise en relation de centaines d'équipements informatiques permet de créer une infrastructure capable de supporter des traitements complexes pour virtualiser des services à très large échelle ou pour établir une ferme de calcul scientifique.
\end{itemize}

%    Interactions nombreuses et volatiles
Ainsi, l'ordinateur personnel n'est plus qu'une partie des possibles interactions qu'un utilisateur peut établir avec l'informatique. Ces systèmes partagent la caractéristique principale suivante~: ce sont des dispositifs qui interagissent via un réseau pour fournir un ou des services de haut niveau. Ces interactions sont nombreuses et potentiellement volatiles. Afin de mieux comprendre ces systèmes, il est nécessaire de les observer.

%    Établissement d'un diagnostic
L'\textbf{observation} d'un système est un processus en charge de collecter, traiter, et éventuellement archiver les données d'un système en fonctionnement pour en vérifier son bon déroulement. Cette surveillance peut être faite en temps réel afin d'être réactif aux événements importants, ou par analyse a posteriori sur l'ensemble des données collectées. Ainsi, lorsqu'un problème surgit sur un système ce procédé est au cœur de sa résolution. En effet, le traitement en temps réel peut détecter l'anomalie et peut transmettre l'information à qui est capable d'y remédier (l'utilisateur ou un système tiers). Si les données sont archivés, alors il devient possible de retracer l'origine du problème par le parcours des données collectées afin d'établir un diagnostic.

%    Aide à l'administration
Dans le domaine de l'administration du système, le terme de \textit{supervision} est souvent évoqué. Ce processus permet de surveiller le système et d'agir en conséquence en modifiant par exemple la configuration de chacun des dispositifs et services. L'observation est une base précieuse d'informations pour l'aide à la gestion du système car cela permet à l'administrateur d'avoir une vue détaillée du fonctionnement de son système.

%    Autoconfiguration
De plus, avec l'émergence des intergiciels pour la construction de services de haut-niveau a révélé plusieurs problèmes sur la gestion de qualité~\cite{Geihs:challenges}. Un des aspects critique de ce domaine est la capacité à se reconfigurer à la demande. Afin de pouvoir s'auto-adapter aux différentes environnements, il est nécessaire que l'intergiciel soit conscient d'un ensemble de paramètres qui lui permette de prendre la bonne décision. Ces paramètres sont fournis par un système d'observation.

Il existe de nombreux produits commerciaux ou académiques pour permettre la supervision ou l'observation. Ces solution permettent la surveillance d'infrastructures réseaux (d'un point de vue réseau~\cite{url:zabbix} ou applicatif~\cite{url:manageengine}), la gestion de processus opérationnels d'entreprises~\cite{url:systar} ou encore l'administration d'un parc de dispositifs~\cite{IETF:SNMP}. Toutefois, le système d'observation doit être capable de répondre à la diversité et la grandissante complexité des systèmes informatiques. L'enjeu principal de cette thèse est la capacité d'un système d'observation à s'adapter à l'hétérogénéité des systèmes, des dispositifs et des données. La section suivante présente plus en détail la problématique de cette thèse.