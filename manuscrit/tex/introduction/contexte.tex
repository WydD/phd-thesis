\section{De l'importance de la supervision}\label{sec:intro:contexte}
%    Systèmes de plus en plus complexes

L'informatique a évolué de façon drastique au cours de ces dernières années. Grâce à l'amélioration des technologies de la micro-électronique, du réseau, et évidemment des techniques logicielles, il est désormais possible de concevoir des systèmes complexes et repartis tels que :
\begin{itemize}
 \item Un réseau de capteurs~\cite{Akyildiz:wsn, Szewczyk:monitoring}. Composé de nombreux micro-dispositifs autonomes, ou à grande durée de vie, capables de transmettre sans-fil des informations sur une quantité physique. Ensembles, ils permettent de créer un système de surveillance dans des cadres comme l'agriculture de précision~\cite{Jurdak:sumac}.
 \item Un environnement domestique intelligent~\cite{Harper:smarthome, Chan:smarthome, Coyle:assisted}. Grâce aux technologies sans-fil ou courant-porteur, des dispositifs sont capables de communiquer pour fournir des services de haut-niveaux. De tels services peuvent être multimédia tels que le partage de flux vidéos~\cite{Kang:upnpav} ou la visiophonie~\cite{Vilei:videophone}. Mais aussi, ils peuvent être orienté sur l'amélioration du confort de vie de l'usager comme la gestion automatique des lumières ou encore l'assistance aux personnes malades ou handicapées~\cite{Korhonen:health}.
 \item Un centre de traitement de données. Que ce soit pour virtualiser des services à très large échelle ou pour établir une ferme de calcul scientifique, la mise en relation de centaines d'équipements informatiques permet de créer une infrastructure capable de supporter des traitements complexes.
\end{itemize}

%    Interactions nombreuses et volatiles
Ainsi, l'ordinateur personnel n'est plus qu'une partie des possibles interactions qu'un utilisateur quelconque peut faire avec l'informatique. Ces systèmes partagent la caractéristique principale suivante~: ce sont des dispositifs qui interagissent via un réseau pour fournir un ou des services de haut niveau. Ces interactions sont nombreuses et potentiellement volatiles. Afin de mieux comprendre ces systèmes, il est nécessaire de les observer.

%    Établissement d'un diagnostic
La \textbf{supervision} d'un système est un processus en charge de collecter, traiter, et éventuellement archiver les données d'un système en fonctionnement pour en vérifier son bon déroulement\footnote{Plusieurs définitions conflictuelles existent pour le terme de supervision. Par exemple, nous ne considérons pas dans notre définition le fait d'agir sur le système.}. Cette surveillance peut s'effectuer en temps réel afin d'être pro-actif sur la réaction aux événements importants, ou par analyse a posteriori sur l'ensemble des données qui ont été collectées. Ainsi, lorsqu'un problème surgit sur un système ce procédé est au cœur de la résolution de problèmes. En effet, si le traitement est en temps réel pourra détecter l'anomalie et pourra transmettre l'information à qui est capable d'y remédier (l'utilisateur ou un système tiers). Sinon, il sera possible de retracer l'origine du problème par le parcours des données collectées afin d'en établir un diagnostic.

%    Aide à l'administration
La supervision est un outil central lors de l'administration du système. En effet, comme la surveillance passe aussi par la gestion de la configuration de chacun des dispositifs et services, il devient une base précieuse d'informations pour l'aide à la gestion du système car cela permet à l'administrateur d'avoir une vue détaillée du fonctionnement de son système.

%    Autoconfiguration
De plus, avec l'émergence des intergiciels pour la construction de services de haut-niveau a révélé plusieurs problèmes sur la gestion de qualité~\cite{Geihs:challenges}. Un des aspects critique de ce type d'architecture est le fait de pouvoir se reconfigurer à la demande. Afin de pouvoir s'auto-adapter aux différentes environnements, il est nécessaire que l'intergiciel soit conscient d'un ensemble de paramètres qui lui permettrons de prendre la bonne décision. Ces paramètres sont fournis par une solution de supervision.

Toutefois, le processus de surveillance doit être capable de répondre à la diversité et la grandissante complexité des systèmes informatiques. L'enjeu principal étant la capacité du système de supervision à s'adapter à l'hétérogénéité des systèmes, des dispositifs et des données. La section suivante présente plus en détail la problématique de cette thèse.