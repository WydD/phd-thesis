\section{De l'importance de la supervision}\label{sec:intro:contexte}
%    Systèmes de plus en plus complexes

L'informatique a évolué de façon drastique au cours de ces dernières années. Grâce à l'amélioration des technologies de la micro-électronique, du réseau, et évidemment des techniques logicielles, il est désormais possible de concevoir des systèmes tels que :
\begin{itemize}
 \item Un réseau de capteurs. De nombreux micro-dispositifs autonomes, ou à durée de vie longue, capables de transmettre des informations sur une quantité physique sans fils. Ensembles, ils sont capables de créer un système de surveillance dans le cadre par exemple de l'agriculture de précision.
 \item Un environnement domestique intelligent. Grâce aux technologies sans-fil ou courant-porteur, des dispositifs sont capables de communiquer pour fournir des services de haut-niveaux. De tels services peuvent être  multimédia tels que le partage de flux vidéos ou la visiophonie. Mais aussi, ils peuvent être plus orienté sur l'amélioration du confort de vie de l'usager comme la gestion automatique des lumières ou encore l'assistance aux personnes handicapées.
 \item Un centre de traitement de données. Que ce soit pour virtualiser des services à très large échelle ou pour établir une ferme de calcul scientifique, la mise en relation de centaines d'équipements informatiques permet d'effectuer des traitements complexes.
\end{itemize}

%    Interactions nombreuses et volatiles
Ainsi, l'ordinateur personnel n'est plus qu'une partie des possibles interactions qu'un utilisateur quelconque peut faire avec l'informatique. Ces systèmes partagent la caractéristique principale suivante~: ces sont des dispositifs qui interagissent via un réseau pour fournir un service de haut niveau. Ces interactions sont nombreuses et potentiellement volatiles. Afin de mieux comprendre ces systèmes, il est nécessaire de les observer.

%    Établissement d'un diagnostic
La supervision d'un système est le processus en charge de collecter, traiter, et éventuellement stoquer les données relatives à un système pour en vérifier son bon fonctionnement. Cette surveillance peut s'effectuer en temps réel afin d'être pro-actif sur la réaction aux événements importants, ou par analyse a posteriori sur l'ensemble des données qui ont été collectées. Ainsi, lorsqu'un problème surgit sur un système, critique ou non, ce procédé est au cœur de la résolution de problèmes. En effet, si le traitement est en temps réel pourra détecter l'anomalie et pourra transmettre l'information à qui est capable d'y remédier (l'utilisateur ou système tiers). Sinon, il sera possible de retracer l'origine du problème par le parcours des données collectées afin d'établir un diagnostic.

%    Aide à l'administration
Enfin, la supervision est un outil central lors de l'administration du système. En effet, comme la surveillance passe aussi par la gestion de la configuration de chacun des dispositifs et services, il devient une base précieuse d'informations pour l'aide à la gestion du système car cela permet à l'administrateur d'avoir une vue détaillée du fonctionnement de son système.

Toutefois, le processus de surveillance doit être capable de répondre à la diversité et la grandissante complexité des systèmes informatiques. L'enjeu principal étant la capacité du système de supervision à s'adapter à l'hétérogénéité des systèmes, des dispositifs et des données. La section suivante présente plus en détail la problématique de cette thèse.