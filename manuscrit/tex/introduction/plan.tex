\section{Plan de thèse}\label{sec:intro:plan}
Ce manuscrit est découpé en cinq parties majeures :
\begin{enumerate}
\item \textbf{L'état de l'art}. Nous analysons d'abord les solutions actuelles capables de faire de l'observation de systèmes génériques dans le chapitre~\ref{chap:rw:supervision}. Ensuite, nous détaillons les travaux existant dans le domaine plus spécifique de la gestion des flux de données dans le chapitre~\ref{chap:rw:sgfd}.
\item \textbf{Modèle algébrique}. Le chapitre~\ref{chap:contrib:astral} présente les définitions d'\textit{Astral}, notre algèbre de gestion de données. Le chapitre~\ref{chap:validation:expressivite} analyse en détail l'expressivité offerte par \textit{Astral}, en la comparant à l'état de l'art, puis en démontrant des propriétés non-triviales d'équivalences de requêtes. 
\item \textbf{Mise en œuvre et couplage relationnel}. Le chapitre~\ref{chap:contrib:astronef} décrit \textit{Astronef}, un intergiciel capable d'interpréter et évaluer une requête \textit{Astral}. Enfin, nous détaillons dans le chapitre~\ref{chap:contrib:asteroid} l'extension \textit{Asteroid} capable de manipuler un support persistant relationnel, complétant ainsi notre gestion de données pour l'observation.
\item \textbf{Expérimentations}. La mise en œuvre de notre solution d'observation dans le cadre du réseau local domestique est décrite dans le chapitre~\ref{chap:valid:domvision}. Puis, nous présentons dans le chapitre~\ref{chap:valid:perfs} des éléments d'évaluation de performances afin de démontrer que notre système de règle permet d'évaluer de manière efficace : des requêtes continues et des requêtes couplés avec un SGBD.
\item \textbf{Gestion des préférences}. Le chapitre~\ref{chap:prefs} présente une extension à notre contribution introduisant des opérateurs permettant de personnaliser les résultats en fonction de l'utilisateur.
\end{enumerate}

Finalement, le chapitre~\ref{chap:conclusion} conclut ce travail et présente quelques perspectives de recherches.