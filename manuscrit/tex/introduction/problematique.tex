\section{L'hétérogénéité au cœur du problème}\label{sec:intro:problematique}
La supervision est un processus qui est nécessaire et applicable à une grande quantité de domaine. Cela ouvre la porte à la problématique majeure de cette thèse : la diversité des applications. Cela a pour conséquence une hétérogénéité à plusieurs niveau. Cette section développe les trois axes majeurs de diversité en terme de : systèmes (section~\ref{sec:intro:problematique:devices}), besoins d'observation (section~\ref{sec:intro:problematique:observation}) et enfin en terme de données (section~\ref{sec:intro:problematique:data}).

\subsection{Une grande variété de systèmes et dispositifs}\label{sec:intro:problematique:devices}
Plus l'informatique évolue, plus le nombre de dispositifs créés varie. Grâce à l'émergence de l'informatique ubiquitaire, les dispositifs se font de plus en plus nombreux et de natures complètement différentes. Ces caractéristiques rendent leur observation plus délicate car la surveillance des données d'un capteur est sensiblement différente à celle de l'utilisation d'un service d'hébergement web sur un serveur grande capacité. D'une part, les ressources utilisables pour extraire les données seront plus limités sur le capteur, augmentant ainsi le risque d'ingérance dans le fonctionnement du système. D'autre part, la nature des méthodes de collecte des données pourra être très variée, en terme de protocole d'accès, comme de mode d'interrogation (\textit{push} ou \textit{pull}).

D'un point de vue structurel, les dispositifs ne sont pas composés de la même manière. Alors qu'un capteur pourra être modélisé avec quelques variables d'états (niveau de batterie, valeur numérique du capteur, puissance signal radio, ...), un décodeur TV est quant à lui beaucoup plus complexe par essence. En effet, il est composé de modules matériels, d'un système logiciel, de divers applications et de modules de communications. Chacune de ces entités regorge de variables d'états potentiellement à surveiller, suivant l'application. Ainsi, l'hétérogénéité des dispositifs ouvre la voie à une grande diversité d'entités complexes à observer.

De plus, la mise en communication de ces divers dispositifs fait encore augmenter la complexité d'un système complet de manière exponentielle. Comme présenté précédemment, des capteurs sont capables de former un réseau qui peut lui même communiquer avec d'autres équipements domotiques ou avec un serveur d'application distant. Ainsi, la structure même du système est hétérogène car dans le cas présenté, la communication entre les équipements ne se fera pas directement suivant les cas. Ainsi, il est nécessaire de représenter les différentes connections possibles ou établies entre les dispositifs, qui elles-mêmes possèdent des caractéristiques. 

La description des différentes entités du système et leurs liens compose ainsi une structure complexe du système. Cette structure est malheureusement attaché par nature au système et variera d'une application à l'autre.

\subsection{Les perspectives d'observations}\label{sec:intro:problematique:observation}
La supervision ne se justifie que par l'utilisation qui en est faite et par son application. En effet, pour tout système, il existe plusieurs angles de vues possibles. Cet angle définira par la suite les données surveillées mais aussi la représentation du système.

Par exemple, lors de l'observation d'un système domotique. Un expert réseau s'intéressera aux liens entre les dispositifs, à la qualité de ces liens, et évidemment à l'évolution de la topologie. Alors qu'un expert multimédia s'intéressera lui au matériel de TV, aux logiciels d'encodage/décodage et aux intergiciels de partages de contenus. Les conséquences sont majeures car dans le premier cas, la représentation qui sera faite du système sera principalement un graphe annoté (la topologie, les propriétés des nœuds et liens). Alors que dans le deuxième cas, la représentation pourra être un modèle de classe extrait du domaine du multimédia.

Choisir un point de vue permet de restreindre la complexité inhérante au système observé. Toutefois, il est nécessaire que la supervision s'adapte aux perspectives dans lequels se placent les experts. Ainsi une grande flexibilité est nécessaire pour gérer cette diversité de besoins.

\subsection{Les données}\label{sec:intro:problematique:data}
La perspective d'observation fixe ainsi l'ensemble des données qu'il est nécessaire de gérer pour répondre aux besoins d'observations. Cependant, cet ensemble est d'une grande hétérogénéité suivant plusieurs axes :
\begin{itemize}
 \item \textbf{Syntaxe} : Il apparait naturel qu'une grande diversité syntaxique soit présente sur l'ensemble de données. Un capteur renseignera son relevé avec une valeur numérique, tandis qu'une carte réseau donnera son adresse \textit{IP} (v4 ou v6), et un serveur d'application sera capable de fournir un rapport sous forme complexe (document \textit{XML} ou \textit{JSON}). À chacun de ces types, plusieurs opérations deviennent possibles. Une valeur numérique pourra, souvent, être agrégée via des fonctions statistiques, alors qu'un document \textit{XML} pourra être lui décomposé par nœud.
 \item \textbf{Sémantique} : Chaque donnée a une sémantique propre définie par son attachement à différents concepts. Par exemple, la donnée \textit{Statut} indiquant l'état d'un objet, n'aura de sens que si elle est liée à une entité du système comme : le statut du service décodage TV. Des liens sémantiques peuvent être tissés entre les données afin de définir des contraintes ou des règles de fonctionnement. Ainsi, il peut être défini que le statut de l'équipement de décodage TV est intrinsèquement associé au statut du service décodage TV. 
 \item \textbf{Pertinence} : Devant la masse de données pouvant être présente dans le système observé, il est nécessaire d'extraire les quelques éléments pertinents pour répondre aux besoins de l'expert. De ce fait, chaque donnée ne partage pas la même importance aux yeux de l'utilisateur. Dans le cadre de la surveillance de fonctionnement, la notification d'un dépassement de seuil d'une variable d'état critque aura certainement plus de pertinence que le numéro de série d'un équipement. Il est important de noter que ceci est très dépendant de la perspective d'observation car dans le cas de l'aide à l'administration d'un parc de dispositifs, le numéro de série aura beaucoup de valeur.
 \item \textbf{Dynamique} : Une donnée n'est pas toujours statique dans le temps. En effet, son évolution implique des conséquences directe sur sa sémantique et son importance, modifiant ainsi son traitement. Quatre dynamiques principales sont identifiables.
    \begin{itemize} 
        \item \textit{Évenementielle} : la donnée survient uniquement de manière non-prédictible. Ces événements ont de l'importance car ils traduisent un comportement particulier du système pouvant être une alerte ou un changement d'état brusque, ce qui est très pertinent pour de l'observation.
        \item \textit{Régulière} : la mise à jour se fera suivant un schéma connu et prédictible, par exemple, un capteur envoie sa mesure toutes les cinq secondes. L'importance de cette donnée sera visible principalement en considérant l'ensemble de son historique et non la donnée figée à un instant précis.
        \item \textit{Stable} : la donnée ne changera que très rarement, c'est le cas des informations de configurations comme la version d'un logiciel. Lors de l'établissement d'un diagnostic, l'état courant de ce type de variable est pertinent pour analyser. Il est important de noter que pour certaines données critique, l'action de mise à jour peut constituer un événement qui sera nécessaire de traiter.
        \item \textit{Statique} : la donnée ne changera jamais sous toute condition comme le numéro de série d'un équipement, ou les informations constructeurs. Ce type d'information est nécessaire si l'application de surveillance justifie l'utilisation d'une description du système.
    \end{itemize}
\end{itemize}

En conclusion, le système de supervision doit être capable de s'adapter à tout type de système informatique ainsi qu'aux angles de vues des experts métiers. Une fois ce contexte d'utilisation correctement défini, il est nécessaire de gérer la grande diversité de données présente, que ce soit en terme de type, de sémantique ou d'évolution temporelle. Ces différents aspects de la données impliqueront des conséquences directes sur sa pertinence.
