\section{Approche et contributions de cette thèse}\label{sec:intro:demarche}
Nous proposons une approche orientée donnée pour l'observation de systèmes. En particulier, nous nous sommes orientés sur le domaine de la gestion des flux de données. Ceci nous permet de créer un langage algébrique unique d'interrogation pour toutes les catégories de données (persistantes et temps réel). Grâce à ce langage, nous développons un intergiciel capable de gérer les données issues de l'observation. Cette section présente brièvement nos contributions.

\subsection{Un langage unique d'interrogation}
La gestion de flux de données est un domaine permettant de gérer de manière déclarative les données temps-réel. Cette approche permet une grande souplesse, car il est possible de déployer des requêtes complexes via un langage déclaratif sur les données. Toutefois, les langages manquent encore de fondations théoriques pour correctement maîtriser ses données.

Nous avons spécifié une algèbre permettant d'interroger de manière unifiée les données sous forme de flux ou relations de bases de données. Elle a pour propriété d'être capable de manipuler les deux modes d'interrogation sur des données persistantes ou temps réel. De plus, ces définitions sont déterministes. Ainsi, lors de l'expression d'une requête par cette algèbre, il n'existe qu'une interprétation du résultat, ce qui permet une gestion plus claire de ses données.

\subsection{Un intergiciel extensible d'évaluation de requête}
Il existe plusieurs intergiciels capables d'évaluer des requêtes continues sur les systèmes de gestions de flux de données. Toutefois, leur implémentation peut influencer les sémantiques d'évaluations de requête. Des approches existent pour intégrer des évaluateurs d'un point de vue architectural. Toutefois, il existe peu d'approches pour spécifier formellement le comportement d'un opérateur d'un évaluateur.

Nous proposons un intergiciel capable d'évaluer des requêtes exprimées dans le langage algébrique que nous avons développé. Cette algèbre nous sert aussi pour la spécification des composants de l'intergiciel. En effet, chaque module implémentant un opérateur doit spécifier sa sémantique selon une ou plusieurs règles se basant sur des opérateurs algébriques. Ce principe couplé avec les notions architecturales de composants orientés services nous permettent d'avoir une grande flexibilité pour permettre aux utilisateurs de personnaliser au mieux cette solution.

De plus, notre approche à base de règle nous permet de développer une approche générale d'optimisation des requêtes. Tout comme en SGBD, nous appliquons une optimisation logique, puis physique de notre requête. Ces optimisations sont possibles grâce aux résultats d'équivalence de requête démontrable avec l'algèbre.

\subsection{Une intégration des supports persistants}
Enfin, nous avons vu que notre langage est capable d'interroger de manière unifiée les données temps réel et persistantes. Dans la littérature, plusieurs rapprochements existent entre les deux mondes de manière ad hoc ou implicite.

Grâce à notre langage, nous proposons une infrastructure capable d'intégrer un SGBD à notre intergiciel de gestion de flux. Ainsi, l'utilisateur doit spécifier le schéma du système observé, ainsi que des requêtes permettant l'intégration des données flux dans la base. À l'issue de ce travail de spécifications, l'utilisateur devient capable d'interroger les données persistantes et temps réel via l'algèbre. Il est alors garanti de la mise en œuvre de sa requête de façon efficace. Cette infrastructure est proposée comme extension de notre intergiciel.
