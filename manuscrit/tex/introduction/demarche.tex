\section{Démarche et contribution de cette thèse}\label{sec:intro:demarche}
Nous proposons une approche orientée données pour l'observation de système. En particulier, nous avons choisi l'approche de la gestion des flux de données afin de : créer un langage algébrique unique d'interrogation pour toutes les catégories de données ; concevoir un intergiciel capable de gérer les données issus de l'observation de façon uniforme. Cette section présente brièvement nos contributions.

\subsection{Un langage unique d'interrogation}
La gestion de flux de données est un domaine permettant de gérer de manière déclarative les données temps-réel. Cette approche permet une grande souplesse car il est possible de déployer des requêtes complexes via un langage déclaratif sur les données. Toutefois, les langages manquent encore de fondations théoriques pour correctement maîtriser ses données.

Nous avons spécifié une algèbre permettant d'interroger de manière unifiée les données sous forme de flux ou relations de bases de données. Elle a pour propriété d'être capable de manipuler les deux modes d'interrogations sur des données archivés ou temps-réel. De plus, ses définitions sont entièrement déterministe ne laissant pas de place à l'interprétation, ce qui permet une gestion plus claire de ses données.

\subsection{Un intergiciel extensible d'évaluation de requête}
Il existe plusieurs intergiciels capables d'évaluer des requêtes continues sur les SGFD. Toutefois, leur implémentation peut influencer les sémantiques d'évaluations de requête. Des approches existent pour intégrer des évaluateurs d'un point de vue architectural. Toutefois, il existe peu d'approche pour spécifier formellement le comportement d'un opérateur d'un évaluateur.

Nous proposons un intergiciel capable d'évaluer des requêtes exprimés dans le langage algébrique que nous avons développé. Cette algèbre nous sert aussi pour la spécification des composants de l'intergiciel. En effet, chaque module implémentant un opérateur doit spécifier sa sémantique selon une ou plusieurs règles se basant sur des opérateurs algébriques. Ce principe couplé avec les notions architecturales de composants orientés services nous permettent d'avoir une grande flexibilité pour permettre aux utilisateurs de personnaliser au mieux cette solution.

De plus, notre approche à base de règle nous permet de développer une approche générale d'optimisation des requêtes. Tout comme en SGBD, nous appliquons une optimisation logique, puis physique de notre requête. Ces optimisations sont possibles grâce aux résultats d'équivalence de requête démontrable avec l'algèbre.

\subsection{Une intégration des supports persistants}
Enfin, nous avons vu que notre langage est capable d'interroger de manière unifié les données temps-réel et archivés. Dans la littérature, plusieurs existent entre les deux mondes de manière ad-hoc ou implicite.

Grâce à notre langage, nous proposons une infrastructure capable d'intégrer un SGBD à notre intergiciel. Ainsi, l'utilisateur doit spécifier son schéma contextuel précisant sa représentation du système, ainsi que des requêtes permettant l'intégration des données flux dans la base. À l'issue de ce travail de spécifications, l'utilisateur devient capable d'interroger les données archivés et temps-réel via l'algèbre. Il est alors garanti de la mise en œuvre de sa requête de façon efficace. Cette infrastructure est proposé comme extension de notre intergiciel.
