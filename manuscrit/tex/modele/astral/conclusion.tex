\section{Conclusion}\label{sec:contrib:astral:conclusion}
Dans ce chapitre, nous avons présenté un nouveau modèle de gestion de flux de données. Ce modèle permet d'obtenir une spécification claire et précise du résultat attendu pour une requête. Ainsi, une implémentation de système d'évaluation de requête peut décrire exactement son exécution grâce à ce modèle théorique.

Nous avons clarifié la gestion de l'ordre des n-uplets dans les flux. Ce point nous a conduits à reformuler certaines définitions classiques (telles que la jointure ou l'union) et à découvrir des comportements encore non détaillés dans la littérature (telle que l'asymétrie de ces opérateurs). Nous voyons que grâce à l'opérateur de manipulation temporelle, nous sommes capable d'exécuter des requêtes instantanées en figeant les relations temporelles. Enfin, nous avons défini strictement la notion d'équivalence de requête, autant à \textit{timestamp} de départ fixe, que différent. Ce qui nous permet par la suite de formuler mathématiquement des preuves.

Le chapitre suivant détaille l'expressivité de cette algèbre par la confrontation aux formalisations existantes et par la présentation de théorèmes d'équivalences de requêtes.