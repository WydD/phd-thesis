\section{Conclusion}\label{sec:contrib:astral:conclusion}
Dans ce chapitre, nous avons présenté un nouveau modèle de gestion de flux de données. Celui-ci est un modèle entièrement déterministe, car toutes les définitions permettent d'obtenir une spécification claire et précise du résultat attendu pour une requête. Ainsi, une implémentation de système d'évaluation de requête peut décrire exactement son exécution grâce à ce modèle théorique.

Si nous reprenons les points que nous avions décrits en section~\ref{sec:rw:sgfd:synthese}. Nous pouvons d'ores et déjà répondre à plusieurs des problèmes qui ont été identifiés. Nous avons clarifié la gestion de l'ordre des n-uplets dans les flux. Ce point nous a conduits à reformuler certaines définitions classiques (telles que la jointure ou l'union) et à découvrir des comportements encore non détaillés dans la littérature (telle que l'asymétrie de ces opérateurs). Nous avons défini strictement la notion d'équivalence de requête, autant à \textit{timestamp} de départ fixe, que différent. Ce qui nous permet par la suite de formuler mathématiquement des preuves exactes.

Quant à l'intégration des supports relationnels persistants, nous avons défini l'opérateur de manipulation temporelle applicable sur les relations temporelles. Ce qui permet, par exemple, de transformer une requête continue en requête statique. Ceci va être la clé du couplage possible avec les relations issues d'un système de gestion de base de données. Notre système d'évaluation de requête continue s'appuie fortement sur cette algèbre. Ainsi, nous détaillons d'abord la validation de ce modèle par l'analyse de son expressivité par rapport à l'état de l'art.
