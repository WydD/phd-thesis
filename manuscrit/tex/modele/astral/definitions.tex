\section{Définitions générales}\label{sec:contrib:astral:definitions}
Avant de présenter les définitions plus spécifiques à la gestion des flux de données, définissons les notions les plus élémentaires de notre algèbre. Nous présentons en premier les notions d'n-uplets et d'identifiants, puis celles de flux et relations. Nous continuons par la présentation d'un exemple fil rouge, qui permet d'illustrer les définitions de ce chapitre. Enfin, nous présentons les notions fondamentales de l'algèbre : les ordres, les requêtes et les équivalences de requêtes.

\subsection{N-uplets et identifiants}
La définition~\ref{def:tuple} présente un n-uplet de manière similaire au modèle relationnel avec une fonction partielle.
\begin{defi}[n-uplet]\label{def:tuple}
    Un n-uplet est une fonction partielle de l'ensemble des attributs vers l'espace des valeurs. Le domaine de cette fonction est appellé le schéma du n-uplet.
    
    Comme toute fonction partielle, un n-uplet peut être représenté comme un ensemble de couples $Attribut\times Valeur$.
\end{defi}

Ajoutons désormais la définition~\ref{def:phy} décrivant l'identifiant physique d'un n-uplet. Cet attribut particulier a deux buts : identifier un n-uplet et définir la position (définition~\ref{def:pos}) dans une séquence d'n-uplet (définition~\ref{def:seq}).
\begin{defi}[Identifiant physique]\label{def:phy}
    Nous appelons un $\Phi$-espace, un ensemble dénombrable totalement ordonné.

    L'identifiant physique d'un n-uplet $s$ est l'attribut $\varphi$ dont la valeur $s(\varphi)$ est un élément d'un $\Phi$-espace.
\end{defi}
\begin{defi}[Séquence d'n-uplet]\label{def:seq}
    Un ensemble dénombrable d'n-uplets $TS$ est une séquence si et seulement si : 
    \begin{itemize}
     \item Tout n-uplet de $TS$ partage le même schéma $A$ contenant $\varphi$.
     \item L'ensemble $\I_{TS}=\{s(\varphi), s\in TS\}$ forme un $\Phi$-espace.
     \item $\forall s,s' \in TS^2$, $s = s' \equ s(\varphi) = s'(\varphi)$
    \end{itemize}

    Une séquence est naturellement totalement ordonné par son identifiant physique.
\end{defi}
\begin{defi}[Position d'un n-uplet]\label{def:pos}
    La position d'un n-uplet $s$ dans une séquence $TS$ correspond au nombre d'n-uplet ayant un identifiant physique plus petit que $s$.
    $$\pos_{TS}(s) = \#\{s'\in TS,\ s'(\varphi) \leq s(\varphi) \}$$
\end{defi}

La gestion des flux de données nécessite de plus le concept de \textit{timestamp} (def~\ref{def:timestamp}) qui est associé aux n-uplets d'un flux. Nous adoptons un modèle continu pour le temps car cela nous permet de ne pas faire d'hypothèse sur la plus courte distance entre deux \textit{timestamps} consécutifs.
\begin{defi}[Timestamp]\label{def:timestamp}
    L'espace temps $\T$ est un corps totalement ordonné, isomorphique à $\R$. 

    Un \textit{timestamp} est un élément de $\T$ et son attribut est noté $\t$.
\end{defi}

\subsection{Flux et relations}
Comme présenté dans le chapitre~\ref{chap:rw:sgfd}, la notion de batch est introduite pour gérer les n-uplets simultanés (égalité de timestamp). La définition~\ref{def:flux}  définie un flux de donnée comme une séquence d'n-uplet organisés en une séquence de batch. Tous les n-uplets simultanés sont partitionnés en batch. Ceux-ci sont identifiés (def~\ref{def:batch}) par un entier identifiant le batch en plus du \textit{timestamp}.
\begin{defi}[Identifiant de batch]\label{def:batch}
    Un identifiant de batch est un élément de l'ensemble $\TN$ totalement ordonné par l'ordre naturel du produit cartésien.
\end{defi}
\begin{defi}[Flux]\label{def:flux}
    Un flux est un couple $(S,\BS)$ tel que :
    \begin{itemize}
        \item $S$ est une séquence d'n-uplet potentiellement infinie possédant un schéma contenant l'attribut spécial $\t$.
        \item $\BS$ est une fonction $S\to \TN$ définissant le batch d'appartenance d'un n-uplet
    \end{itemize}
    
    Par mesure de commodité, sauf mention contraire, nous écrivons $S$ pour désigner le couple $(S,\BS)$.
\end{defi}

Notre algèbre définie les deux concepts comme présenté dans la section~\ref{sec:rw:sgfd:modeles}. Nous définissons donc une relation temporelle (def~\ref{def:relation} comme une fonction associant le temps à un ensemble d'n-uplet. Dans notre cas, le temps est définit comme un identifiant de batch, et l'ensemble d'n-uplet est en réalité une séquence. 
\begin{defi}[Relation temporelle]\label{def:relation}
	Une relation temporelle $R$ est une fonction en escalier associant un identifiant de batch $(t,i)\in\TN$ à une séquence d'n-uplet $R(t,i)$
\end{defi}
L'évaluation d'une relation temporelle à un instant donnée est donc une séquence (ou \textit{relation instantannée}) avec donc un ordre strict sur les n-uplets. Par abus de langage, nous utiliserons le terme \textit{relation} pour désigner une relation temporelle.

La définition des relations temporelles et des flux implique qu'il existe une quantité dénombrable de \textit{batch} différents dans le flux ou dans une relation temporelle. Ainsi, pour tout \textit{batch} $(t,i)$, il existe un autre \textit{batch} proche de $(t,i)$ sans y être égal même si $t$ est assimilable à un réel\footnote{Au final, $(t,i)^- = (t,i-1)$ si $i\neq 0$ et $(t^-,0)$ sinon}. Ce type de \textit{batch} est noté $(t,i)^-$. Par exemple, pour une relation temporelle dont l'état change au \textit{batch} $b$, nous pouvons comparer les états $R(b)$ et $R(b^-)$ correspondant à avant et après le changement.

Par la suite, nous utilisons le terme \textbf{relation} pour relation temporelle. Les relations classiques sont mentionnées par l'appellation \textit{relation instantanée} ou par \textit{séquence d'n-uplet} si le contexte s'y prête. Enfin, nous appelons \textbf{entité} tout objet pouvant être une relation ou un flux.

\subsection{Exemple}\label{sec:contrib:astral:definitions:exemple}
Dans la suite de ce chapitre, nous développons plusieurs exemples. Ces exemples se baseront sur les entités suivantes issus de l'application du réseau local domestique.
\begin{itemize}
    \item[\textbf{CPU}] Flux(appId, cpu, $\t$) : flux de relevé des charges processeurs. Un n-uplet de ce flux indique qu'au timestamp $\t$, l'application \textit{appId} indique que sur son équipement hôte, la charge processeur est égale à \textit{cpu}.
    \item[\textbf{Applications}] Relation(appId, appName, deviceId) : catalogue des applications. Un n-uplet de cette relation indique que l'application \textit{appId} appelée \textit{appName} est déployée sur le l'équipement \textit{deviceId}.
    \item[\textbf{Devices}] Relation(deviceId, deviceName, deviceType, deviceStatus) : catalogue des équipements. Un n-uplet de cette relation indique que l'équipement \textit{deviceId} appelé \textit{deviceName} est de type \textit{deviceType} et est actuellement dans le status \textit{deviceStatus} (1 ou 0 pour allumé et éteint).
\end{itemize}

\subsection{Fondations de l'algèbre}
\subsubsection{Hypothèse fondamentale de l'ordre}
La notion d'ordre est crucial dans la gestion de flux de données. Nous avons introduit l'ordre naturel dans la séquence d'n-uplet. Ceci est l'ordre positionnel. Mais pour un flux, nous pouvons aussi définir l'ordre des \textit{batchs} et l'ordre temporel. La définition~\ref{def:ordres} spécifie l'expression formelle de ces ordres.
\begin{defi}[Ordres d'un flux]\label{def:ordres}
Soit $S$ un flux, trois ordres sont naturellement définis : $\forall s,s' \in S^2$,

$$\begin{array}{lcrcl} 
\textrm{L'ordre positionnel : } &\qquad & s \leq_\varphi s' & \equ & s(\varphi) \leq s'(\varphi)\\
\textrm{L'ordre des batch : } & & s \leq_{\B{}} s' &\equ& \BS(s) \leq \BS(s')\\
\textrm{L'ordre temporel : } & & s \leq_\t s' &\equ& s(\t) \leq s'(\t)\\
\end{array}$$
\end{defi}

Nous supposons maintenant que nos flux sont correctement ordonnés (hyp~\ref{hyp:ordres}). Si un n-uplet est positionnellement avant un autre, alors son batch est inférieur ou égal. Et par conséquent son \textit{timestamp} est lui aussi inférieur ou égal.
\begin{hyp}[Cohérence temporelle]\label{hyp:ordres}
L'ordre positionnel implique l'ordre des batchs.
$$\forall s, s'\in S^2, \qquad s <_\varphi s' \im s \leq_{\B{}} s'$$
\end{hyp}

\subsubsection{Requêtes}
Tout d'abord, il nous faut définir le fait que dans la gestion de flux de données, une entité n'a pas d'existence avant la création de son instance à $t_0$. En effet, lorsqu'une requête est déployée sur un système, sauf opération explicite, le flux arrivant n'a pas d'informations sur les données passées. D'un point de vue algébrique, une entité est initialisée (def~\ref{def:init}) à un \textit{timestamp} donné si toutes ses données sont accessibles pour un \textit{timestamp} supérieur à celui-ci.
\begin{defi}[Entité initialisée]\label{def:init}
	Une entité $E$ initialisée à un timestamp $t_0$ est une entité vérifiant :
	\begin{itemize}
		\item Si $E$ est une relation : $\forall b \in \TN$, tel que $b<(t_0,0)$, alors $E(b) = \emptyset$.
		\item Si $E$ est un flux : $\forall s\in E$, $s(\t) \geq t_0$.
	\end{itemize}
\end{defi}

La notion de requête s'appuie directement sur cette notion d'entité initialisée. Ainsi, la requête (def~\ref{def:requete}) est une composition d'opérateurs appliqués sur ces entités initialisées. Il est important de voir que les opérateurs sont potentiellement dépendants du timestamp d'initialisation.
\begin{defi}[Requête]\label{def:requete}
	Soient $E=(E_1, ..., E_n)$, $n$ entités,
	
	Une requête continue sur $E$ démarrée au temps $t_0$ est défini comme une composition d'opérateurs $Q$ appliqués sur les éléments de $E$ initialisés à $t_0$. Cette requête est noté $(Q(E),t_0)$.
\end{defi}

\subsubsection{Équivalences de requêtes}
La séquence d'n-uplet est le type central de notre algèbre car elle seule établit la notion d'ensemble d'n-uplets. Nous définissons l'inclusion de séquences (def~\ref{def:inclusion}) de manière similaire à la notion de suites extraites. La notion d'\textbf{équivalence} de deux séquences est donc définie naturellement par la double inclusion. 
\begin{defi}[Inclusion et équivalences de séquences d'n-uplets]\label{def:inclusion}
	Soient $TS_1$ et $TS_2$ deux séquences possédant le même schéma $A$,
	
	$TS_1$ est inclue dans $TS_2$ ($TS_1 \subseteqq TS_2$) si et seulement si : 
	
	\quad \quad $\exists f:\I \mapsto \I$ strictement croissante telle que 
$$\forall s\in TS_1, \exists s'\in TS_2, \textrm{ tel que } \forall a \in A, s'(a) = \begin{cases} f(s(\varphi)) & a = \varphi \\ s(a) & \textrm{sinon}\end{cases}$$
	L'équivalence ($TS_1 \equiv TS_2$) est définie par $TS_1 \subseteqq TS_2$ et $TS_1 \supseteqq TS_2$. 
\end{defi}%


\begin{example}
La table~\ref{tab:contrib:astral:exequivalence} représente quatre exemple de séquences d'une instance figée de la relation temporelle \textbf{Device}. Ces séquences pourraient être équivalentes d'un point de vue algèbre relationnelle (en substituant $\varphi$).

Les séquences (a) et (b) sont équivalentes car leurs identifiants physiques sont différentes mais représentent le même ordre des n-uplets. Selon la définition, nous pouvons effectivement trouver une fonction $f$ strictement croissante pour transformer les $\varphi$ de (a) en (b) et inversement. 

Les séquences (a) et (c) ne sont pas équivalentes car l'ordre des n-uplets est inversé. Il est en effet impossible de trouver une transformation $f$ de $\varphi$ \textbf{croissante} telle que l'égalité des n-uplets soit exacte. Nous remarquons que la valeur de $\varphi$ n'a que peut d'importance car même (b) et (d) ne sont pas équivalentes malgré l'égalité notoire des identifiants. Bien évidemment, (c) et (d) sont équivalentes.
\end{example}

\begin{table}[ht]
\centering
(a)
\begin{tabular}{c|ccc} 
      $\varphi$ & id & name & type \\ \hline 
       1  & 1 & Livebox & Passerelle \\
       5  & 4 & hecate & PC\\
       12 & 3 & iPad & Tablette\\
\end{tabular}
\hspace{1cm}
(b)
\begin{tabular}{c|ccc} 
      $\varphi$ & id & name & type \\ \hline 
       7  & 1 & Livebox & Passerelle \\
       9  & 4 & hecate & PC\\
       10 & 3 & iPad & Tablette\\
\end{tabular}\\
(c)
\begin{tabular}{c|ccc} 
      $\varphi$ & id & name & type \\ \hline 
       3  & 4 & hecate & PC\\
       6  & 1 & Livebox & Passerelle \\
       11 & 3 & iPad & Tablette\\
\end{tabular}
\hspace{1cm}
(d)
\begin{tabular}{c|ccc} 
      $\varphi$ & id & name & type \\ \hline 
       7  & 4 & hecate & PC\\
       9  & 1 & Livebox & Passerelle \\
       10 & 3 & iPad & Tablette\\
\end{tabular}
\caption{Quatre séquences représentant une instance figée de \textbf{Device}}\label{tab:contrib:astral:exequivalence}
\end{table}

La notion d'inclusion et d'équivalence s'étend naturellement aux flux et aux relations. Nous pouvons maintenant définir proprement la notion d'équivalence de requêtes comme l'équivalence de l'entité résultante quelques soient les entités d'entrées. Il est important de noter que l'équivalence suppose que les requêtes ont démarré au même \textit{timestamp} $t_0$. La section~\ref{sec:contrib:astral:transposabilite} présente les conséquences du changement de timestamp.
\begin{defi}[Équivalences de requêtes]\label{def:equivalence}
	Soient $E=(E_1, ..., E_n)$, $n$ entités,
	
	Les requêtes $Q$ et $Q'$ démarrées à $t_0$ sont équivalentes si et seulement si pour toutes entités $E'$ de même types et schéma, les entités $Q(E')$ et $Q'(E')$ sont équivalentes.
\end{defi}



Nous venons de poser les bases fondamentales pour notre algèbre de gestion de données. Il nous faut maintenant construire des opérateurs pour pouvoir manipuler nos concepts. Nous allons voir que les contraintes que nous nous sommes fixés sur l'ordre, la nature du temps.
